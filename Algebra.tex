\documentclass[10pt,a4paper]{article}
\usepackage[utf8]{inputenc}
\usepackage[german]{babel}
\usepackage{amsmath}
\usepackage{amsfonts}
\usepackage{amssymb}
\usepackage{amsthm}
\usepackage{graphicx}
\usepackage{tikz,pgf}
\usepackage{mathrsfs}
\usepackage{framed}
\usepackage{hyperref}
\usepackage[all]{xy}

\newcommand{\N}{\ensuremath{\mathbb{N}}}
\newcommand{\Z}{\ensuremath{\mathbb{Z}}}
\newcommand{\Q}{\ensuremath{\mathbb{Q}}}
\newcommand{\R}{\ensuremath{\mathbb{R}}}
\newcommand{\C}{\ensuremath{\mathbb{C}}}
\newcommand{\F}{\ensuremath{\mathbb{F}}}
\newcommand{\la}{\ensuremath{\lambda}}
\newcommand{\al}{\ensuremath{\alpha}}
\newcommand{\ol}[1]{\overline{#1}}
\newcommand{\ul}[1]{\underline{#1}}
\newcommand{\norm}[1]{\|#1\|}
\newcommand{\Hlb}{\ensuremath{\mathscr H}}
\newcommand{\surjfunc}{\ensuremath{\xrightarrow{\sim}}}
\newcommand{\Hom}{\text{Hom}}
\newcommand{\Ker}{\ensuremath{\operatorname{Ker}}}
\newcommand{\Img}{\ensuremath{\operatorname{Im}}}
\newcommand{\Qud}{\text{Qud}}
\newcommand{\inj}{\xrightarrow{\sim}}
\newcommand{\Spec}{\ensuremath{\operatorname{Spec}}}

\newenvironment{sumup}{\begin{framed}\textbf{Zusammenfassung:}\\}{\end{framed}}


\newcounter{thm}[section]
\theoremstyle{definition}
\newtheorem{definition}[thm]{Definition}
\newtheorem{satz}[thm]{Satz}
\newtheorem{prop}[thm]{Proposition}

\theoremstyle{plain}
\newtheorem{lem}[thm]{Lemma}
\newtheorem{kor}[thm]{Korollar}

\theoremstyle{remark}
\newtheorem{bem}[thm]{Bemerkung}
\newtheorem{rem}[thm]{Erinnerung}
\newtheorem{exm}[thm]{Beispiel}


\author{Prof Wedhorn, Mitschrift von Daniel Kallendorf}
\title{Algebra SS16}

\begin{document}
	\maketitle
	\tableofcontents
	\setcounter{section}{2}
%	\setcounter{theorem}{6}
	\begin{bem}
		$A[X_1,...,X_n]$ ist ein freier $A$-Modul, wobei die Monome eine Basis bilden.
	\end{bem}
	\begin{satz}[Universaleigenschaft des Polynomrings]
		Sei $\phi:A\rightarrow B$ eine $A$-Algebra und seine $b_1,...,b_n\in B$ Elemente. Dann existiert genau ein $A$-Algebra-Homomorphismus $\psi:A[X_1,...,X_n]\rightarrow B$, so dass $\psi(x_i)=b_i$ für alle $i=1,..,n$, nämlich
		\[\psi\underbrace{\left(\sum_{i_1,...,i_n\ge 0}a_{i_1,...,i_n}X_1^{i_1}\cdot...\cdot X_n^{i_1}\right)}_{=:f}=\underbrace{\sum_{i_1,...,i_n\ge 0}\phi(a_{i_1,...,i_n})b_1^{i_1}\cdot...\cdot b_n^{i_n}}_{=f(b_1,...,b_n)}\]
	\end{satz}
	\begin{bem}
		\begin{align*}
		\operatorname{Im}(\psi)&=\text{kleinste $A$-Unteralgebra die $b_1,...,b_n$ enthält}\\
		&=A[b_1,...,b_n]\subset B
		\end{align*}
	\end{bem}
	\begin{exm}
		Sei $\phi:A\rightarrow B$ eien $A$-Algebra, $b\in B$. Es existiere ein $g\in A[X]$ mit $g(b)=0$. Sei $g$ nomriert. Dann gilt\\
		\[A[b]=\{f(b)|f\in A[x],\deg(f)<\deg (g)\}\]
	\end{exm}
	\begin{exm}
		Sei $A=\Q\hookrightarrow\C$,$i\in \C$. \\
		Dann gilt $g(i)=0$ wobei $g=X^3+X=X(X^2+1)$. Es folgt:
		\begin{align*}
		\Q[i]&=\{a_0+q_1i+a_2i^2|a_0,a_1,a_2\in\Q\}\\
		\Q[i]&=\operatorname{Im}(\Q[X]\xrightarrow[X\mapsto i, f\mapsto f(i)]{\psi}\C)
		\end{align*}
		Dann $\tilde{g}\in\Q[X]: \psi(\tilde{g})=0\Leftrightarrow \tilde{g}(i)=0$.\\
		Also $g\in\operatorname{Ker}(\psi)\Rightarrow (g)\subseteq\operatorname{Ker}(\psi)$.\\
	In diesem Fall $\operatorname{Ker}\psi=(X^2+1)$.\\
	\end{exm}
	Begründung von 2.8:
	\[(g)\subseteq\operatorname{Ker}\left(A[X]\xrightarrow[f\mapsto f(b)]{\psi}B\right)\]
	Also $\psi$ faktorisiert:
	\[A[X]/(g)\xrightarrow{\ol{\psi}}A[b]\subseteq B\]
	mit $\ol{\psi}$ surjektiv.
	\begin{prop}
		Sei $g\in A[X]$ normiert. Dann ist\[\{f\in A[X],\deg(f)<\deg(g)\}\hookrightarrow A[X]\rightarrow A[X]/(g)\]
		bijektiv.
	\end{prop}
	\begin{proof}
		Gilt, da für alle $f\in A[X]$ genau ein $r\in A[X]$ exitiert mit $\deg(r)<\deg(g)$ mit $f\in r+(g)$
		%TODO
	\end{proof}
\section{Tensorprodukte}
(A) Tensorprodukte von Moduln\\
(B) Tensorprodukte von Algebren und Basiswechsel\\
(C) Exaktheitseigenschaften des Tensorprodukts\\
\subsection{Erinnerung}
\begin{definition}
	$A$-Modul$:=(M,+,\cdot)$ wobei $(M,+)$ abelsche Gruppe und $\cdot:A\times X\rightarrow M$ ein Skalarprodukt.
\end{definition}
\begin{bem}
	$\Z$-Modul=ablesche Gruppe
\end{bem}
\begin{exm}
	Sei $I$ eine Menge
	\[A^{(I)}=\{(a_i)_{i\in I}|a_i\in A, a_i=0\text{für fast alle $i\in I$}\}\]
	$A$-Modul mit Addition und Skalarprodukt.\\
	Für $i\in I:e_i\in A^{(I)}$ mit
	\[e_i=\begin{cases}
	\text{1 an der i-ten Stelle}\\
	\text{0 sonst}
	\end{cases}\]
\end{exm}
\begin{definition}
	Ein $A$-Modul heißt frei, falls $M\approxeq A^{(I)}$ für eine Menge $I$
\end{definition}
\begin{definition}
	Sei $M,N$ $A$-Modul. Dann heißt $u:M\rightarrow N$ A-linear oder Homomorphismus von $A$-Moduln, falls
	\[u(am+m')=au(m)+u(m')\forall a\in A,m,m'\in M\]
\end{definition}
\begin{bem}
	Sei $I$ eine Menge, $M$ ein $A$-Modul $\underline{m}=(m_i)_{i\in I}$ ein Tupel von Elementen $m_i\in M$. Dann Existiert genau eine Abbildung:
	\[A^{(I)}\xrightarrow{u_{\underline m} }M\]
	mit $u_{\underline{m}}(e_i)=m_i$.\\
	$(m_i)_i=\underline{m}$ heißt linear Unabhängig/ Erzeugende-System/ Basis, \\
	falls $u_{\underline{m}}$ injektiv/ surjektiv / bijektiv ist.
\end{bem}
\begin{bem}
	Der $A$-Modul M ist endlich erzeugt, genau dann wenn ein $n\in \N$und eine $A$-lineare Surjektion $A^m\rightarrow M$ existieren.
\end{bem}
\subsection{Multilineare Abbildungen}
\begin{definition}
	Sei $r\in \N_0$, $M_1,...,M_r,P$ A-Moduln.\\
	Eine Abbildung $\al:M_1\times...\times M_r\rightarrow P$ heißt \underline{r-multilinear}, falls sie in jeder Komponente linear ist, d.h. Für alle $i=1,...,r$ gilt:
	\[\al(m_1,...,am_{i}+m_i',m_{i+1},...,m_r)=a\al(m_1,...,m_i,...,m_r)+\al(m_1,...,m_i',...,m_r)\]
	Für alle $m_j\in M_j,m_i\in M_i,a\in A$.
	($r=1$: linear, $r=2$: bilinear)
\end{definition}
%TODO Setze...
\subsection{..}
\begin{definition}
	Sei $r\ge 2$, $M_1,..,M_r$ A-Moduln.\\
	Dann existiert ein $A$-Modul $M_1\otimes_AM_2\otimes_A...\otimes_AM_r$ und eine $r$-multilineare Abbildung $\tau:M_1\times...\times M_r\rightarrow M_1\otimes_AM_2\otimes_A...\otimes_AM_r$, sodass für jede $r$-multilineaer Abbildung:
	\[\al M_1\times...\times M_r\rightarrow P\]
	wobei $P$ ein A-Modul, genau ein A-lineare Abbildung 
	\[\ol\al:M_1\otimes_A...\otimes_AM_r\rightarrow P\]
	existiert.
	$\xymatrix{
		M_1\times...\times M_r \ar[r]^{\forall\al: \text{r-multilinear}} & P\\
		M_1\otimes_AM_2\otimes_A...\otimes_AM_r&
	}$
\end{definition}
%VL 31.10.2016
\begin{satz}[Eindeutigkeit des Tensorprodukts]
	Seien $(T,\tau:M_1\times...\times M_r\rightarrow T)$ und $(T',\tau')$ Tensorprodukte:
	\[\xymatrix{ 
		M_1\times...\times M_r \ar[d]^{\tau} \ar[dr]^{\tau'}&\\
		T\ar[r]_{\exists!v}&T'
	}\]
	$u$ existiert aufgrund der universellen Eigenschaft von $(T,\tau)$.\\
	$v$ existiert aufgrund der universellen Eigenschaft von $(T',\tau')$.\\
	Ferner kommutiert\\
	
	Die Universelle Eigschaft von $(T,\tau)$ zeigt, dass $v\circ u=id_T$, genauso $u\circ v=id_T$.
\end{satz}
\begin{satz}[Existenz des Tensorprodukts]
	\begin{enumerate}
		\item Suche einen $A$-Modul $N$ und eine Abbildung $c:M_1\times...\times M_r\rightarrow R$, sodass
		\[\Hom_A(N,P)\xrightarrow[u\mapsto u\circ\tau]{ }\text{Abb}(M_1\times...\times M_r,P)\]
		Für alle $A$-Moduln $P$.
		%TODO
		\item Wir wollen, dass $(am_1+m_1',m_2,...,m_r)$ und $a(m_1,...,m_r)+(m_1',...,m_r)$ auf das gleiche Element abgebildet werden.\\
		Sei $Q\subseteq N$ der von
		\begin{align*}
		e_{(m_1,...,m_{i-1},am_i+m_i',m_{i+1},...,m_r)}-\left(ae_{(m_1,...,m_i,..,m_r)}+e_{(m_1,...,m_i',...,m_r)}\right)
		\end{align*}
		für alle $i=1,...,r$ und $m_i,m_i'\in M_i$ und $a\in A$ erzeugt Untermodul.\\
		Dann setze $T:=N/Q$. Dann gilt
		\begin{align*}
		\Hom_A(T,P)&=\{u\in \Hom(N,P)|u(Q)=0\}\\
		&=L_A(M_1,...,M_r,P)
		\end{align*}
		mit $\tau:M_1\times...\times M_r\rightarrow N\rightarrow N/Q$.
	\end{enumerate}
\end{satz}
\begin{bem}
	3.4\\
	$e_{(m_1,...,m_r)}\in A^{(M_1\times...\times M_r)}$ bilden ein Erzeugndensystem.\\
	Also bilden auch die $\tau(m_1,...,m_r)=:m_1\otimes...\otimes m_r$ eine Erzeugenden-System des $A-$Moduls $M_1\otimes...\otimes M_r$.\\
	\textbf{Aber:} Nicht jedes Element von $M_1\otimes...\otimes M_r$ ist in dieser Form.\\
	\\
	Also genüt es eine lineare Abbildung $u:M_1\otimes...\otimes M_r\rightarrow P$ auf den erzeugdnesn $m_1\otimes...\otimes m_r$ mit ($m_i\in M_i$) anzugeben.\\
	Umgekehrt sei $P$ ein A-mOdul und es seien elemente $u(m_1\otimes...\otimes m_r)\in P$ gegeben für alle $m_i\in M_i$.\\
	Genau dann existiert eine $A$-lineare Abbildung $u:M_1\otimes...\otimes M_r\rightarrow P$ mit $m_1\otimes ...\otimes m_r\mapsto u(m_1\otimes ...\otimes m_r)$, wenn für alle $i=1,...,r$, $a\in A$, $m_j\in M_j$ und $m_i'\in M_i$ gilt:
	\[u(m_1\otimes ..\otimes a m_i+m_i'\otimes..\otimes m_r)=a u(m_1\otimes ..\otimes m_i\otimes..\otimes m_r)+u(m_1\otimes ..\otimes a m_i'\otimes..\otimes m_r)\]
\end{bem}
\begin{satz}[Tensorprodukt linearer Abbildungen]
	Seien $M,M',N,n'$ $A$-Moduln, $u:M\rightarrow M',v:N\rightarrow N'$ $A$-lineare Abbildungen.\\
	Dann definiert
	\begin{align*}
	M\otimes_A N&\rightarrow M'\otimes A N'\\
	m\otimes n &\mapsto u(m)\otimes u(n)
	\end{align*}
	eine $A$-lineare Abbildung bezüglich $u\otimes v:M\otimes N\rightarrow M'\otimes N$.
\end{satz}
\begin{proof}
	Zu zeigen: $u(am+m')\otimes v(n)=a(u(m)\otimes v(n))+u(m')\otimes v(n)$\\
	Es gilt da das Tensorprodukt $r$-linear ist.
	\begin{align*}
	u(am+m')\otimes v(n)&=(au(m)+u(n))\otimes v(n)\\
	&=(au(m)\otimes v(n))+u(m')\otimes v(n)
	\end{align*}
	\\
	Außerdem zu zeigen: $u(m)\otimes v(an+n')=a(u(m)\otimes v(n))+u(m)\otimes v(n)$\\
	($\rightarrow $ Genauso.)
\end{proof}
\begin{bem}
	3.6\\
	\begin{enumerate}
		\item $A\otimes_A M\approxeq M$\\
		$u: a\otimes m\mapsto am$\\
		$v: 1\otimes m.....m $
		Dabei ist $u$ wohldefiniert, d.h. $(a,m)\rightarrow am$ ist bilinear.
		%TODO
		\item $M\otimes_A N\xrightarrow{\sim}N\otimes_A M, m\otimes n\mapsto n\otimes m$ ist ... von A-Moduln.\\
		Zu zeigen: Wohldefineirtheit 
		%TODO
		\item $M\otimes_A N\otimes_A P\simeq (M\otimes_A N)\otimes_A P$\\
		$m\otimes n\otimes p\mapsto (m\otimes n)\otimes p$\\
		$m\otimes n\otimes p\mapsto m\otimes (n\otimes p)$
	\end{enumerate}
\end{bem}
\begin{prop}3.7
	Sei $(M_i)_{i\in I}$ eine Familie von $A$-Moduln, $N$ ein A-Modul:\begin{align*}
	\left(\bigotimes_{i\in I}M_i\right)\otimes_A N\xrightarrow{\sim} \bigotimes_{i\in I}\left(M_1\otimes_A N\right)\\
	(m_i)_{i\in I}\otimes n\mapsto(m_i\otimes n)_{i\in I}
	\end{align*}
\end{prop}
\begin{proof}
	Umkehrabbildung gegeben durch:\[Inhalt..m_i\otimes n\mapsto (m_j)_{j\in I}\otimes n\]
	mit $m_j:=\begin{cases}
	m_i, &j=i\\
	0 &j\neq i
	\end{cases}$
\end{proof}
\subsection{Basiswechsel von Tensorprodukten}
\begin{satz}
	\begin{enumerate}
		\item Sei $M$ ein A-Modul. Dann wird
		\[\varphi^*(M):=B\otimes_A M\]
		zu einerm $B$-Modul mit dem Skalarprodukt
		\begin{align*}
		B\times(B\otimes_A M)&\rightarrow B\otimes_A M\\
		(b,b'\otimes m)&\mapsto bb'\otimes m
		\end{align*}
		\item Sei $U:M\rightarrow M'$ ein Homomorphismus von A-Moduln. Dann ist
		\begin{align*}
		id_B\otimes u:B\otimes M&\rightarrow  B\otimes_A M'\\
		b\otimes m\mapsto b\otimes u(m)
		\end{align*}
		eine B-lineare Abbildung.S
	\end{enumerate}
\end{satz}
\begin{prop}
	Sei $\varphi:A\rightarrow B$ eine A-Algebra.\\
	Sei $M$ ein freier A-Modul. Dann ist $B\otimes_A M$ ein freier B-Modul und
	\[\vartheta_A(M)=\vartheta_B(B\otimes_A M)\]
\end{prop}
\begin{proof}
	Sei $M$ ein freier A-Modul. Dazu ist äquivalent, dass $M\simeq A^{(I)}$.\\
	Daraus folgt, dass
	\begin{align*}
	B\otimes_A M&\simeq B\otimes_A A^{(I)}\\
	&\simeq B\otimes_A\left(\bigoplus_{i\in I}A\right)\\
	&\simeq \left(\bigoplus_{i\in I}B\otimes_A A\right)\\
	&\simeq\bigoplus_{i\in I}B\\
	&=B^{(I)}
	\end{align*}
	Also ist $B\otimes_A M$ frei.
	%TODO
\end{proof}
%VL 02.11.2016
\begin{prop}
	Sei $\mathfrak{a}\subseteq A$ ein Ideal, $M$ ein A-Modul.Setze 
	\begin{align*}
	\mathfrak a\cdot M&=\langle \{am|a\in\mathfrak a,m\in M\}\\
	&=\left\{\sum_{i=1}^{m}a_im_i\mid n\in\N_0,a_i\in\mathfrak a,m_i\in M \right\}\\
	&\subseteq M \quad \text{Untermodul}
	\end{align*}
	Dann ist
	\begin{align*}
	A/\mathfrak a\otimes_A M&\xrightarrow{\sim} M/\mathfrak a M\\
	\ol a\otimes m&\mapsto \ol{am}
	\end{align*}
	ein Homomorphismus von $A/\mathfrak a$-Moduln.
\end{prop}
\begin{proof}$\ol a\oplus m\mapsto \ol{am}$ ist wohldefiniert:
	Zu zeigen:
	\begin{enumerate}
		\item Sei $a'\in A$ mit $\ol{a'}=\ol a\in A/\mathfrak a$.\\
		Dann ist $\ol{am}=\ol{a'm}\in M/\mathfrak aM$.
		Es gilt $\ol{a}'=\ol a$ gena dann wenn es ein $x\imath\mathfrak a$ gibt sodass $a'=a+x$.\\
		Daruas folgt, dass $a'm=am+xm$, und da $xm\in\mathfrak aM$ folgt $\ol{a'm}=\ol{am}$.
		\item $\ol{am}$ is linear in $a$, d.h.
		\[\ol{(ba+a')m}=b\ol{am}+a'\ol m \quad \text{für $a,a'\in A$, $b\in A$}\]
		\item $\ol{am}$ ist linear in $m$, d.h.
		\[\ol{a(bm+m')}=b\ol{am}+\ol{am'}\quad\text{für $m,m'\in M$, $b\in A$}\]
	\end{enumerate}
\end{proof}
\begin{prop}
	Eine Umkehrabbildung ist gegeben durch
	\begin{align*}
	v:M&\rightarrow A/\mathfrak a\otimes_A M\\
	m&\mapsto 1\otimes m
	\end{align*}
\end{prop}
\begin{proof}
	Zu zeigen: $\mathfrak aM\subseteq Ker(v)$, also für alle $x\in\mathfrak a,m\in M$ gilt $v(xm)=0$.
	\[v(xm)=1\otimes xm=\ol{x}\otimes m=0\]
	da $\ol{x}=\ol{0}\in A/\mathfrak a$.\\
	Noch zu zeigen:: $v$ ist Umkehrabbildung zu $\ol a\otimes m\mapsto \ol{am}$.
\end{proof}
\begin{definition}[Tensorprodukte von Algebren]
	Sei $A\rightarrow  B_1$, $A\rightarrow  B_2$ A-Algebren.\\
	Dann definieren wir auf dem A-Modul $B_1\otimes_A B_2$ eine Multiplikation:
	\begin{align*}
	(B_1\otimes B_2)\times(B_1\otimes B_2)&\rightarrow B_1\otimes B_1\otimes B_2\\
	(a_1\otimes b_2,b_1'\otimes b_2')&\mapsto b_1b_1'\otimes b_2b_2'
	\end{align*}
	und erhalten die $A$-Algebra $B_1\otimes_A B_2$.
\end{definition}
\begin{exm}
	Sei $A\xrightarrow{\varphi} B$ eine A-Algebra und sei $C=A[X_1,...,X_n]/(f_1,...,f_r)$ und $f_i\in A[X-1,...,X_n]$.Dann ist
	\[B\otimes_A A[X-1,...,X_n]/(f_1,...,f_r)=B[X_1,...,X_n]/(\tilde{f}_1,...,\tilde{d}_r)\]
	wobei 
	\[f_i=\sum_{\ul{j}\in \N_0^n}a_{\ul{j}}X^{\ul{j}}\rightarrow \tilde{f}_i=\sum_j\varphi(a_j)\]
	\begin{enumerate}
		\item Sei $A=\Q$, $C=\Q[i]=\{a+b_i|a,b\in\Q\}=\Q[X]/(X^2+1)$
		\item $\R\otimes_Q Q[i]=\R[X]/(X^2+1)=\C$
		\item $C\otimes_Q Q[i]=C[X]/(X^2+1)=\C[X]/(X+i)\times\C[X]/(X-i)\simeq\C\times\C$
	\end{enumerate}
\end{exm}
\begin{exm}
	$A[X]\otimes_A A[Y]=(A[X])[Y]=A[X,Y]$ mit $f\otimes g\mapsto fg$.\\
	Dann ist die Umkehrabbildung
	%TODO
\end{exm}
\subsection*{C) Exaktheitseigenschaften}
\begin{definition}[Homomorphismen-Funktor]
	Seien $M,P$ A-Moduln.\\
	Wir Definiere auf $\Hom_A(M,P):=\{u:M\rightarrow P \text{A-linear}\}$ die Struktur eines $A$-Moduls.
	\begin{align*}
	(u+v)(m)&:=u(m)+v(m) &u,v\in\Hom_A(M,P)\\
	(au)(m)&:=au(m)		 &a\in A,m\in M
	\end{align*}
	Sei $u:M\rightarrow M'$ eine A-lineare Abbildung. Wir erhalten die A-lineare Abbildung
	\begin{align*}
	\Hom_A(u,P):\Hom_A(M',P)&\rightarrow \Hom_A(M,P)\\
	w'&\mapsto w'\cdot u
	\end{align*}
	Sei $v:P\rightarrow P'$ eine A-lineare Abbildung. Wir erhalten die A-lineare Abbildung
	\begin{align*}
	\Hom_A(M,v):\Hom_A(M,P)&\rightarrow \Hom_A(M,P')\\
	w'&\mapsto v\cdot w
	\end{align*}
\end{definition}
\begin{rem}
	Eine Sequnez von $A$-lineare Abbildungen\[...\rightarrow M_{i-1}\xrightarrow{u_{i-1}}M_i\xrightarrow{u_{u_i}}M_{i+1}\rightarrow ...\]
	heißt exakt, falls $\text{Ker}(u_i)=\text{Im}(u_{i-1})$
\end{rem}
\begin{exm}
	$0\rightarrow M*\xrightarrow{u}M$ ist exakt genau dann wenn $u$ injektiv ist.\\
	$M\xrightarrow{v}M''\rightarrow 0$ ist exakt genau dann wenn $v$ surjektiv ist
\end{exm}
\begin{satz}
	\begin{enumerate}
		\item Sei $0\rightarrow M'\xrightarrow{u}M\xrightarrow{v}M''(*)$ eine Sequenz von $A-$Moduln.\\
		Dann ist $(*)$ genau dann exakt, wenn für jeden A-Modul $P$ die Sequenz
		\begin{align*}
		\Hom_A(P,(*)):0\rightarrow \Hom_A(P,M')&\rightarrow \Hom_A(P,M)&&\rightarrow \Hom_A(P,M'')\\
		w'&\mapsto u\circ w'&w&\mapsto v\circ w
		\end{align*}
		exakt ist.
		\item %TODO
	\end{enumerate}
\end{satz}
\begin{proof}Wir beweisen Schrittweise:
	\begin{enumerate}
		\item ``$(*)$ ist exakt $\Rightarrow$ $\Hom_A(P,(*))$ ist exakt``
		\begin{enumerate}
				\item $w'\mapsto u\circ w'$ injektiv:\\
				Sei $w\in\Hom_A(P,M')$ mit $u\circ w'=0$.\\
				Dann ist (da $u$ injektiv) $w'=0$. Also ist $\Ker(w'\mapsto u\circ w')=0.$
				\item $\Img(w'\mapsto u\circ w')\subseteq\Ker(w\mapsto v\circ w)$:\\
				Kompoosition: $w'\mapsto u\circ w'\mapsto \underbrace{(v\circ u)}_{=0}\circ w'$ ist Null.
				\item $\Img(w\mapsto v\circ w)\subseteq\Ker(w'\mapsto u\circ w')$:\\
				Sei $w\in\Hom_A(P,M)$ mit $v\circ w=0$, sodass $\Img(w)\subseteq\Ker(v)=\Img(u)$.
			\end{enumerate}
			\begin{center}
				\xymatrix{
				0\ar[r]&M'\ar[r]^u&M\ar[r]^v&M''\\
				&&P\ar@{..>}[lu]^{w'?}\ar[u]_w\ar[ru]_0&
				}
			\end{center}
			``$\Leftarrow$``
			\begin{enumerate}
				\item $u$ injektiv: Sie $m'\in M$ mit $u(m')=0$, $P:=<m'>=Am'\subseteq M',w':P\rightarrow M'$ Inklusion.\\
				Dann ist...
				%TODO Rest der VL
			\end{enumerate}
	\end{enumerate}
\end{proof}
%VL 07.11.2016
\begin{bem}
	\label{bem313}
	Seiene $M,N,P$ A-Moduln. Dann ist
	\begin{align*}
	\Hom_A(M\otimes_A N,P)&=L_A(M,N;P)&\quad (*)\\
	&=\Hom_A(M,\Hom_A(N,P))\\
	(\al:M\times N\rightarrow P)&\mapsto (n\mapsto \al(m,n))
	\end{align*}
	\begin{align*}
	\text{Sei }T_N:(\text{A-Modul})&\rightarrow (\text{A-Modul})\\
	M&\mapsto M\otimes_AN\\
	(u:M\rightarrow M')&\mapsto u\otimes id_N\\
	N_N:(\text{A-Modul})&\rightarrow \text{(A-Modul)}\\
	P&\mapsto \Hom_A(N,P)
	\end{align*}
	Dann besagt $(*)$:
	\[\Hom(T_M(M),P)=\Hom(M,H_N(P))\]
	d.h. $T_N$ ist linksadjungiert zu $H_N$.\\
	Dann ist $T_N$ rechtsexakt und $H_N$ ist linksexakt.
\end{bem}
\begin{prop}
	Sei $M'\xrightarrow{u}M\xrightarrow{v}M''\rightarrow 0$ eine exakte Sequenz von A-Moduln. Dann ist für jeden A-Modul $N$ die Sequenz
	\[M'\otimes N\xrightarrow{u\otimes id_N}M\otimes_AN\xrightarrow{u\otimes id_N}M''\otimes_A N\rightarrow 0\]
	exakt.
\end{prop}
\begin{proof}
	Formal mit \ref{bem313}.\\
	Sei $M'\rightarrow M \rightarrow M''\rightarrow 0$ exakt.\\
	Dann gilt mit $\ref{bem312}$, dass für alle A-Mdouln $P$:
	\[0\rightarrow \Hom_A(M'',H_N(P))\rightarrow \Hom_A(M,H_N(P))\rightarrow \Hom_A(M',H_N(P))\]
	Ist jeweils gleich (\ref{bem313})
	\[0\rightarrow \Hom_A(T_N(M''),P)\rightarrow \Hom_A(T_N(M),P)\rightarrow \Hom_A(T_N(M'),P)\]
	exakt, sodass mit \ref{312}
	\[T_N(M')\rightarrow \underbrace{T_N(M)}_{=M\otimes_AN}\rightarrow T_N(M'')\rightarrow 0\]
	exakt ist.
\end{proof}
\begin{exm}
	Sei $A=\Z$, $u:\Z\xrightarrow{x\mapsto2x}\Z$. \\
	Dann ist $0\rightarrow \Z\xrightarrow{u}\Z$ exakte und $A\otimes_A M=M$.\\
	Aber 
	\begin{align*}
	0\rightarrow \Z\otimes\Z/2\Z&\xrightarrow{u\otimes id_{\Z/2\Z}}\Z\otimes_\Z\Z/2\Z\\
	\Z/2\Z&\xrightarrow{\cdot 2}\Z/2\Z
	\end{align*}
	ist nicht injektiv.
\end{exm}
\section{Lokalisierung}
\subsection*{A) Lokalisierung von Ringen und Moduln}
\begin{definition}
	Eine Teilmenge $S\subseteq A$ heißt \underline{multiplikativ}, falls $1\in S$ uns $s,t\in S\Rightarrow st\in A$.
\end{definition}
\begin{exm}
	\begin{enumerate}
		\item $S=\Z\setminus \{0\}\subseteq A=\Z$
		\item Sei $f\in A$, dann ist $S_f=\{1,f,f^2,...,\}$ eine multiplikative Teilmenge.
		\item Sei $y\subset A$ Primideal. Dann ist $A\setminus y\subset A$ eine multiplikative Teilmenge.
	\end{enumerate}
\end{exm}
\begin{definition}
	Sei $A$ ein Ring, $S\subseteq A$ eine multiplikative Teilmenge.\\
	Definiere auf $A\times S$ eine Äquivalenzrelation durch
	\[(a,s)\sim (b,t):\Leftrightarrow at=bs\]
\end{definition}
\begin{proof}
	Dies ist eine Äquivalenzrelation:\begin{itemize}
		\item Refelxivität
		\item Symmetrie
		\item Transitiv: $(a,s)\sim (b,t)$, $(b,t)\sim (c,u)$
		\[\exists v,w\in S:\quad vat=bvs\quad,\quad wba=wtc\]
		Dann ist $vbsw=^!$ %TODO bis "genannt kanonisch"
	\end{itemize}
\end{proof}
\begin{satz}[Universelle Eigenschaft]
	Sei $S\subseteq A$ eine multiplikative Teilmenge und sei $1:A\rightarrow S^{-1}$ kanonisch. Sei $B$ ein Ring, $\varphi:A\rightarrow B$ ein Ring-Homomorphimsmus mit $\varphi(s)\in B^\times=\{b\in B\mid  \exists c\in B: bc=1\}$ für alle $s\in S$. Dann existiert ein eindeutiger RIngHomomorphismus $\tilde{\varphi}S^{-1 A\rightarrow B}$ mit $\tilde{\varphi}\circ 1=\varphi:$.
	
	\[\xymatrix{
		 A\ar[r]^{\forall\varphi:\varphi(s)\subseteq B^\times}\ar[d]^{1}&B\\
		S^{-1}A\ar[ru]^{\exists!\tilde\varphi}&
	}\]
\end{satz}
\begin{proof}
	\underline{Eindeutigkeit} Für $\frac{a}{s}-in S^{-1}A$ muss für $\tilde{\varphi}$ gilt:
	\begin{align*}
	\tilde\varphi\left(\frac{a}{a}\right)&=\tilde{\varphi}\left(\frac{a}{1}\left(\frac{s}{1}\right)^{-1}\right)=\tilde{\varphi}\left(\frac{a}{1}\right)\tilde{\varphi}\left(\frac{s}{1}\right)^{-1}&(*)\\
	&=\varphi(a)\varphi(s)^{-1}
	\end{align*}
	\underline{Eindeutigkeit}
	Definiere $\tilde\varphi$ durch $(*)$\\
	Z.z: $\tilde\varphi$ ist wohldefiniert.
\end{proof}
\begin{bem}
	Sei $S\subseteq A$ eine multilineare Teilmenge.\\
	Dann gilt: $1:A\rightarrow S^{-1}A$ ist injektive $\Leftrightarrow$ S enthält keien Nullteiler.
\end{bem}
\begin{proof}
	\begin{align*}
	&1\text{ ist injektiv }\\
	\Leftrightarrow&\Ker(1)=0\\
	\Leftrightarrow&(\forall a\in A:\frac{a}{1}=1\Rightarrow a=0)
	\Leftrightarrow&(\forall a\in A:\exists s\in S:as=0\Rightarrow a=0)
	\Leftrightarrow&\text{$S$ enthält eine Nullteiler}
	\end{align*}
\end{proof}
\begin{satz}[Lokalisierung von Moduln]
	Sei $S\subseteq A$ ein multiplikative Teilmenge, $M$ ein A-Modul. Definiere auf $M\times S$ eine Äquivalenz Relation:
	\[(m,s)\sim (n,t)\Leftrightarrow\exists v\in S:vtm=vsm\]
	Man erhält den $S^{-1}A$-Modul $S^{-1}M=(M\times S)/\sim$:
	\begin{itemize}
		\item Mit Addition: $\frac{m}{s}+\frac{n}{t}:=\frac{tm+sn}{st}$
		\item Mit Skalarmultiplikation: $\frac{a}{s}\cdot \frac{m}{t}:=\frac{am}{st}$
	\end{itemize}
\end{satz}
\begin{satz}[Lokalisierung als Funktor]
	Sei $u:M\rightarrow  N$ eine A-lineare Abbildung, $S\subseteq A$ ein multiplikative Teilgruppe. Dann ist
	\begin{align*}
	S^{-1}u:S^{-1}M&\rightarrow S^{-1}N\\
	\frac{m}{s}&\mapsto \frac{u(m)}{s}
	\end{align*}
	eine $S^{-1}A$ lineare Abbildung.
\end{satz}
\begin{satz}[Lokalisierung ist exakt]
	InhaltSei $M'\xrightarrow{u}M\xrightarrow{v}M''$ eine exakte Sequenz von A-Moduln, $S\subseteq$ eine multilineare Teilmenge. Dann ist
	\[S^{-1}M'\xrightarrow{S^{-1}u}S^{-1}M\xrightarrow{S^{-1}v}S^{-1}M''\]
	eine exakte Sequnez von $S^{-1}A$ Moduln.
\end{satz}
\begin{proof}
	$v\circ u=0$. Also ist $S^{-1}v\circ S^{-1}u=0$.\\
	Noch zu zeigen: $\Ker(S^{-1}v)\subseteq \Img(S^{-1}u)$.\\
	Sei $\frac{m}{s}\in S^{-1}M$ mit $S^{-1}v\frac{v}{s}=\frac{v(m)}{s}=0$.\\
	Also gibt es $t\in S:tv(m)=v(tm)=0$.\\
	Damit liegt $tm\in\Ker(v)=\Im(u)$.\\
	Also existiert $m\in M:u(m'=tm)$.
	Dann ist $S^{-1}u\left(\frac{m'}{st}\right)=\frac{u(m')}{st}=\frac{m}{s}$ und damit $\frac{m}{s}\in\Img(S^{-1}u)$
\end{proof}
\begin{prop}
	Sei $M$ ein A-Modul, $S\subseteq A$ eine multiplikative Teilmenge, dann ist
	\begin{align*}
	u:S^{-1}A\otimes_A M&\inj S^{-1 M}\\
	\frac{a}{s}\otimes m\mapsto \frac{am}{s}
	\end{align*}
	ist Homomorphismus von $S^{-1}A$-Moduln.
\end{prop}
\begin{proof}
	\begin{enumerate}
		\item $1$ ist wohldefiniert: z.Z:
		\begin{enumerate}
			\item $\frac{a}{s}=\frac{b}{t}\Rightarrow \frac{am}{s}=\frac{bm}{t}$.
			\item $\frac{am}{s}$ ist linear in $\frac{a}{s}$ und in $m$.
		\end{enumerate}
		\item %TODO
	\end{enumerate}
\end{proof}
%VL 09.11.2016
\begin{satz}[Ideal in $S^{-1}A$]
	Sei $S\subseteq A$ eine multilineare Teilmenge.
	\begin{align*}
	\left\{\text{Ideale in A}\right\}
	{\xrightarrow{\mathfrak a\mapsto S^{-1\mathfrak a}}\atop \xleftarrow[b\mapsto \iota^{-1}(b)]{}
	}\left\{\text{Ideale in $S^{-1}A$}\right\}
	\end{align*}
	\[1:A\rightarrow S^{-1}A,a\mapsto \frac{a}{1}\]
	Nicht zu einander invers.
	\begin{enumerate}
		\item Sei $\mathfrak a\subseteq A$ ein Ideal. Dann ist $S^{-1\mathfrak a}=S^{-1}A$ genau dann wenn $\mathfrak a\cap S\neq 0$.\\
		Dann folgt auch, dass $\mathfrak\mapsto S^{-1\mathfrak a}$ ist nur invertierbar , falls $S\subseteq A^\times$.
		\item Für $b\subseteq S^{-1}A$ Ideal gilt:\\
		\[S^{-1}(\iota^{-1}(b))=b\]
		Dann folgt $b\mapsto\iota^{-1}(b)$ ist injektiv und jedes Ideal von $S^{-1}A$ ist von der Form $S^{-1}\mathfrak a$ für einIdeal $\mathfrak a\subseteq A$.
		\item Sei $\mathfrak a\subseteq A$ ein Ideal. Dann gilt:
		Es gibt ein Ideal $b\subseteq S^{-1}A$ mit $\mathfrak a=\iota^{-1(b)}$.\\
		Dies ist Äquivalent dazu, dass kein $s\in S$ ins $A/\mathfrak a$ Nullteiler ist.
		\item Man hat zueinander inverse Bijektionen: \label{4104}
		\[\left\{q\subset S^{-1 A}\mid \text{Primideal}\right\}
		\xrightarrow{q\mapsto\iota^{-1(q)}}\atop\xleftarrow[\mathfrak p\mapsto S^{-1\mathfrak p}]{}\left\{\text{Primiedeale $\mathfrak p\subset A$ mit $\mathfrak p\cap S=\emptyset$}\right\}
		\]
		%TODO 4
	\end{enumerate}
\end{satz}
\begin{proof}
	\begin{enumerate}
		\item $\frac{1}{1}-in S^{/1 A}$ ist genau dann wenn es ein $a\in\mathfrak a,s\in S$ gibt, sodass $\frac{a}{s}=\frac{1}{1}$.
		\begin{align*}
		&\Leftrightarrow \exists a\in\mathfrak a,s,t\in S: ta=ts\\
		&\Leftrightarrow\mathfrak a\cap S\neq 0
		\end{align*}
		\item Sei $\frac{a}{s}\in S^{-1}(\iota^{-1(b)})$.\\
		Ist äquivalent zu $\exists t\in S$ und $b\in A$ mit $\frac{b}{1}\in b$, so dass
		\[\frac{a}{s}=\frac{b}{t}=\frac{b}{1}\frac{1}{t}\]
		$\Leftrightarrow \frac{a}{s}\in b$
		\item Sei $\mathfrak a=\iota^{-1}(b)$ für ein Ideal $b\subseteq S^{-1}A$.
		\begin{align*}
		&\Leftrightarrow\mathfrak a=\iota^{-1}(S^{-1}\mathfrak a)\\
		&\Leftrightarrow A/\mathfrak 
		a\xrightarrow{\ol{a} \mapsto \ol{\left(\frac{a}{1}\right)}}
		S^{-1}A/S^{-1}\mathfrak a=^{\ref{480}}S^{-1}A/\mathfrak a\quad\text{injektiv}\\
		\end{align*}
		(Wende $\ref{480}$ an auf die exakte Sequenz
		\begin{align*}
		0\rightarrow \mathfrak a&\rightarrow A\rightarrow A/\mathfrak a\rightarrow 0
		\intertext{Dann ist auch}
		0\rightarrow S^{-1}\mathfrak a&\rightarrow S^{-1}A\rightarrow S^{-1}(A/\mathfrak a)\rightarrow 0\\
		\end{align*} exakt.)
		Mit $\ref{450}$ gilt äquivalenz dazu, dass kein $s\in S$ ist Nullteiler in $A/\mathfrak a$.
		\item %TODO Proof 4
	\end{enumerate}
\end{proof}
%TODO 4.10, 4.11
\begin{satz}[Universelle Eigenschaft des Quotientenkörpers]
	Sei $\iota:A\rightarrow \Qud(A)$ kanonisch und sei $\varphi:A\rightarrow K$ ein injektiver Ring-Homomorphismus wobei $K$ ein Körper.\\
	Dann existiert genau ein Homomorphismus von Körpern $\tilde{\varphi}:\Qud(A)\rightarrow K$.
\end{satz}
%TODO. Beweis?
\subsection*{(B) Lokale Ringe und Restklassenkörper}
\begin{definition}
	Ein Ring $A$ heißt \underline{lokal} wenn er genau ein Maximales Ideal besitzt.\\
	Dann bezeichnet $\mathfrak m_A$ dieses Maximales Ideal.\\
	Der Körper $\kappa(A):=A/\mathfrak m_A$ heißt \underline{Restklassenkörper von $A$}.
\end{definition}
\begin{exm}
	\begin{itemize}
		\item Jeder Körper ist ein lokaler Ring.
		\item Ein Hauptidealring $A$ ist genau dann lokal, wenn bis auf Multiplikation mit Einheiten genau ein irreduzibles Element existiert.\\
		Oder wenn $A$ Körper ist
	\end{itemize}
\end{exm}
\begin{definition}
	Ein lokaler Hauptideal Ring der kein Körper ist, heißt \underline{diskreter Bewertungsring}.
\end{definition}
\begin{exm}
	Sei $\mathfrak p\subset A$ Primideal, $S:=A\backslash \mathfrak p$ multiplikative Teilmenge, $A_{\mathfrak p}:=S^{-1}A$.
	\[\{\text{Primideals in $A-\mathfrak p$}\}\leftrightarrow\{\text{Primideals $q\subset A$ mit $q\subseteq \mathfrak p$}\}\]
	(mit $\ref{4104}$).\\ 
	Also ist $A_{\mathfrak p}$ ein lokaler Ring mit maximalem Ideal $S^{-1}\mathfrak p$.\\
	Der Körper $\kappa(\mathfrak p):=A/S^{-1}\mathfrak p$ heißt \underline{Restklassenkörper in $\mathfrak p$}.
\end{exm}
\begin{bem}
	Seien $q\subseteq\mathfrak p\subset A$ Primideale.\begin{enumerate}
		\item \begin{align*}
		\{\text{Primideale in $A_{\mathfrak p}$}\}&=\{\text{Primideale in $A$, die in $\mathfrak p$ enthalten sind}\}\\
		\{\text{Primideal in $A/q$}\}&=\{\text{Primideal in $A$, die $q$ enthalten.}\}
		\end{align*}
		\item  Sei $S:=S\backsim\mathfrak p$. Dann ist $S^{-1}(A/q)=S^{-1} A/S^{-1}q$ und
		\[\{\text{Primideal in $S^{-1}(A/q)$}\}=\{\text{Primideals in $A$ die zwischen $q$ und $\mathfrak p$ liegen}\}\]
		\item Speziell für $q=\mathfrak p$:
		\begin{align*}
		S^{-1}(A/\mathfrak p)&=\kappa(\mathfrak p)\\
		&=\Qud(A/\mathfrak p)
		\end{align*}
	\end{enumerate}
\end{bem}
\subsection*{(C)Spektren}
\begin{rem}
	Ein \underline{Topologischer Raum} ist ein Paar $(X;\mathfrak T)$ wobei $X$ eine Menge, $\mathfrak T\subseteq \mathscr P(X)$, sodass gilt:
	\begin{enumerate}
		\item $\emptyset \in\mathfrak T,X\in\mathfrak T$
		\item Sei $(U_i)_{i\in I}$ eine Familie von Mengen $U_i\in\mathfrak T$ dann gilt $\forall i\in I:\bigcup_{i\in I}U_i\in\mathfrak T$
		\item $U,V\in\mathfrak T$, dann $U\cap V\in\mathfrak T$
	\end{enumerate}
Die Mengen in $\mathfrak T$ heißen \underline{offen}.
\end{rem}
\begin{rem}
	Seine $X,Y$ topologische Räume. Eine Abbildung $f:X\rightarrow Y$ heißt \ul{stetig}, falls $f^{-1}(V)\subseteq X$ ist offen für alle offenen $V\subseteq Y$.
\end{rem}
\begin{rem}
	Sei $(X,\mathfrak T)$ ein topologischer Raum $B\subseteq \mathfrak T$ heißt \ul{Basis der Topologie}, falls jeder offenen Teilmenge Vereinigung von Menge aus $B$ ist.
\end{rem}
\begin{exm}
	Sei $(X,d)$ eien metrischer Raum, dann heißt $U\subseteq X$ offen, falls \[\forall x\in U\exists\epsilon>0:B_\epsilon(x)\{y\in X\mid M(x,y)<\epsilon\}\subseteq U\]Basis der Topologie: $\{B_\epsilon(x)\mid\epsilon\in \R^{>0},x\in X\}$
\end{exm}
\begin{definition}
	Sein topologischer Raum $X$ heißt \underline{Hausdorffsch}, falls $\forall x,y\in X$ mit $x\neq y$ existieren $x\in U\subseteq X$, $y\in V\subseteq X$ offen, sodass $U\cap V=\emptyset$.
	Metrische Räume sind Hausdorffsch.
\end{definition}
\begin{definition}
	Ein topologischer Raum $X$ heißt \ul{quasikompakt}, falls jede offene Überdeckung $(U_i)_{i\in I}$ von $X$ (d.h. $U_i\subseteq X$ offen für alle $i\in I$ mit $\bigcup_{i\in I}U_i=X$) eine endliche Teilüberdeckung besitzt.
	(d.h. $\exists J\subseteq I$ endliche Teilmenge, sodass $\bigcup_{i\in I}U_i=X$.)
\end{definition}
%VL 14.11.2016
\subsection{Spielzeugmodell (der Funktionalanalysis)}
Sei $X$ ein kompakter topologischer Raum,
\[A:=A_X:=\xi (X,\C):=\{f:X\to \C\text{ stetig}\}\]
Sei $x\in X$, dann betrachte
\[\mathfrak M_x:=\{f\in A\mid f(x)=0\}\subseteq A\]
Dies ist ein Minimales Ideal, denn
\[A/\mathfrak M_x\inj \C, \ol{f}\mapsto f(x)\]
\begin{satz}
	Die Abbildung 
\begin{align*}
	X&\rightarrow \operatorname{Max}(A):=\{\mathfrak M\subset A\mid \text{maximales Ideal}\}\\
	x&\mapsto\mathfrak M_x
\end{align*}
	ist bijektiv.
\end{satz}
\begin{kor}
	Sei $f\in A$ und für $\mathfrak M_x\in\operatorname{Max}(A)$ sie $f(x)=$ Bild von $f$ in $A/\mathfrak M_x=\C$.
	\begin{align*}
	D(f)&=\{\mathfrak M\in \operatorname{Max}(A)\mid \text{$\ol{f}$ in $A/\mathfrak M$ ist $\neq 0$}\}\\
	&=\{\mathfrak M\in\operatorname{Max}(A)\mid f\notin\mathfrak M\}\\
	&=\sigma(\{x\in X\mid f(x)\neq 0\})
	\end{align*}
\end{kor}
\begin{definition}
	$U\subseteq\operatorname{Max}(A)$ heißt \textbf{offen}, falls $\exists F\subseteq\operatorname{Max}(A)$ mit
	\[U=\bigcup_{f\in F}D(f)\]
	Dies ist die Topologie uf $\operatorname{Max}(A)$.\\
	(Bemerke: $D(f)\cap D(g)=D(fg)$)
\end{definition}
\begin{satz}
	$\sigma$ ist Homomorphismus
\end{satz}
Seien $X,Y$ kompakte topologische Räume, $F:X\rightarrow Y$ stetig.\\
Mann erhält den $\C-$Algebra-Homomorphismus:
\begin{align*}
\varphi:A_Y&\rightarrow A_x\\
f&\mapsto f\circ F
\end{align*}
Habe Kommutierendes Diagramm
\[\xymatrix{
	X\ar[r]^F\ar[d]_\sigma^\sim& Y\ar[d]_\sigma^\sim\\
	\operatorname{Max}(A_x)\ar[r]_{\vspace{0.5cm} \mathfrak M\mapsto\varphi^{-1}(\mathfrak M)}&\operatorname{Max}(A_Y)
	}\]
Es folgt $\forall \mathfrak M\subset A_x$ maximal, sodass $\varphi^{-1}(\mathfrak M)\subset A$ maximal ist.\\
%TODO Kommentar
\begin{def}
	\label{satz414}
	Sei $A$ ein Ring. Setze $X=\Spec(A):=\{y\subset A\mid\text{Primideal con $A$}\}$ als das \textbf{Spektrum von $A$}.\\
	Für $x\in X$ bezeichne $y_x\subset A$ das korrespondierene Primideal.
	Sei $f\in A$, $x\in X$. Dann definiere
	\[f(x):=\text{Bild von $f$ unter}A\rightarrow A/y_x\hookrightarrow\Qud(A/y_x)=\kappa(x)\]
\end{def}
\begin{bem}
	$f$ ist keine Funktion $X\rightarrow ?$.\\
	Seetze
	\begin{align*}
	D(f):&=\{x\in X\mid f(x)\neq 0\}\\
	&=\{x\in X\mid f\notin y_x\}
	\end{align*}
\end{bem}
\begin{definition}
	Eine Teilmenge $U\subseteq X=\Spec(A)$ heißt \textbf{offen}, falls $F\subseteq A$ Teilmenge existiert, sodass $U=\bigcup_{f\in F}D(f)$.\\
	Wir erhalten die sogenannte \textbf{Zanski-Topologie}.
	Dabie
	\begin{align*}
	D(f)\cap D(g)&=D(fg)\\
	\emptyset&=D(0)\\
	x&=D(x)
	\end{align*} 
\end{definition}
\begin{kor}[$D(f)$ als Spektrum]
	Sei $f\in A$ und sei $S_f:=\{1,f,f^2,...,\}$. Dann ist\begin{align*}
	\Spec(S^{-1}_fA)&=\{y\in\Spec(A)\mid y\cap S_f=\emptyset\}\\
	\{y\in\Spec(A)\mid f\notin y\}\\
	&=D(f)
	\end{align*}
\end{kor}
\begin{satz}[Abgeschlossenen Teilmengen]
	Sei $X=\Spec(A)$, $Y\subseteq X$ Teilmenge. Dann
	\[\text{$Y\subseteq X$ abgeschlossen}
	\Leftrightarrow \text{$X\setminus Y\subseteq X$ offen}
	\Leftrightarrow \exists F\subseteq A:X\setminus Y=\bigcup_{f\in f}D(f)\]
	Genau dann wenn
	\begin{align*}
	&\exists F\subseteq A&Y&=\bigcap_{f\in F}(X\setminus D(f))\\
	&&&=\bigcap_{f\in F}\{y\in A\mid f\in y\}\\
	&&&=\{y\in A\text{ Primideal}\mid(F)\subseteq y\}\\
	\Leftrightarrow&\exists\mathfrak a\subseteq A\text{ Ideal}&Y&=\{y\in A\text{ Primiedeal}\mid\mathfrak a\subseteq y\}\\
	&&&=\Spec(A/\mathfrak a)
	\end{align*}
\end{satz}
\begin{satz}[Funktorialität]
	\label{satz420}
	Sei $\varphi
	A\rightarrow  B$ ein Homomorphismus on Ringen.\\
	Dann ist $\varphi\Spec B\to\Spec(A),q\mapsto \varphi^{-1}(q)$ stetig.
\end{satz}
\begin{proof}
	Für $f\in A$ gilt
	\begin{align*}
	\varphi^{-1}(D(f))&=\{y\in\Spec(B)\mid\varphi(y)\in D(f)\}\\
	&=\{q\subset B\text{ Primideal}\mid\varphi^{-1}(q)\in D(f)\}\\
	&=\{q\subset B\text{ Primideal}\mid f\in \varphi^{-1}(q)\}\\
	&=\{q\subset B\text{ Primideal}\mid\varphi(f)\notin q\}\\
	&=D(\varphi(f))\subseteq\Spec(B)\text{ offen.}
	\end{align*}
\end{proof}
\subsection{(D) Lemma von Nokagama???}
\begin{definition}
	Sei $u:M\rightarrow N$ ein Homomorphismus von $A$-Moduln und sei $(m_1,...,m_r)$ ein Erzeugendensystem von $M$ und $(n_1,...,n_s)$ von N.\\
	Dann exitsiert
	\[T=(t_{ij})_{\substack{1\le i\le s\\ 1\le j\le r}}\in M_{s\times r}(A)\]
	sodass
	\[n(m_j)=\sum_{i=1}^{s}t_{ij}n_i\]
	Dann heißt $T$ eine \textbf{Matrix von $U$ bezüglich $(m_1,...,m_r)$ und $(n_1,...,n_s)$}.
\end{definition}
\begin{bem}
	\begin{enumerate}
		\item T ist nicht eindeutig duch $u$ bestimmt\\
		(es sei denn $(n_1,...,n_s)$ ist Basis)
		\item Nicht jede Matrix in $M_{s\times r}(A)$ ist eine Matrix von $u$ bezüglich $(m_1,...,m_r)$ und $(n_1,...,n_s)$.\\
		(Es sei denn $m_1,...,m_r$ ist Basis von $M$)
	\end{enumerate}
\end{bem}
\begin{rem}
	Sei $T\in M_n(A)=A^{n\times m}$, $n\in \N$.\\
	Dann existiert $S\in M_n(A)$, sodass $TS=ST=\det TI_m$. Dann ist $S=(s_{ij})$
	\[s_{ij}=(-1)^{i+j}\det(T_{ji})\]
	($T$ mit $j$-ter Spalte und $i-$ter Spalte gestrichen.)\\
	$S$ heißt die Adjunkte von $T$.
\end{rem}
\begin{satz}[Cayley-Hamilton]
	\label{423CayHam}
	Sei $M$ ein $A$-Modul, $(m_1,...,m_n)$ ein Erzeugendensystem und sei $u:m\rightarrow M$ eine $A$-Lineare Abbildung. Sei $T\in M_r(A)$ eine Matrix von $u$ bezüglich $(m_1,...,m_r)$.\\
	Setze
	\[\chi_T:=\det\underbrace{(X I_r-A)}_{\in M_r(A[x])}=X^r+a_1X^{r-1}+...+a_{r-1}X+a_r\]
	Dann gilt
	\[\chi_T(u)=u^r+a_1u^{r-1}+...+a_{r-1}+a_r\operatorname{Id}_M=0\in\operatorname{End}_A(M)\]
	\begin{enumerate}
		\item Seo $\mathfrak a\subseteq A$ Idela, sod ass $u(M)\subseteq\mathfrak aM$. Dann $a_i\in\mathfrak a^i\forall i=1,...,r$.
	\end{enumerate}
\end{satz}
\begin{proof}
	$u(M)\subseteq\mathfrak aM$. Es folgt, dass die Koeffizienten von $T$ in $\mathfrak a$ liegen.\\
	$a_i$ ist Summe von $i$-fachen Produkten von Koeffizienten von $T$.\\
	Also $a\in\mathfrak a^i\forall i=1,...,r$.\\
	Sei nun $T^T=(t_{ji})_{i\le i,j\le r}$ aber $u(m_j)=\sum_{i}t_{ji}m_i$.\\
	Dann gilt
	\[\sum_i(u\delta_{ji})-t_{ji}m_i=0\]
	Sei nun
	\[C:=(X\delta_{ji}-t_{ji})_{ji}\in M_r(A[X])\]
	wobei $\chi_T=\det(C)$.\\
	Sei
	\[D:=(d_{jk})_{jk}\]
	Die Adjungte von $C$, also
	\[CD=\chi_TI_r\in M_r(A[X])\]
\end{proof}
\end{document}