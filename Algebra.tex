\documentclass[10pt,a4paper]{article}
\usepackage[utf8]{inputenc}
\usepackage[german]{babel}
\usepackage{amsmath}
\usepackage{amsfonts}
\usepackage{amssymb}
\usepackage{amsthm}
\usepackage{graphicx}
\usepackage{tikz,pgf}
\usepackage{mathrsfs}
\usepackage{framed}
\usepackage{hyperref}
\usepackage[all]{xy}

\newcommand{\gdw}{\ensuremath{\Leftrightarrow}}
\newcommand{\N}{\ensuremath{\mathbb{N}}}
\newcommand{\Z}{\ensuremath{\mathbb{Z}}}
\newcommand{\Q}{\ensuremath{\mathbb{Q}}}
\newcommand{\R}{\ensuremath{\mathbb{R}}}
\newcommand{\C}{\ensuremath{\mathbb{C}}}
\newcommand{\F}{\ensuremath{\mathbb{F}}}
\newcommand{\impl}{\ensuremath{\Rightarrow}}
\newcommand{\la}{\ensuremath{\lambda}}
\newcommand{\al}{\ensuremath{\alpha}}
\newcommand{\ol}[1]{\overline{#1}}
\newcommand{\ul}[1]{\underline{#1}}
\newcommand{\norm}[1]{\|#1\|}
\newcommand{\dif}[2]{\frac{\partial #1}{\partial #2}}
\newcommand{\mdif}[3]{\frac{\partial^#1 #2}{\partial #2^#3}}
\renewcommand{\P}{\ensuremath{\mathbb P}}
\newcommand{\E}{\ensuremath{\mathbb E}}
\newcommand{\Hlb}{\ensuremath{\mathscr H}}
\newcommand{\surjfunc}{\ensuremath{\xrightarrow{\sim}}}
\newenvironment{sumup}{\begin{framed}\textbf{Zusammenfassung:}\\}{\end{framed}}

\theoremstyle{definition}
\newtheorem{definition}{Definition}
\newtheorem{satz}{Satz}
\newtheorem{prop}{Proposition}

\theoremstyle{plain}
\newtheorem{lem}{Lemma}
\newtheorem{kor}{Korollar}

\theoremstyle{remark}
\newtheorem{bem}{Bemerkung}
\newtheorem{exm}{Beispiel}


\author{Prof Wedhorn, Mitschrift von Daniel Kallendorf}
\title{Algebra SS16}

\begin{document}
	\maketitle
	\tableofcontents
	\setcounter{section}{2}
%	\setcounter{theorem}{6}
	\begin{bem}
		$A[X_1,...,X_n]$ ist ein freier $A$-Modul, wobei die Monome eine Basis bilden.
	\end{bem}
	\begin{satz}[Universaleigenschaft des Polynomrings]
		Sei $\phi:A\rightarrow B$ eine $A$-Algebra und seine $b_1,...,b_n\in B$ Elemente. Dann existiert genau ein $A$-Algebra-Homomorphismus $\psi:A[X_1,...,X_n]\rightarrow B$, so dass $\psi(x_i)=b_i$ für alle $i=1,..,n$, nämlich
		\[\psi\underbrace{\left(\sum_{i_1,...,i_n\ge 0}a_{i_1,...,i_n}X_1^{i_1}\cdot...\cdot X_n^{i_1}\right)}_{=:f}=\underbrace{\sum_{i_1,...,i_n\ge 0}\phi(a_{i_1,...,i_n})b_1^{i_1}\cdot...\cdot b_n^{i_n}}_{=f(b_1,...,b_n)}\]
	\end{satz}
	\begin{bem}
		\begin{align*}
		\operatorname{Im}(\psi)&=\text{kleinste $A$-Unteralgebra die $b_1,...,b_n$ enthält}\\
		&=A[b_1,...,b_n]\subset B
		\end{align*}
	\end{bem}
	\begin{exm}
		Sei $\phi:A\rightarrow B$ eien $A$-Algebra, $b\in B$. Es existiere ein $g\in A[X]$ mit $g(b)=0$. Sei $g$ nomriert. Dann gilt\\
		\[A[b]=\{f(b)|f\in A[x],\deg(f)<\deg (g)\}\]
	\end{exm}
	\begin{exm}
		Sei $A=\Q\hookrightarrow\C$,$i\in \C$. \\
		Dann gilt $g(i)=0$ wobei $g=X^3+X=X(X^2+1)$. Es folgt:
		\begin{align*}
		\Q[i]&=\{a_0+q_1i+a_2i^2|a_0,a_1,a_2\in\Q\}\\
		\Q[i]&=\operatorname{Im}(\Q[X]\xrightarrow[X\mapsto i, f\mapsto f(i)]{\psi}\C)
		\end{align*}
		Dann $\tilde{g}\in\Q[X]: \psi(\tilde{g})=0\Leftrightarrow \tilde{g}(i)=0$.\\
		Also $g\in\operatorname{Ker}(\psi)\Rightarrow (g)\subseteq\operatorname{Ker}(\psi)$.\\
	In diesem Fall $\operatorname{Ker}\psi=(X^2+1)$.\\
	\end{exm}
	Begründung von 2.8:
	\[(g)\subseteq\operatorname{Ker}\left(A[X]\xrightarrow[f\mapsto f(b)]{\psi}B\right)\]
	Also $\psi$ faktorisiert:
	\[A[X]/(g)\xrightarrow{\ol{\psi}}A[b]\subseteq B\]
	mit $\ol{\psi}$ surjektiv.
	\begin{prop}
		Sei $g\in A[X]$ normiert. Dann ist\[\{f\in A[X],\deg(f)<\deg(g)\}\hookrightarrow A[X]\rightarrow A[X]/(g)\]
		bijektiv.
	\end{prop}
	\begin{proof}
		Gilt, da für alle $f\in A[X]$ genau ein $r\in A[X]$ exitiert mit $\deg(r)<\deg(g)$ mit $f\in r+(g)$
		%TODO
	\end{proof}
\section{Tensorprodukte}
(A) Tensorprodukte von Moduln\\
(B) Tensorprodukte von Algebren und Basiswechsel\\
(C) Exaktheitseigenschaften des Tensorprodukts\\
\subsection{Erinnerung}
\begin{definition}
	$A$-Modul$:=(M,+,\cdot)$ wobei $(M,+)$ abelsche Gruppe und $\cdot:A\times X\rightarrow M$ ein Skalarprodukt.
\end{definition}
\begin{bem}
	$\Z$-Modul=ablesche Gruppe
\end{bem}
\begin{exm}
	Sei $I$ eine Menge
	\[A^{(I)}=\{(a_i)_{i\in I}|a_i\in A, a_i=0\text{für fast alle $i\in I$}\}\]
	$A$-Modul mit Addition und Skalarprodukt.\\
	Für $i\in I:e_i\in A^{(I)}$ mit
	\[e_i=\begin{cases}
	\text{1 an der i-ten Stelle}\\
	\text{0 sonst}
	\end{cases}\]
\end{exm}
\begin{definition}
	Ein $A$-Modul heißt frei, falls $M\approxeq A^{(I)}$ für eine Menge $I$
\end{definition}
\begin{definition}
	Sei $M,N$ $A$-Modul. Dann heißt $u:M\rightarrow N$ A-linear oder Homomorphismus von $A$-Moduln, falls
	\[u(am+m')=au(m)+u(m')\forall a\in A,m,m'\in M\]
\end{definition}
\begin{bem}
	Sei $I$ eine Menge, $M$ ein $A$-Modul $\underline{m}=(m_i)_{i\in I}$ ein Tupel von Elementen $m_i\in M$. Dann Existiert genau eine Abbildung:
	\[A^{(I)}\xrightarrow{u_{\underline m} }M\]
	mit $u_{\underline{m}}(e_i)=m_i$.\\
	$(m_i)_i=\underline{m}$ heißt linear Unabhängig/ Erzeugende-System/ Basis, \\
	falls $u_{\underline{m}}$ injektiv/ surjektiv / bijektiv ist.
\end{bem}
\begin{bem}
	Der $A$-Modul M ist endlich erzeugt, genau dann wenn ein $n\in \N$und eine $A$-lineare Surjektion $A^m\rightarrow M$ existieren.
\end{bem}
\subsection{Multilineare Abbildungen}
\begin{definition}
	Sei $r\in \N_0$, $M_1,...,M_r,P$ A-Moduln.\\
	Eine Abbildung $\al:M_1\times...\times M_r\rightarrow P$ heißt \underline{r-multilinear}, falls sie in jeder Komponente linear ist, d.h. Für alle $i=1,...,r$ gilt:
	\[\al(m_1,...,am_{i}+m_i',m_{i+1},...,m_r)=a\al(m_1,...,m_i,...,m_r)+\al(m_1,...,m_i',...,m_r)\]
	Für alle $m_j\in M_j,m_i\in M_i,a\in A$.
	($r=1$: linear, $r=2$: bilinear)
\end{definition}
%TODO Setze...
\subsection{..}
\begin{definition}
	Sei $r\ge 2$, $M_1,..,M_r$ A-Moduln.\\
	Dann existiert ein $A$-Modul $M_1\otimes_AM_2\otimes_A...\otimes_AM_r$ und eine $r$-multilineare Abbildung $\tau:M_1\times...\times M_r\rightarrow M_1\otimes_AM_2\otimes_A...\otimes_AM_r$, sodass für jede $r$-multilineaer Abbildung:
	\[\al M_1\times...\times M_r\rightarrow P\]
	wobei $P$ ein A-Modul, genau ein A-lineare Abbildung 
	\[\ol\al:M_1\otimes_A...\otimes_AM_r\rightarrow P\]
	existiert.
	$\xymatrix{
		M_1\times...\times M_r \ar[r]^{\forall\al: \text{r-multilinear}} & P\\
		M_1\otimes_AM_2\otimes_A...\otimes_AM_r&
	}$
\end{definition}
%VL 31.10.2016
\begin{satz}[Eindeutigkeit des Tensorprodukts]
	Seien $(T,\tau:M_1\times...\times M_r\rightarrow T)$ und $(T',\tau')$ Tensorprodukte:
	\[\xymatrix{ 
		M_1\times...\times M_r \ar[d]^{\tau} \ar[dr]^{\tau'}&\\
		T\ar[r]_{\exists!v}&T'
	}\]
	$u$ existiert aufgrund der universellen Eigenschaft von $(T,\tau)$.\\
	$v$ existiert aufgrund der universellen Eigenschaft von $(T',\tau')$.\\
	Ferner kommutiert\\
	
	Die Universelle Eigschaft von $(T,\tau)$ zeigt, dass $v\circ u=id_T$, genauso $u\circ v=id_T$.
\end{satz}
\begin{satz}[Existenz des Tensorprodukts]
	\begin{enumerate}
		\item Suche einen $A$-Modul $N$ und eine Abbildung $c:M_1\times...\times M_r\rightarrow R$, sodass
		\[\text{Hom}_A(N,P)\xrightarrow[u\mapsto u\circ\tau]{ }\text{Abb}(M_1\times...\times M_r,P)\]
		Für alle $A$-Moduln $P$.
		%TODO
		\item Wir wollen, dass $(am_1+m_1',m_2,...,m_r)$ und $a(m_1,...,m_r)+(m_1',...,m_r)$ auf das gleiche Element abgebildet werden.\\
		Sei $Q\subseteq N$ der von
		\begin{align*}
		e_{(m_1,...,m_{i-1},am_i+m_i',m_{i+1},...,m_r)}-\left(ae_{(m_1,...,m_i,..,m_r)}+e_{(m_1,...,m_i',...,m_r)}\right)
		\end{align*}
		für alle $i=1,...,r$ und $m_i,m_i'\in M_i$ und $a\in A$ erzeugt Untermodul.\\
		Dann setze $T:=N/Q$. Dann gilt
		\begin{align*}
		\text{Hom}_A(T,P)&=\{u\in \text{Hom}(N,P)|u(Q)=0\}\\
		&=L_A(M_1,...,M_r,P)
		\end{align*}
		mit $\tau:M_1\times...\times M_r\rightarrow N\rightarrow N/Q$.
	\end{enumerate}
\end{satz}
\begin{bem}
	3.4\\
	$e_{(m_1,...,m_r)}\in A^{(M_1\times...\times M_r)}$ bilden ein Erzeugndensystem.\\
	Also bilden auch die $\tau(m_1,...,m_r)=:m_1\otimes...\otimes m_r$ eine Erzeugenden-System des $A-$Moduls $M_1\otimes...\otimes M_r$.\\
	\textbf{Aber:} Nicht jedes Element von $M_1\otimes...\otimes M_r$ ist in dieser Form.\\
	\\
	Also genüt es eine lineare Abbildung $u:M_1\otimes...\otimes M_r\rightarrow P$ auf den erzeugdnesn $m_1\otimes...\otimes m_r$ mit ($m_i\in M_i$) anzugeben.\\
	Umgekehrt sei $P$ ein A-mOdul und es seien elemente $u(m_1\otimes...\otimes m_r)\in P$ gegeben für alle $m_i\in M_i$.\\
	Genau dann existiert eine $A$-lineare Abbildung $u:M_1\otimes...\otimes M_r\rightarrow P$ mit $m_1\otimes ...\otimes m_r\mapsto u(m_1\otimes ...\otimes m_r)$, wenn für alle $i=1,...,r$, $a\in A$, $m_j\in M_j$ und $m_i'\in M_i$ gilt:
	\[u(m_1\otimes ..\otimes a m_i+m_i'\otimes..\otimes m_r)=a u(m_1\otimes ..\otimes m_i\otimes..\otimes m_r)+u(m_1\otimes ..\otimes a m_i'\otimes..\otimes m_r)\]
\end{bem}
\begin{satz}[Tensorprodukt linearer Abbildungen]
	Seien $M,M',N,n'$ $A$-Moduln, $u:M\rightarrow M',v:N\rightarrow N'$ $A$-lineare Abbildungen.\\
	Dann definiert
	\begin{align*}
	M\otimes_A N&\rightarrow M'\otimes A N'\\
	m\otimes n &\mapsto u(m)\otimes u(n)
	\end{align*}
	eine $A$-lineare Abbildung bezüglich $u\otimes v:M\otimes N\rightarrow M'\otimes N$.
\end{satz}
\begin{proof}
	Zu zeigen: $u(am+m')\otimes v(n)=a(u(m)\otimes v(n))+u(m')\otimes v(n)$\\
	Es gilt da das Tensorprodukt $r$-linear ist.
	\begin{align*}
	u(am+m')\otimes v(n)&=(au(m)+u(n))\otimes v(n)\\
	&=(au(m)\otimes v(n))+u(m')\otimes v(n)
	\end{align*}
	\\
	Außerdem zu zeigen: $u(m)\otimes v(an+n')=a(u(m)\otimes v(n))+u(m)\otimes v(n)$\\
	($\rightarrow $ Genauso.)
\end{proof}
\begin{bem}
	3.6\\
	\begin{enumerate}
		\item $A\otimes_A M\approxeq M$\\
		$u: a\otimes m\mapsto am$\\
		$v: 1\otimes m.....m $
		Dabei ist $u$ wohldefiniert, d.h. $(a,m)\rightarrow am$ ist bilinear.
		%TODO
		\item $M\otimes_A N\xrightarrow{\sim}N\otimes_A M, m\otimes n\mapsto n\otimes m$ ist ... von A-Moduln.\\
		Zu zeigen: Wohldefineirtheit 
		%TODO
		\item $M\otimes_A N\otimes_A P\simeq (M\otimes_A N)\otimes_A P$\\
		$m\otimes n\otimes p\mapsto (m\otimes n)\otimes p$\\
		$m\otimes n\otimes p\mapsto m\otimes (n\otimes p)$
	\end{enumerate}
\end{bem}
\begin{prop}3.7
	Sei $(M_i)_{i\in I}$ eine Familie von $A$-Moduln, $N$ ein A-Modul:\begin{align*}
	\left(\bigotimes_{i\in I}M_i\right)\otimes_A N\xrightarrow{\sim} \bigotimes_{i\in I}\left(M_1\otimes_A N\right)\\
	(m_i)_{i\in I}\otimes n\mapsto(m_i\otimes n)_{i\in I}
	\end{align*}
\end{prop}
\begin{proof}
	Umkehrabbildung gegeben durch:\[Inhalt..m_i\otimes n\mapsto (m_j)_{j\in I}\otimes n\]
	mit $m_j:=\begin{cases}
	m_i, &j=i\\
	0 &j\neq i
	\end{cases}$
\end{proof}
\subsection{Basiswechsel von Tensorprodukten}
\begin{satz}
	\begin{enumerate}
		\item Sei $M$ ein A-Modul. Dann wird
		\[\varphi^*(M):=B\otimes_A M\]
		zu einerm $B$-Modul mit dem Skalarprodukt
		\begin{align*}
		B\times(B\otimes_A M)&\rightarrow B\otimes_A M\\
		(b,b'\otimes m)&\mapsto bb'\otimes m
		\end{align*}
		\item Sei $U:M\rightarrow M'$ ein Homomorphismus von A-Moduln. Dann ist
		\begin{align*}
		id_B\otimes u:B\otimes M&\rightarrow  B\otimes_A M'\\
		b\otimes m\mapsto b\otimes u(m)
		\end{align*}
		eine B-lineare Abbildung.S
	\end{enumerate}
\end{satz}
\begin{prop}
	Sei $\varphi:A\rightarrow B$ eine A-Algebra.\\
	Sei $M$ ein freier A-Modul. Dann ist $B\otimes_A M$ ein freier B-Modul und
	\[\vartheta_A(M)=\vartheta_B(B\otimes_A M)\]
\end{prop}
\begin{proof}
	Sei $M$ ein freier A-Modul. Dazu ist äquivalent, dass $M\simeq A^{(I)}$.\\
	Daraus folgt, dass
	\begin{align*}
	B\otimes_A M&\simeq B\otimes_A A^{(I)}\\
	&\simeq B\otimes_A\left(\bigoplus_{i\in I}A\right)\\
	&\simeq \left(\bigoplus_{i\in I}B\otimes_A A\right)\\
	&\simeq\bigoplus_{i\in I}B\\
	&=B^{(I)}
	\end{align*}
	Also ist $B\otimes_A M$ frei.
	%TODO
\end{proof}
%VL 02.11.2016
\begin{prop}
	Sei $\mathfrak{a}\subseteq A$ ein Ideal, $M$ ein A-Modul.Setze 
	\begin{align*}
	\mathfrak a\cdot M&=\langle \{am|a\in\mathfrak a,m\in M\}\\
	&=\left\{\sum_{i=1}^{m}a_im_i\mid n\in\N_0,a_i\in\mathfrak a,m_i\in M \right\}\\
	&\subseteq M \quad \text{Untermodul}
	\end{align*}
	Dann ist
	\begin{align*}
	A/\mathfrak a\otimes_A M&\xrightarrow{\sim} M/\mathfrak a M\\
	\ol a\otimes m&\mapsto \ol{am}
	\end{align*}
	ein Homomorphismus von $A/\mathfrak a$-Moduln.
\end{prop}
\begin{proof}$\ol a\oplus m\mapsto \ol{am}$ ist wohldefiniert:
	Zu zeigen:
	\begin{enumerate}
		\item Sei $a'\in A$ mit $\ol{a'}=\ol a\in A/\mathfrak a$.\\
		Dann ist $\ol{am}=\ol{a'm}\in M/\mathfrak aM$.
		Es gilt $\ol{a}'=\ol a$ gena dann wenn es ein $x\imath\mathfrak a$ gibt sodass $a'=a+x$.\\
		Daruas folgt, dass $a'm=am+xm$, und da $xm\in\mathfrak aM$ folgt $\ol{a'm}=\ol{am}$.
		\item $\ol{am}$ is linear in $a$, d.h.
		\[\ol{(ba+a')m}=b\ol{am}+a'\ol m \quad \text{für $a,a'\in A$, $b\in A$}\]
		\item $\ol{am}$ ist linear in $m$, d.h.
		\[\ol{a(bm+m')}=b\ol{am}+\ol{am'}\quad\text{für $m,m'\in M$, $b\in A$}\]
	\end{enumerate}
\end{proof}
\begin{prop}
	Eine Umkehrabbildung ist gegeben durch
	\begin{align*}
	v:M&\rightarrow A/\mathfrak a\otimes_A M\\
	m&\mapsto 1\otimes m
	\end{align*}
\end{prop}
\begin{proof}
	Zu zeigen: $\mathfrak aM\subseteq Ker(v)$, also für alle $x\in\mathfrak a,m\in M$ gilt $v(xm)=0$.
	\[v(xm)=1\otimes xm=\ol{x}\otimes m=0\]
	da $\ol{x}=\ol{0}\in A/\mathfrak a$.\\
	Noch zu zeigen:: $v$ ist Umkehrabbildung zu $\ol a\otimes m\mapsto \ol{am}$.
\end{proof}
\subsection{Tensorprodukte von Algebren}
\begin{definition}
	Sei $A\rightarrow  B_1$, $A\rightarrow  B_2$ A-Algebren.\\
	Dann definieren wir auf dem A-Modul $B_1\otimes_A B_2$ eine Multiplikation:
	\begin{align*}
	(B_1\otimes B_2)\times(B_1\otimes B_2)&\rightarrow B_1\otimes B_1\otimes B_2\\
	(a_1\otimes b_2,b_1'\otimes b_2')&\mapsto b_1b_1'\otimes b_2b_2'
	\end{align*}
	und erhalten die $A$-Algebra $B_1\otimes_A B_2$.
\end{definition}
\begin{exm}
	Sei $A\xrightarrow{\varphi} B$ eine A-Algebra und sei $C=A[X_1,...,X_n]/(f_1,...,f_r)$ und $f_i\in A[X-1,...,X_n]$.Dann ist
	\[B\otimes_A A[X-1,...,X_n]/(f_1,...,f_r)=B[X_1,...,X_n]/(\tilde{f}_1,...,\tilde{d}_r)\]
	wobei 
	\[f_i=\sum_{\ul{j}\in \N_0^n}a_{\ul{j}}X^{\ul{j}}\rightarrow \tilde{f}_i=\sum_j\varphi(a_j)\]
	\begin{enumerate}
		\item Sei $A=\Q$, $C=\Q[i]=\{a+b_i|a,b\in\Q\}=\Q[X]/(X^2+1)$
		\item $\R\otimes_Q Q[i]=\R[X]/(X^2+1)=\C$
		\item $C\otimes_Q Q[i]=C[X]/(X^2+1)=\C[X]/(X+i)\times\C[X]/(X-i)\simeq\C\times\C$
	\end{enumerate}
\end{exm}
\begin{exm}
	$A[X]\otimes_A A[Y]=(A[X])[Y]=A[X,Y]$ mit $f\otimes g\mapsto fg$
\end{exm}
\end{document}