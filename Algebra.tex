\documentclass[10pt,a4paper]{article}

\usepackage{luatex85}
\def\pgfsysdriver{pgfsys-pdftex.def}

%\usepackage[utf8]{inputenc}
%\usepackage{fontenc}
\usepackage[utf8]{luainputenc}

\usepackage[german]{babel}
\usepackage{amsmath}
\usepackage{amsfonts}
\usepackage{amssymb}
\usepackage{amsthm}
\usepackage{graphicx}
\usepackage{tikz,pgf}
\usetikzlibrary{cd}
\usetikzlibrary{babel}
\usepackage{mathrsfs}
\usepackage{framed}
\usepackage[hidelinks]{hyperref}
\usepackage{ulem}


\newcommand{\N}{\ensuremath{\mathbb{N}}}
\newcommand{\Z}{\ensuremath{\mathbb{Z}}}
\newcommand{\Q}{\ensuremath{\mathbb{Q}}}
\newcommand{\R}{\ensuremath{\mathbb{R}}}
\newcommand{\C}{\ensuremath{\mathbb{C}}}
\newcommand{\F}{\ensuremath{\mathbb{F}}}
\newcommand{\la}{\ensuremath{\lambda}}
\newcommand{\al}{\ensuremath{\alpha}}
\newcommand{\ol}[1]{\overline{#1}}
\newcommand{\ul}[1]{\underline{#1}}
\newcommand{\todomark}[1]{\fbox{\Large Hier könnte \sout{Ihre Werbung} #1 stehen}}
\newcommand{\norm}[1]{\left|#1\right|}
\newcommand{\mapsfrom}{\ensuremath{\mathrel{\reflectbox{\mapsto}}}}
\newcommand{\isomfunc}{\ensuremath{\xrightarrow{\sim}}}
\newcommand{\isomorph}{\ensuremath{\tilde{=}}}
\newcommand{\Hom}{\operatorname{Hom}}
\newcommand{\End}{\operatorname{End}}
\newcommand{\Aut}{\operatorname{Aut}}
\newcommand{\Ker}{\ensuremath{\operatorname{Ker}}}
\newcommand{\Img}{\ensuremath{\operatorname{Im}}}
\newcommand{\Quot}{\operatorname{Quot}}
\newcommand{\Spec}{\ensuremath{\operatorname{Spec}}}
\newcommand{\id}{\operatorname{id}}
\newcommand{\CoKer}{\operatorname{CoKer}}
\newcommand{\Nil}{\ensuremath{\operatorname{Nil}}}
\newcommand{\rad}{\ensuremath{\operatorname{rad}}}
\newcommand{\Jac}{\ensuremath{\operatorname{Jac}}}
\newcommand{\ord}{\ensuremath\operatorname{Ord}}
\newcommand{\cha}{\ensuremath{\operatorname{char}}}
\newcommand{\Gal}{\ensuremath{\operatorname{Gal}}}


\newenvironment{sumup}{\begin{framed}\textbf{Zusammenfassung:}\\}{\end{framed}}


\newcounter{thm}[section]
\renewcommand{\thethm}{\arabic{section}.\arabic{thm}}
\renewcommand{\thesubsubsection}{\arabic{section}.\arabic{thm}}
\renewcommand{\thesubsection}{\arabic{section}\Alph{subsection}}

\let\oldsubsubsection\subsubsection
\renewcommand{\subsubsection}{\stepcounter{thm}\oldsubsubsection}

\theoremstyle{definition}
\newtheorem{definition}[thm]{Definition}
\newtheorem{prop}[thm]{Proposition}

\theoremstyle{plain}
\newtheorem{lem}[thm]{Lemma}
\newtheorem{kor}[thm]{Korollar}
\newtheorem{satz}[thm]{Satz}
\newtheorem{theorem}[thm]{Theorem}

\theoremstyle{remark}
\newtheorem{bem}[thm]{Bemerkung}
\newtheorem*{bem*}{Bemerkung}
\newtheorem{rem}[thm]{Erinnerung}
\newtheorem{exm}[thm]{Beispiel}
\newtheorem*{exm*}{Beispiel}


\author{Prof Wedhorn, Mitschrift von Daniel Kallendorf}
\title{Algebra SS16}

\begin{document}
\maketitle
\tableofcontents


%VL 17.10.2017/19.10.2016/24.10.2016
\section{Erinnerung: Ringe und Ideale}
\subsection{Ideale, Primideal, maximale Ideale und Ring-Homomorphismen}
\setcounter{thm}{-10}
	\begin{definition}
		Man nennt $(A,+,\cdot)$ einen \textbf{Ring}(in dieser VL=kommutativer Ring), wenn
		\begin{enumerate}
			\item $(A,+)$ abelsch
			\item Es gibt ein neutrales Element der Multiplikation $1\in A:1a=a\forall a\in A$
			\item Die Multiplikation ist $\cdot$ assoziativ und kommutativ
			\item Distributivität
		\end{enumerate}
	\end{definition}
	\begin{definition}
		Seien $A,B$ Ringe. Eine Abbildung $\varphi:A\rightarrow B$ heißt \textbf{Ringhomomorphismus}, falls
		\begin{enumerate}
			\item $\varphi(a+a')=\varphi(a)+\varphi(a')$ für alle $a,a'\in A$
			\item $\varphi(aa')=\varphi(a)\varphi(a')$ für alle $a,a'\in A$
			\item $\varphi(1)=1$
		\end{enumerate}
	\end{definition}
	\begin{definition}
		Ein $A$-Modul mit $A$-bilinearer, kommutativer und assoziativer Multiplikation und neutralem Element heißt \textbf{$A$-Algebra}
	\end{definition}
	\begin{kor}
		$B$ ist $A-$Algebra genau dann wenn $\varphi:A\rightarrow B$ ein Ringhomomporhismus ist.
	\end{kor}
\begin{definition}
	Man nennt $\mathfrak a\subseteq A$ \textbf{Ideal}, falls
	\begin{enumerate}
		\item $\mathfrak a\subseteq (A,+)$ Untergruppe
		\item $a\in A,b\in\mathfrak a\Rightarrow ab\in\mathfrak a$.
	\end{enumerate}
	Sei $S\subseteq A$, dann ist
	\[AS=SA=(S):=\left\{\sum_{i=1}^{n}a_iS_i\mid n\in\N_0,a_i\in A,s\in S\right\}\]
	das \textbf{Kleinste Ideal} von $A$ das $S$ enthält.
\end{definition}

\begin{kor}
	Sei $\mathfrak a\subseteq A$. Es gilt $1\in\mathfrak a$ genau dann wenn $\mathfrak A$.
\end{kor}

\begin{definition}
	Sei $A$ Ring. $A$ heißt \textbf{nullteilerfrei}, falls $A\neq\{0\}$ und für $a,b\in A$ mit $a,b\neq 0$ auch $ab\neq0$ gilt.
\end{definition}
\begin{exm}
	\begin{itemize}
		\item Körper sind Nullteilerfrei
		
		\item $\Z$ ist Nullteilerfrei
		\item $\Z$ ist HIR
	\end{itemize}
\end{exm}


\begin{definition}
	Sei $A$ Ring. $A$ heißt \textbf{Hauptidealring}(HIR), falls $A$ nullteilrefrei ist und jeds Ideal $\mathfrak a\subset A$ von einem Element erzeugt wird.\\
	(d.h. $\mathfrak a=As=\{as\mid a\in A\}$ für ein $s\in A$)
\end{definition}
\begin{exm}
	\item Körper sind Hauptidealringe (Ideale in einem Körper $K$ sind nur $(0)=\{0\}$ und $(1)=K$)
	\item  $\Z,K[X]$ sind HIR
	\item $Z[X]$ ist nicht HIR $(p,X)$ ist für $p\in\text{Prim}$ nicht von einem Ideal erzeugt.
\end{exm}
\begin{rem}
	Sei $\varphi:A\rightarrow B$ ein Homomorphismus von Ringen
	\begin{enumerate}
		\item $\varphi(A)\subset B$ ist Unterring.\\
		($0,1\in\varphi(A)$, $a,a'\in\varphi(A)\Rightarrow a+a',aa'\in \varphi(A)$)\\
		$\Ker(\varphi)=\{a\in A\mid\varphi(A)=0\}\subseteq A$ ist Ideal\\
		$A/\Ker(\varphi)\isomfunc\varphi(A),\ol{a}\mapsto \varphi(a)$ ist ein Ring Homomorphismus.
		\item Sei $\mathfrak b\in B$ Ideal, dann $\varphi^{-1}(\mathfrak b)=\{y\in A\mid\varphi(a)\in b\}\subseteq A$ Ideal und $\varphi$ induziert einene injektiven Ring-Homomorphismus:
		\[\ol{\varphi}:A/\varphi^{-1}(\mathfrak b)\leftrightarrow B/\mathfrak b, \quad \ol{a}\mapsto\varphi(a)\]
		(wende 1) an auf $A\rightarrow B\rightarrow B/\mathfrak b$)\\
		Falls $\varphi$ surjektiv ist, ist $\varphi$ ein Ring-Homomorphismus.
		\item Sei $\varphi$ surjektiv. Dann sind die Abbildungen
		\begin{align*}
		\{\text{$\mathfrak a\subseteq A$ Ideal mit $\Ker(\varphi)\subseteq\mathfrak a$}\}&\leftrightarrow \{\text{$\mathfrak b\in B$ Ideal}\}\\
		\varphi^{-1(a)}&\leftrightarrow \mathfrak b\\
		\mathfrak a&\leftrightarrow \varphi(a)
		\end{align*}
		zueinander Inverse Bijketionen.
	\end{enumerate}
\end{rem}

\begin{definition}
	Sei $A$ Ring
	\begin{enumerate}
		\item Das Ideal $\mathfrak p\subseteq A$ heißt \textbf{Primideal} falls $A/g$ Nullteilerfrei ist.\\
		(Äquivalent: $\mathfrak p\subsetneq A$ und für alle $a,b\notin\mathfrak p$ gilt $ab\notin\mathfrak p$)
		\item Das Ideal $m\subseteq A$ heißt \textbf{maximales Ideal}, falls $A/m$ ein Körper ist.\\
		(Äquivalent: Es gibt kein Ideal $\mathfrak a$, sodass $m\subsetneqq\mathfrak m\subsetneqq A$).
	\end{enumerate}
	Jedes Maximale Ideal ist Primideal.
\end{definition}

\begin{satz}
	Sei $A$ Ring, $\mathfrak a\subsetneqq A$ Ideal.\\
	Dann existiert ein maximales Ideal $m\subset A$ mit $\mathfrak a\subseteq m$.
\end{satz}
\begin{proof}
	Sei $(I,\leq)=\left(\{\text{$\mathfrak b\subsetneqq A$ Ideal}\mid\mathfrak a\subseteq b\},\leq \right)$\\
	Zu zeigen: $(I,\leq)$ besitzt maximale Elemente:
	\begin{itemize}
		\item $\mathfrak a\in I\Rightarrow I\neq \emptyset$ erfüllt.
		\item Sei $S\subseteq I$ total geordnet und sei $\mathfrak a_0=\bigcup_{\mathfrak b\in S}\mathfrak b\subseteq A$.\\
		Seien $x,y\in\mathfrak a_0$, also existieren $\mathfrak b,\mathfrak b'\in S$, sodass $x\in \mathfrak b,y\in\mathfrak b'$.\\
		Sei O.E. $\mathfrak b\subseteq\mathfrak b'$, dann gilt, da $S$ total geordnet ist, dass $x+y\mathfrak b\leq \mathfrak a_0$.\\
		Es gilt $\mathfrak a_0\neq A$: Angenommen $\mathfrak a_0=A$, dann $1\in \mathfrak a_0$, dann gibt es $b\in S$ mit $1\in\mathfrak b$. dann folgt $b=A$.\\
		Dann folgt mit \ref{104LemZorn}, dass es ein maximales Elemente gibt, also maximale Ideal die $\mathfrak a_0$ enthalten.
	\end{itemize}
\end{proof}


\begin{lem}[Lemma von Zorn]\label{104LemZorn}
	Sei $(I,\leq)$ eine partielle geordnete Menge.\\
	Für jede total geordnete Teilmenge $S\subseteq I$  eine obere Schranke (d.h. $\exists i\in I$ mit $s\leq i\forall s\in S$).\\
	Dann beseitzt $(I,\leq)$ maximale Elemente (d.h. Elemente, sodass für Elemente $i\in I$ gilt, dass $i_0\leq i,i\neq i_0$).
\end{lem}

\begin{exm}
	Sei $A$ ein Hauptidealring, sei $\mathfrak a\subseteq A$ Ideal mit $\mathfrak a=(a)$f+r $a\in A$.
	\begin{enumerate}
		\item $\mathfrak a$ ist genau dann Primideal, wenn $a$ irrduzibel 
		(d.h. $a\neq 0,a\notin A^\times$ und $a=bc$ für $b,c\in A$, dann muss $b\in A^\times$ oder $c\in A^\times$) 
		oder $a=0$.
		\item Sei $\mathfrak a$, dann ist $a$ irreduzibel oder $A$ ist Körper und $a=0$.
	\end{enumerate}
\end{exm}

\begin{exm}
	Sei $A$ ein Ring. Dann ist $A$ genau dann ein Körper, wenn $\{0\}\subseteq A$ maximal ist.
\end{exm}

\begin{bem}
	Sei $\varphi:A\to B$ ein Ring-Homomorphimsmus
	\begin{enumerate}
		\item Sei $q\subseteq B$ Primideal, dann ist $\varphi^{-1}(q)\subset A$ ein Primideal.
		\begin{proof}[Beweis 1]
			Wir wissen, dass $\varphi$ einen injektiven Ring-Homomorphimsmus $A/\varphi^{-1}\to B/q$ induziert.\\
			Da $B/q$ nullteilerfrei ist, folgt, dass $A/\varphi^{-1}(q)$ nullteilerfrei ist. Dann folgt, dass $\varphi^{-1}(q)$ Primideal ist.
		\end{proof}
		\begin{proof}[Beweis 2]
			InhaltEs gilt $1\notin\varphi^{-1}(q)$. Sei nun $x,y\in A$ mit $x,y\in\varphi^{-1}(q)$, also $\varphi(x),\varphi(y)\notin q$.\\
			Dann folgt, da $q$ Primideal ist, dass $\varphi(xy)=\varphi(x)\varphi(y)\notin q$, also auch $xy\notin\varphi^{-1}(q)$.
		\end{proof}
		\item Sei $\varphi$ surjektiv, dann ist $A/\varphi^{-1}(q)\isomorph B/q$. Also ist
		\begin{enumerate}
			\item $q$ genau dann Primideal, wenn $\varphi^{-1}(q)$ Primideal ist.
			\item $q$ genau dann maximales Ideal, wenn $\varphi^{-1}(q)$ maximales Ideal ist.
			\item Es gibt zueinander Inverse Bijektionen:
			\begin{align*}
			\left\{\substack{\mathfrak a\subseteq A \text{ Primideal/maximales Ideal }\\
				 \text{mit $\Ker(\varphi)=\mathfrak a$}}\right\}
			 &\overset{1:1}{\leftrightarrow}
			 \left\{\substack{\text{Primideal/maximales Ideal}\\
			 	q\subset B}\right\}\\
		 	\mathfrak p&\mapsto \varphi(\mathfrak p)\\
		 	q&\mapsfrom \varphi^{-1}(q)
			\end{align*}
		\end{enumerate}
	\end{enumerate}
\end{bem}


\subsection{Operationen mit Idealen}
Sei im folgende $A$ ein Ring.

\begin{definition}
	\begin{enumerate}
		\item Seien $\mathfrak a,\mathfrak b\subseteq A$ Ideale.\\
		Dann ist die \textbf{Summe von Idealen}
		\[\mathfrak{a+b}:=(\mathfrak{a\cup b})\{a+b|a\in\mathfrak a,b\in\mathfrak b\}\]
		Allgemein für  eine Familie von Idealen $(\mathfrak a_i)_{i\in I}$
		\[\sum_{i\in I}:=\left(\bigcup_{i\in I}\mathfrak a_i\right)\]
		Bzw. das Kleinste Ideal $\mathfrak b$ mit $\mathfrak a_i\subseteq \mathfrak b$ für alle $i\in I$.
		\item Sei $(\mathfrak a_i)_{i\in I}$ eine Familie von Idealen. Dann ist der \textbf{Schnitt von Idealen}
		\[\bigcap_{i\in I}\mathfrak a_i\subseteq A\]
		auch ein Ideal.
		\item Sei $\mathfrak{a,b}\subseteq A$ Ideale.\\
		Dann ist das \textbf{Produkt von Idealen}
		\[\mathfrak{a\cdot b}:=(\{a\cdot b\mid a\in\mathfrak a,b\in\mathfrak b\})=\left\{\sum_{i=1}^{n}a_ib_i\mid n\in\N_0,a_i\in\mathfrak a,b_i\in\mathfrak b\right\}\] 
	\end{enumerate}
	Es folgt, dass \[\mathfrak a\cdot\mathfrak b\subseteq \mathfrak a\cap\mathfrak b\subseteq \mathfrak a,\mathfrak b\subseteq\mathfrak a+\mathfrak b\]
\end{definition}

\begin{exm}
	Sei $A$ ein Hauptidealring, $a,b\in A$ und $a,b\neq 0$.\\
	Dann ist $a=up_1^{k_1}p_2^{k_2}...p_{r}^{k_r}$ und $b=vp_1^{l_1}p_2^{l_2}...p_{r}^{l_r}$ für $u,v\in A^\times$, $p_i\in A$ irreduzibel, $(p_i)\neq p_l$ für $i\neq l$ und $k_i,l_i\in\N_0$.
	\begin{enumerate}
		\item $(b)+(b)=\left(p_1^{\min(k_1,l_1)}...p_{r}^{\min(k_r,l_r)}\right)$\\
		(Ähnlich dem ggT)
		\item $(a)\cap(b)=p_i^{\max k_1,l_1}...p_r^{\max(k_rl_r)}$\\
		(Ähnlich dem kgV)
		\item $(b)(b)=(ab)$ in jedem Ring.
	\end{enumerate}
\end{exm}

\begin{theorem}[Chinesischer Restsatz]
	Seien $\mathfrak a_1,...\mathfrak a_n\subseteq A$ Ideale, sodass\\
	$\mathfrak a_i+\mathfrak a_j=A$ für $i\neq j$. Dann gilt
	\begin{enumerate}
		\item \[\bigcap_{i=1}^{n}\mathfrak a_i=\prod_{i=1}^{n}\mathfrak a_i\]
		\item \begin{align*}
		A/\bigcap_{i=1}^n\mathfrak a_i&\isomfunc \prod_{i=1}^nA/\mathfrak a_i\\
		\ol{a}&\mapsto(a\mod\mathfrak a_1,...,a\mod\mathfrak a_n)
		\end{align*}
	\end{enumerate}
\end{theorem}


\begin{prop}
	Sei $\mathfrak p\subset A$ Primideal mit $\bigcap_{i=1}^n\mathfrak a_i\subseteq \mathfrak p$ für Ideale $\mathfrak a_i,...,\mathfrak a_n\subseteq A$.\\
	Dann ist $\mathfrak a_j\subseteq p$ für ein $j$.
\end{prop}
\begin{proof}
	Angenommen für alle $j=1,...,n$ exitsiert $x_j\in\mathfrak a$, sodass $x_j\notin y$. Dann ist $x_1x_2...x_n\in\mathfrak a_1\cap...\cap\mathfrak a_n$.\\
	Da aber $a_1x_2...x_n\notin\mathfrak p$ da $\mathfrak p$ Primideal. Widerspruch!
\end{proof}
%COUNTER
\addtocounter{thm}{-1}
\begin{prop}
	Sei $\mathfrak a$ ein Ideal, $\mathfrak p_1,...,\mathfrak p_n$ Primideale.\\
	Es gelte $\mathfrak a\not\subseteq \mathfrak p_i$ für alle $i$.\\
	Dann gilt 
	\[\mathfrak a\not\subseteq \bigcup_{i=1}^n\mathfrak p_i \]
	(= kein Ideal)
\end{prop}
\begin{proof}
	Induktion nach $n$:
	\begin{itemize}
		\item $n=1$ erfüllt.
		\item Sei $n>0$.\\
		Induktionsvoraussetzung für $n-1$: Für alle $i\in\{1,...,n\}$ existiere $x_i\in\mathfrak a_i$, sodass $x_i\notin\bigcup_{j\neq i}\mathfrak p_j$
		\item Entweder es existiert ein $i$, sodass $x_i\mathfrak p_i$,\\
		oder für $i$ gilt $x_i\notin\mathfrak p_i$.\\
		Definiere $y\in \mathfrak a$ mit
		\[y:=\sum_{i=1}^{n}x_1x_2...x_{i-1}x_{i+1}...x_n\]
		dann $x\notin\mathfrak p_i$ für alle $i=1,...,n$.
	\end{itemize}
\end{proof}

\subsection{Radikal und Jakobson-Radikal}
Sei $A$ weiterhin ein Ring

\begin{definition}
	\begin{enumerate}
		\item $x\in A$ heißt \textbf{nilpotent}, falls es ein $n\in \N$ gibt, sodass $x^n=0$
		\item $A$ heißt \textbf{reduziert}, wenn er keine nilpotenten Elemente außer $0$ enthält.
	\end{enumerate}
\end{definition}

\begin{exm*}
	\begin{enumerate}
		\item $\ol 2\in\Z/8\Z$ ist nilpotent.
		\item nullteilerfreie Ringe sind reduziert.\\
		Aber: $\Z/6\Z=\Z/3\Z\times \Z/2\Z$ ist reduziert aber nicht nullteilerfrei
	\end{enumerate}
\end{exm*}

\begin{definition}
	Sei $\mathfrak a\subseteq A$ Ideal. Dann heißt das Ideal
	\[\rad(\mathfrak a):=\sqrt{\mathfrak a}:=\{x\in A|\exists n\in \N_0:x^n\in\mathfrak a\}\]
	das \textbf{Radikal} von $\mathfrak a$.
\end{definition}

\begin{bem}
	Sei $\mathfrak a\subseteq A$ ein Ideal
	\begin{enumerate}
		\item $\mathfrak a\subseteq \rad(\mathfrak a)$
		\item $\mathfrak a=\rad(\mathfrak a)$ genau dann wenn $A/\mathfrak a$ reduziert ist
	\end{enumerate}
\end{bem}
\begin{proof}
	Es gilt $\mathfrak a=\rad\mathfrak a$ \\
	genau dann wenn für alle $a\in A$ gilt $0^n\in\mathfrak a$ für ein $n\in\N$. Es folgt $a\in\mathfrak a$.\\
	Genau dann wenn für alle $a\in A$ gilt $\ol{a}^n:=(a\mod\mathfrak a)^n=0$ für ein $n$. Es folgt $\ol a=0$.\\
	Ist also äquivalent dazu, dass $A/\mathfrak a$ reduziert ist.
\end{proof}

\begin{satz}\label{115satz}
	Sei $\mathfrak a\subseteq A$ Ideal. Dann gilt 
	\[\rad(\mathfrak a)=\bigcap_{\substack{\mathfrak g\subset A\text{Primideal}\\\mathfrak a\subseteq \mathfrak g}} \mathfrak g\]
\end{satz}
\begin{proof}
	Wir zeigen durch beidseitige Inklusion
	\begin{itemize}
		\item [$\subseteq$] Sei $x\in A$ nilpotent. Dann gibt es ein $n\in\N$, sodass $x^n=0\in g$ für alle Primideale $g$\\
		Dann liegt auch $x\in g$ für alle Primideale $g$.
		\item [$\supseteq$] Sei $x\in A$ nicht nilpotent
		\begin{enumerate}
			\item Zz: Es gibt ein Primideal $\mathfrak g\subset A$, sodass $x\notin \mathfrak g$.\\
			???...
			%TODO bew verstehen
		\end{enumerate}
	\end{itemize}
\end{proof}

\begin{definition}
	$\Nil(A):=\rad(\{0\})=\{x\in A|\text{$x$ ist nilpotent}\}$ heißt das \textbf{Nilradikal} von $A$.\\
	Mit \ref{115satz} folgt die äquivalente Definition
	\[\Nil(A)=\bigcap_{\substack{\mathfrak g\subset A\\\text{$\mathfrak g$ Primideal}}}\mathfrak g\]
\end{definition}


\begin{definition}
	Das \textbf{Jacobson-Radikal} von $A$ ist definiert als
	\[\Jac(A):=\bigcap_{\substack{m\in A\\\text{$m$ maximales Ideal}}}\]
\end{definition}
\begin{exm*}
	\begin{enumerate}
		\item $\Jac(\Z)=\{0\}=\Nil(\Z)$
		\item $\Jac(\Z/8\Z)=2\Z/8\Z$
	\end{enumerate}
\end{exm*}

\begin{prop}
	$\Jac(A)=\{x\in A\mid 1-xy\in A^\times\forall y\in A \}$
\end{prop}
\begin{proof}
	Sei $x\in A$, sodass $y\in A$ existiert mit $1-xy\notin A^\times$ und sei $m\subset A$ maximal, sodass $1-xy\in m$.\\
	Wäre nun $x\in \Jac(a)\subseteq m$, dann $1=1-xy+xy\in m$. Widerspruch!\\
	Sei also $x\notin\Jac(a)$, d.h. es existiert $m\subset A$ mit $x\notin m$.\\
	Dann ist $m+(x)=A$, d.h. es gibt eine Zerlegung der Eins $1=z+yx$.\\
	Es folgt, dass es ein $y\in A$ gibt, sodass $1-xy\in m$ und damit $1-xy\notin A^\times$.
\end{proof}





\section{Polynomringe}


\begin{definition}
	Sei $A^{(\N_0)}:=\{(a_n)_{n\in\N_0}\mid \text{$a_n\in A$, fast alle $a_n=0$}\}$.\\
	Addition und Multiplikation:
	\begin{align*}
	(a_n)+(b_n)&:=(a_n+b_n)\\
	(a_n)\cdot(b_n):=\sum_{k=0}^{n}a_kb_{n-k}
	\end{align*}\\
	Sei nun $X=(0,1,0,...)$. Dann is nur der $n$-te Eintrag von $X^n=1$.\\
	Dann gilt
	\[(a_n)_n=\sum_{n=0}^{\infty}a_nX^n\]\\
	Wir erhalten einen Kommutativen Ring und bezeichnen $A[X]$ als den \textbf{Polynomring} über $A$ in der Unbestimmten $X$.\\
	Mit der Abbildung $A\to A[X],a\mapsto a+0X+0X^2+...$ erhält man eine $A$-Algebra.
\end{definition}

\begin{definition}
	Sei $f=a_nX^n+...+a_1X+a_0\in A[X]$
	\begin{enumerate}
		\item $\deg(f):=\sup\{d\in\N|a_d\neq 0\}$ heißt der \textbf{Grad} von $f$
		(Es folgt $\deg(0)=-\infty$)
		\item $f$ heißt \textbf{normiert}, falls $f=X^n+a_{n-1}X^{n-1}+...+a_1X+a_0$.
		\item $a_0$ heißt \textbf{absoluter Koeffizient} von $f$.
	\end{enumerate}
\end{definition}

\begin{bem}
	Seien $f,g\in A[X]$
	\begin{enumerate}
		\item $\deg(f+g)\leq\max(\deg(f),\deg(g))$
		\item $\deg(fg)\leq\deg(f)+\deg(g)$\\
		(Da Ringe Nullteiler haben können. Gleichheit bei nullteilerfreien Ringen)
		\item $A$ ist genau dann nullteilerfrei wenn $A[X]$ nullteilerfrei ist.
	\end{enumerate}
\end{bem}

\begin{satz}[Division mit Rest]
	Sei $g=a_dX^d+...+a_0\in A[X]$ mit $a_d\in A^\times$.\\
	Dann existieren für alle Polynome $f\in A[X]$ eindeutige $q,r\in A[X]$, sodass $f=qg+r$ mit $\deg(r)<\deg(g)=d$
\end{satz}
\begin{proof}
	\begin{enumerate}
		\item Da $a_d\in A^\times$ ist gilt $\deg(gs)=\deg(g)+\deg(s)$
		\item Eindeutigkeit: Sei $f=qg+r=q'g+r'$ mit $\deg(r),\deg(r')<d$.\\
		Dann folgt, dass $0=(q-q')g+(r-r')$. Und da $\deg(r-r')<d$ muss $q=q'$ und $r=r'$.
		\item Existenz: Induktion nach $\deg(f)$.\\
		\begin{itemize}
			\item [IA] Sei $\deg(f)<d$, dann $f=0g+r$ und $r=f$.
			\item [IV] Für Polynome $f\in A[X]$ mit $\deg(f)\leq n$ sind $r,q$ eindeutig bestimmt.
			\item [IS] Sei $\deg(f)\geq d$
			%TODO bew verstehen
			...
		\end{itemize}
		
	\end{enumerate}
\end{proof}

\begin{definition}
	Definiere rekursiv $A[X_1,...,X_N]:=(A[X_1,...,x_{n-1}])[X_n]$. Also\\
	\[A[X_1,...,X_n]:=\left\{\sum_{k_1,...,k_n}a_{k_1,...,k_n}X_1^{k_1}\cdot...\cdot X_n^{k_n}\vert a\in A\right\}\]
	Elemente der Form $X_1^{k_1}\cdot...\cdot X_n^{k_n}$ heißen \textbf{Monome}.
\end{definition}


%VL 26.10.2016

\begin{bem}
	$A[X_1,...,X_n]$ ist ein freier Modul. Die Monome bilden eine Basis.
\end{bem}

\begin{satz}[Universaleigenschaft des Polynomrings]
	Sei $\phi:A\rightarrow B$ eine $A$-Algebra und seine $b_1,...,b_n\in B$ Elemente. Dann existiert genau ein $A$-Algebra-Homomorphismus $\psi:A[X_1,...,X_n]\rightarrow B$, so dass $\psi(x_i)=b_i$ für alle $i=1,..,n$, nämlich
	\[\psi\underbrace{\left(\sum_{i_1,...,i_n\ge 0}a_{i_1,...,i_n}X_1^{i_1}\cdot...\cdot X_n^{i_1}\right)}_{=:f}=\underbrace{\sum_{i_1,...,i_n\ge 0}\phi(a_{i_1,...,i_n})b_1^{i_1}\cdot...\cdot b_n^{i_n}}_{=f(b_1,...,b_n)}\]
\end{satz}
\begin{bem}
	\begin{align*}
	\operatorname{Im}(\psi)&=\text{kleinste $A$-Unteralgebra die $b_1,...,b_n$ enthält}\\
	&=A[b_1,...,b_n]\subset B
	\end{align*}
\end{bem}
\begin{exm}
	Sei $\phi:A\rightarrow B$ eien $A$-Algebra, $b\in B$. Es existiere ein $g\in A[X]$ mit $g(b)=0$. Sei $g$ nomriert. Dann gilt\\
	\[A[b]=\{f(b)|f\in A[x],\deg(f)<\deg (g)\}\]
\end{exm}
\begin{exm}
	Sei $A=\Q\hookrightarrow\C$,$i\in \C$. \\
	Dann gilt $g(i)=0$ wobei $g=X^3+X=X(X^2+1)$. Es folgt:
	\begin{align*}
	\Q[i]&=\{a_0+q_1i+a_2i^2|a_0,a_1,a_2\in\Q\}\\
	\Q[i]&=\operatorname{Im}(\Q[X]\xrightarrow[X\mapsto i, f\mapsto f(i)]{\psi}\C)
	\end{align*}
	Dann $\tilde{g}\in\Q[X]: \psi(\tilde{g})=0\Leftrightarrow \tilde{g}(i)=0$.\\
	Also $g\in\operatorname{Ker}(\psi)\Rightarrow (g)\subseteq\operatorname{Ker}(\psi)$.\\
In diesem Fall $\operatorname{Ker}\psi=(X^2+1)$.\\
\end{exm}
Begründung von 2.8:
\[(g)\subseteq\operatorname{Ker}\left(A[X]\xrightarrow[f\mapsto f(b)]{\psi}B\right)\]
Also $\psi$ faktorisiert:
\[A[X]/(g)\xrightarrow{\ol{\psi}}A[b]\subseteq B\]
mit $\ol{\psi}$ surjektiv.
\begin{prop}
	Sei $g\in A[X]$ normiert. Dann ist\[\{f\in A[X],\deg(f)<\deg(g)\}\hookrightarrow A[X]\rightarrow A[X]/(g)\]
	bijektiv.
\end{prop}
\begin{proof}
	Gilt, da für alle $f\in A[X]$ genau ein $r\in A[X]$ exitiert mit $\deg(r)<\deg(g)$ mit $f\in r+(g)$
\end{proof}

\section{Tensorprodukte}
	(A) Tensorprodukte von Moduln\\
	(B) Tensorprodukte von Algebren und Basiswechsel\\
	(C) Exaktheitseigenschaften des Tensorprodukts\\
	
	
\subsection{Erinnerung}


\begin{definition}
	Ein $A$-Modul   ist ein Tripel $(M,+,\cdot)$ wobei $(M,+)$ abelsche Gruppe und $\cdot:A\times X\rightarrow M$ eine Skalare Multiplikation ist.
\end{definition}
\begin{bem*}
	Ein $\Z$-Modul entspricht einer ableschen Gruppe.
\end{bem*}
\begin{exm*}
	Sei $I$ eine Menge
	\[A^{(I)}=\{(a_i)_{i\in I}|a_i\in A, a_i=0\text{für fast alle $i\in I$}\}\]
	$A$-Modul mit Addition und Skalarprodukt.\\
	Für $i\in I:e_i\in A^{(I)}$ mit
	\[e_i=\begin{cases}
	\text{1 an der i-ten Stelle}\\
	\text{0 sonst}
	\end{cases}\]
\end{exm*}
\begin{definition}
	Ein $A$-Modul heißt \textbf{frei}, falls $M\approxeq A^{(I)}$ für eine Menge $I$
\end{definition}
\begin{definition}
	Sei $M,N$ $A$-Modul. Dann heißt $u:M\rightarrow N$ $A$-\textbf{linear} oder \textbf{Homomorphismus von $A$-Moduln}, falls
	\[u(am+m')=au(m)+u(m')\forall a\in A,m,m'\in M\]
\end{definition}
\begin{bem*}
	Sei $I$ eine Menge, $M$ ein $A$-Modul $\underline{m}=(m_i)_{i\in I}$ ein Tupel von Elementen $m_i\in M$. Dann Existiert genau eine Abbildung:
	\[A^{(I)}\xrightarrow{u_{\underline m} }M\]
	mit $u_{\underline{m}}(e_i)=m_i$.\\
	$(m_i)_i=\underline{m}$ heißt linear Unabhängig/ Erzeugende-System/ Basis, \\
	falls $u_{\underline{m}}$ injektiv/ surjektiv / bijektiv ist.
\end{bem*}
\begin{bem*}
	Der $A$-Modul M ist endlich erzeugt, genau dann wenn ein $n\in \N$und eine $A$-lineare Surjektion $A^n\rightarrow M$ existieren.
\end{bem*}

\begin{definition}
	Sei $r\in \N_0$, $M_1,...,M_r,P$ A-Moduln.\\
	Eine Abbildung $\al:M_1\times...\times M_r\rightarrow P$ heißt $r$-\textbf{multilinear}, falls sie in jeder Komponente linear ist, d.h. Für alle $i=1,...,r$ gilt:
	\[\al(m_1,...,am_{i}+m_i',m_{i+1},...,m_r)=a\al(m_1,...,m_i,...,m_r)+\al(m_1,...,m_i',...,m_r)\]
	Für alle $m_j\in M_j,m_i\in M_i,a\in A$.\\
	(Insbesondere heißen $r=1$: linear, $r=2$: bilinear)\\
	\\
	Wir definieren
	\[L_a(M_1,...,M_r,P):=\{\al:M_1\times...\times M_r\rightarrow P\mid\text{$\al$ ist $r$-multlinear}\}\]
\end{definition}

%COUNTER
\setcounter{thm}{2}


\begin{satz}
	Sei $r\ge 2$, $M_1,..,M_r$ A-Moduln.\\
	Dann existiert ein $A$-Modul $M_1\otimes_AM_2\otimes_A...\otimes_AM_r$ und eine $r$-multilineare Abbildung $\tau:M_1\times...\times M_r\rightarrow M_1\otimes_AM_2\otimes_A...\otimes_AM_r$, sodass für jede $r$-multilineaer Abbildung:
	\[\al M_1\times...\times M_r\rightarrow P\]
	wobei $P$ ein A-Modul, genau ein A-lineare Abbildung 
	\[\ol\al:M_1\otimes_A...\otimes_AM_r\rightarrow P\]
	existiert.
	\[
	\begin{tikzcd}
		M_1\times...\times M_r \arrow{rr}{\forall\al:\text{r-multilinear}} \arrow{d}{\tau} && P\\
		M_1\otimes_AM_2\otimes_A...\otimes_A M_r \arrow[urr,dashed,"\exists!\ol \al"] &
	\end{tikzcd}
	\]
\end{satz}

%COUNTER
\addtocounter{thm}{-1}
\begin{definition}
	Der $A$-Modul $M_1\otimes_AM_2\otimes_A...\otimes_AM_r$ heißt das \textbf{Tensorprodukt} von $M_1,...,M_r$.
\end{definition}

%VL 31.10.2016
\begin{proof}
	\begin{itemize}
		\item Eindeutigkeit des Tensorprodukts\\
		Seien $(T,\tau:M_1\times...\times M_r\rightarrow T)$ und $(T',\tau')$ Tensorprodukte:
		\[\begin{tikzcd}[column sep=tiny]
			&M_1\times...\times M_r \ar{ld}{\tau} \ar{dr}{\tau'}&\\
			T \ar[rr,dashed,"\exists!v"] &&T' \ar[ll,dashed,"\exists!u"]
		\end{tikzcd}\]
		$u$ existiert aufgrund der universellen Eigenschaft von $(T,\tau)$.\\
		$v$ existiert aufgrund der universellen Eigenschaft von $(T',\tau')$.\\
		\\
		Die Universelle Eigschaft von $(T,\tau)$ zeigt, dass $v\circ u=\id_T$, genauso $u\circ v=\id_T$.
		\item Existenz des Tensorprodukts\\
		\begin{enumerate}
			\item Suche einen $A$-Modul $N$ und eine Abbildung $c:M_1\times...\times M_r\rightarrow R$, sodass
			\[\Hom_A(N,P)\xrightarrow[u\mapsto u\circ\tau]{ }\text{Abb}(M_1\times...\times M_r,P)\]
			Für alle $A$-Moduln $P$. Wähle also $N:=A^{(M_1\times...\times M_r)}$ und\\ $l:M_1\times...\times M_r \to N,i\mapsto e_i$.
				
			\item Wir wollen, dass $(am_1+m_1',m_2,...,m_r)$ und $a(m_1,...,m_r)+(m_1',...,m_r)$ auf das gleiche Element abgebildet werden.\\
			Sei $Q\subseteq N$ der von
			\begin{align*}
			e_{(m_1,...,m_{i-1},am_i+m_i',m_{i+1},...,m_r)}-\left(ae_{(m_1,...,m_i,..,m_r)}+e_{(m_1,...,m_i',...,m_r)}\right)
			\end{align*}
			für alle $i=1,...,r$ und $m_i,m_i'\in M_i$ und $a\in A$ erzeugt Untermodul.\\
			Dann setze $T:=N/Q$. Dann gilt
			\begin{align*}
			\Hom_A(T,P)&=\{u\in \Hom(N,P)|u(Q)=0\}\\
			&=L_A(M_1,...,M_r,P)
			\end{align*}
			mit $\tau:M_1\times...\times M_r\rightarrow N\rightarrow N/Q$.
		\end{enumerate}
	\end{itemize}
\end{proof}
\begin{bem}\label{304bem}
	$e_{(m_1,...,m_r)}\in A^{(M_1\times...\times M_r)}$ bilden ein Erzeugndensystem.\\
	Also bilden auch die $\tau(m_1,...,m_r)=:m_1\otimes...\otimes m_r$ eine Erzeugenden-System des $A-$Moduls $M_1\otimes...\otimes M_r$.\\
	\textbf{Aber:} Nicht jedes Element von $M_1\otimes...\otimes M_r$ ist in dieser Form.\\
	\\
	Also genügt es eine lineare Abbildung $u:M_1\otimes...\otimes M_r\rightarrow P$ auf den erzeugenden $m_1\otimes...\otimes m_r$ mit ($m_i\in M_i$) anzugeben.\\
	Umgekehrt sei $P$ ein A-Modul und es seien Elemente $u(m_1\otimes...\otimes m_r)\in P$ gegeben für alle $m_i\in M_i$.\\
	Genau dann existiert eine $A$-lineare Abbildung $u:M_1\otimes...\otimes M_r\rightarrow P$ mit $m_1\otimes ...\otimes m_r\mapsto u(m_1\otimes ...\otimes m_r)$, wenn für alle $i=1,...,r$, $a\in A$, $m_j\in M_j$ und $m_i'\in M_i$ gilt:
	\[u(m_1\otimes ..\otimes a m_i+m_i'\otimes..\otimes m_r)=a u(m_1\otimes ..\otimes m_i\otimes..\otimes m_r)+u(m_1\otimes ..\otimes a m_i'\otimes..\otimes m_r)\]
\end{bem}
\begin{satz}[Tensorprodukt linearer Abbildungen]
	Seien $M,M',N,n'$ $A$-Moduln, $u:M\rightarrow M',v:N\rightarrow N'$ $A$-lineare Abbildungen.\\
	Dann definiert
	\begin{align*}
	M\otimes_A N&\rightarrow M'\otimes A N'\\
	m\otimes n &\mapsto u(m)\otimes u(n)
	\end{align*}
	eine $A$-lineare Abbildung bezüglich $u\otimes v:M\otimes N\rightarrow M'\otimes N$.
\end{satz}
\begin{proof}
	Zu zeigen: $u(am+m')\otimes v(n)=a(u(m)\otimes v(n))+u(m')\otimes v(n)$\\
	Es gilt da das Tensorprodukt $r$-linear ist.
	\begin{align*}
	u(am+m')\otimes v(n)&=(au(m)+u(n))\otimes v(n)\\
	&=(au(m)\otimes v(n))+u(m')\otimes v(n)
	\end{align*}
	\\
	Außerdem zu zeigen: $u(m)\otimes v(an+n')=a(u(m)\otimes v(n))+u(m)\otimes v(n)$\\
	($\rightarrow $ Genauso.)
\end{proof}
\begin{bem}\label{306bem}
	\begin{enumerate}
		\item $A\otimes_A M\isomorph M$\\
		\item $M\otimes_A N \isomfunc N\otimes_A M, m\otimes n\mapsto n\otimes m$ ist ... von A-Moduln.\\
		\item 
		\begin{align*}
			M\otimes_A N\otimes_A P&\isomfunc (M\otimes_A N)\otimes_A P&&\isomfunc M\otimes_A(N\otimes_AP)\\
			m\otimes n\otimes p&\mapsto (m\otimes n)\otimes p&&(m\otimes(n\otimes P))
		\end{align*}
	\end{enumerate}
\end{bem}
\begin{proof}
	\begin{enumerate}
		\item Sei $u: a\otimes m\mapsto am$,$v: 1\otimes m\mapsfrom m$
		\begin{itemize}
			\item Z.z. $u$ wohldefiniert, d.h. $(a,m)\rightarrow am$ ist bilinear:\\
			Dann $(ba+a')=bam+a'm$ für alle $a,a',b\in A$ und $m\in M$.\\
			Analog gilt Linearität in $m$.\\
			Daraus folgt, dass $u$ $A$-linear ist.
			\item Z.z. $v$ ist wohldefiniert:\\
			analog zu $u$.
			\item Z.z.: $v\circ u=\id_{A\otimes_A M}$:\\
			\[(v\circ u)(a\times m)=v(am)=1\otimes am=a(1\otimes m)=a\otimes m\]
			\item Z.z.: $u\circ v=\id_{M}$:
		\end{itemize}
		\item Es gilt zu zeigen
			\begin{itemize}
				\item Z.z. Wohldefiniertheit, also $(m,n)\mapsto n\otimes m$ ist bilinear
				\item Existenz der Umkehrabbildung $n\otimes m\mapsto m\otimes n$
			\end{itemize}
	\end{enumerate}
\end{proof}


\begin{prop}3.7
	Sei $(M_i)_{i\in I}$ eine Familie von $A$-Moduln, $N$ ein A-Modul:\begin{align*}
	\left(\bigotimes_{i\in I}M_i\right)\otimes_A N\xrightarrow{\sim} \bigotimes_{i\in I}\left(M_1\otimes_A N\right)\\
	(m_i)_{i\in I}\otimes n\mapsto(m_i\otimes n)_{i\in I}
	\end{align*}
\end{prop}
\begin{proof}
	Umkehrabbildung gegeben durch:\[Inhalt..m_i\otimes n\mapsto (m_j)_{j\in I}\otimes n\]
	mit $m_j:=\begin{cases}
	m_i, &j=i\\
	0 &j\neq i
	\end{cases}$
\end{proof}
\subsection{Basiswechsel von Tensorprodukten}
\begin{satz}
	\begin{enumerate}
		\item Sei $M$ ein A-Modul. Dann wird
		\[\varphi^*(M):=B\otimes_A M\]
		zu einerm $B$-Modul mit dem Skalarprodukt
		\begin{align*}
		B\times(B\otimes_A M)&\rightarrow B\otimes_A M\\
		(b,b'\otimes m)&\mapsto bb'\otimes m
		\end{align*}
		\item Sei $U:M\rightarrow M'$ ein Homomorphismus von A-Moduln. Dann ist
		\begin{align*}
		id_B\otimes u:B\otimes M&\rightarrow  B\otimes_A M'\\
		b\otimes m\mapsto b\otimes u(m)
		\end{align*}
		eine B-lineare Abbildung.S
	\end{enumerate}
\end{satz}
\begin{prop}
	Sei $\varphi:A\rightarrow B$ eine A-Algebra.\\
	Sei $M$ ein freier A-Modul. Dann ist $B\otimes_A M$ ein freier B-Modul und
	\[\vartheta_A(M)=\vartheta_B(B\otimes_A M)\]
\end{prop}
\begin{proof}
	Sei $M$ ein freier A-Modul. Dazu ist äquivalent, dass $M\simeq A^{(I)}$.\\
	Daraus folgt, dass
	\begin{align*}
	B\otimes_A M&\simeq B\otimes_A A^{(I)}\\
	&\simeq B\otimes_A\left(\bigoplus_{i\in I}A\right)\\
	&\simeq \left(\bigoplus_{i\in I}B\otimes_A A\right)\\
	&\simeq\bigoplus_{i\in I}B\\
	&=B^{(I)}
	\end{align*}
	Also ist $B\otimes_A M$ frei.
\end{proof}
%VL 02.11.2016
\begin{prop}
	Sei $\mathfrak{a}\subseteq A$ ein Ideal, $M$ ein A-Modul.Setze 
	\begin{align*}
	\mathfrak a\cdot M&=\langle \{am|a\in\mathfrak a,m\in M\}\\
	&=\left\{\sum_{i=1}^{m}a_im_i\mid n\in\N_0,a_i\in\mathfrak a,m_i\in M \right\}\\
	&\subseteq M \quad \text{Untermodul}
	\end{align*}
	Dann ist
	\begin{align*}
	A/\mathfrak a\otimes_A M&\xrightarrow{\sim} M/\mathfrak a M\\
	\ol a\otimes m&\mapsto \ol{am}
	\end{align*}
	ein Homomorphismus von $A/\mathfrak a$-Moduln.
\end{prop}
\begin{proof}$\ol a\oplus m\mapsto \ol{am}$ ist wohldefiniert:
	Zu zeigen:
	\begin{enumerate}
		\item Sei $a'\in A$ mit $\ol{a'}=\ol a\in A/\mathfrak a$.\\
		Dann ist $\ol{am}=\ol{a'm}\in M/\mathfrak aM$.
		Es gilt $\ol{a}'=\ol a$ gena dann wenn es ein $x\imath\mathfrak a$ gibt sodass $a'=a+x$.\\
		Daruas folgt, dass $a'm=am+xm$, und da $xm\in\mathfrak aM$ folgt $\ol{a'm}=\ol{am}$.
		\item $\ol{am}$ is linear in $a$, d.h.
		\[\ol{(ba+a')m}=b\ol{am}+a'\ol m \quad \text{für $a,a'\in A$, $b\in A$}\]
		\item $\ol{am}$ ist linear in $m$, d.h.
		\[\ol{a(bm+m')}=b\ol{am}+\ol{am'}\quad\text{für $m,m'\in M$, $b\in A$}\]
	\end{enumerate}
\end{proof}
\begin{prop}
	Eine Umkehrabbildung ist gegeben durch
	\begin{align*}
	v:M&\rightarrow A/\mathfrak a\otimes_A M\\
	m&\mapsto 1\otimes m
	\end{align*}
\end{prop}
\begin{proof}
	Zu zeigen: $\mathfrak aM\subseteq Ker(v)$, also für alle $x\in\mathfrak a,m\in M$ gilt $v(xm)=0$.
	\[v(xm)=1\otimes xm=\ol{x}\otimes m=0\]
	da $\ol{x}=\ol{0}\in A/\mathfrak a$.\\
	Noch zu zeigen:: $v$ ist Umkehrabbildung zu $\ol a\otimes m\mapsto \ol{am}$.
\end{proof}

%COUNTER
\addtocounter{thm}{-1}
\begin{definition}[Tensorprodukte von Algebren]
	Sei $A\rightarrow  B_1$, $A\rightarrow  B_2$ A-Algebren.\\
	Dann definieren wir auf dem A-Modul $B_1\otimes_A B_2$ eine Multiplikation:
	\begin{align*}
	(B_1\otimes B_2)\times(B_1\otimes B_2)&\rightarrow B_1\otimes B_1\otimes B_2\\
	(a_1\otimes b_2,b_1'\otimes b_2')&\mapsto b_1b_1'\otimes b_2b_2'
	\end{align*}
	und erhalten die $A$-Algebra $B_1\otimes_A B_2$.
\end{definition}
\begin{exm}
	Sei $\varphi:A\to B$ eine A-Algebra und sei $C=A[X_1,...,X_n]/(f_1,...,f_r)$ und $f_i\in A[X-1,...,X_n]$.Dann ist
	\[B\otimes_A A[X-1,...,X_n]/(f_1,...,f_r)=B[X_1,...,X_n]/(\tilde{f}_1,...,\tilde{d}_r)\]
	wobei 
	\[f_i=\sum_{\ul{j}\in \N_0^n}a_{\ul{j}}X^{\ul{j}}\rightarrow \tilde{f}_i=\sum_j\varphi(a_j)\]
\end{exm}
\begin{exm}
	\begin{enumerate}
		\item Sei $A=\Q$, $C=\Q[i]=\{a+b_i|a,b\in\Q\}=\Q[X]/(X^2+1)$
		\item $\R\otimes_Q Q[i]=\R[X]/(X^2+1)=\C$
		\item $C\otimes_Q Q[i]=C[X]/(X^2+1)=\C[X]/(X+i)\times\C[X]/(X-i)\simeq\C\times\C$
	\end{enumerate}
\end{exm}
\begin{exm}
	$A[X]\otimes_A A[Y]=(A[X])[Y]=A[X,Y]$ mit $f\otimes g\mapsto fg$.\\
	Dann ist die Umkehrabbildung
	\[\sum a_{ij}X^iY^j\mapsto \sum_{i,j}(a_{ij}X^i\otimes Y^j)\]
\end{exm}

%COUNTER
\setcounter{thm}{10}

\subsection*{C) Exaktheitseigenschaften}
\begin{definition}[Homomorphismen-Funktor]
	Seien $M,P$ A-Moduln.\\
	Wir Definiere auf $\Hom_A(M,P):=\{u:M\rightarrow P \text{A-linear}\}$ die Struktur eines $A$-Moduls.
	\begin{align*}
	(u+v)(m)&:=u(m)+v(m) &u,v\in\Hom_A(M,P)\\
	(au)(m)&:=au(m)		 &a\in A,m\in M
	\end{align*}
	Sei $u:M\rightarrow M'$ eine A-lineare Abbildung. Wir erhalten die A-lineare Abbildung
	\begin{align*}
	\Hom_A(u,P):\Hom_A(M',P)&\rightarrow \Hom_A(M,P)\\
	w'&\mapsto w'\cdot u
	\end{align*}
	Sei $v:P\rightarrow P'$ eine A-lineare Abbildung. Wir erhalten die A-lineare Abbildung
	\begin{align*}
	\Hom_A(M,v):\Hom_A(M,P)&\rightarrow \Hom_A(M,P')\\
	w'&\mapsto v\cdot w
	\end{align*}
\end{definition}

\begin{rem}
	Eine Sequnez von $A$-lineare Abbildungen\[...\rightarrow M_{i-1}\xrightarrow{u_{i-1}}M_i\xrightarrow{u_{u_i}}M_{i+1}\rightarrow ...\]
	heißt exakt, falls $\text{Ker}(u_i)=\text{Im}(u_{i-1})$
\end{rem}

%COUNTER
\addtocounter{thm}{-1}

\begin{exm*}
	$0\rightarrow M*\xrightarrow{u}M$ ist exakt genau dann wenn $u$ injektiv ist.\\
	$M\xrightarrow{v}M''\rightarrow 0$ ist exakt genau dann wenn $v$ surjektiv ist
\end{exm*}
\begin{satz}
	\begin{enumerate}
		\item Sei $0\rightarrow M'\xrightarrow{u}M\xrightarrow{v}M''(*)$ eine Sequenz von $A-$Moduln.\\
		Dann ist $(*)$ genau dann exakt, wenn für jeden A-Modul $P$ die Sequenz
		\begin{align*}
		\Hom_A(P,(*)):0\rightarrow \Hom_A(P,M')&\rightarrow \Hom_A(P,M)&&\rightarrow \Hom_A(P,M'')\\
		w'&\mapsto u\circ w'&w&\mapsto v\circ w
		\end{align*}
		exakt ist.
		\item Sei $M'\to M\to M''\to 0 (**)$ eine Sequenz von $A$-Moduln. Dann ist $(**)$ genau dann exakt, wenn für jeden $A$-Modul $P$ die Sequenz
		\begin{align*}
		0\to\Hom_A(M'',P)&\to\Hom_A(M,P)&&\to\Hom_A(M',P)\\
		w''&\mapsto w''\otimes v&w&\mapsto w\otimes u
		\end{align*}
	\end{enumerate}
\end{satz}
\begin{proof}Wir beweisen Schrittweise:
	\begin{enumerate}
		\item ``$(*)$ ist exakt $\Rightarrow$ $\Hom_A(P,(*))$ ist exakt``
		\begin{enumerate}
			\item $w'\mapsto u\circ w'$ injektiv:\\
			Sei $w\in\Hom_A(P,M')$ mit $u\circ w'=0$.\\
			Dann ist (da $u$ injektiv) $w'=0$. Also ist $\Ker(w'\mapsto u\circ w')=0.$
			\item $\Img(w'\mapsto u\circ w')\subseteq\Ker(w\mapsto v\circ w)$:\\
			Kompoosition: $w'\mapsto u\circ w'\mapsto \underbrace{(v\circ u)}_{=0}\circ w'$ ist Null.
			\item $\Img(w\mapsto v\circ w)\subseteq\Ker(w'\mapsto u\circ w')$:\\
			Sei $w\in\Hom_A(P,M)$ mit $v\circ w=0$, sodass $\Img(w)\subseteq\Ker(v)=\Img(u)$.
		\end{enumerate}
		\[\begin{tikzcd}
			0\ar{r} & M'\ar{r}{u} & M\ar{r}{v} & M''\\
			&& P \ar[dashed]{lu}{w'?} \ar{u}{w} \ar{ru}{0}&
		\end{tikzcd}\]
		``$\Leftarrow$``
		\begin{enumerate}
			\item $u$ injektiv: Sie $m'\in M$ mit $u(m')=0$, $P:=<m'>=Am'\subseteq M',w':P\rightarrow M'$ Inklusion.\\
			Dann folgt aus $u\circ w'=0$, da $w'\mapsto u\circ w'$ injektiv ist, dass $w'=0$ und damit $P=0$ sodass $m'=0$.
			\item $\Img(u)\subseteq \Ker(v)$. Z.z. $v\circ u=0$.\\
			Wir wissen bereits, dass für alls $A$-Moduln $P$ die Abbildung $\Hom_A(P,M')\to\Hom_A(P,M''),w'\mapsto v\circ u\circ w$ die Nullabbildung ist.\\
			Wähle $P=M'$ und $w'=\id_{M'}$, dann ist $v\circ u=0$.
			\item $\Ker(v)\subseteq\Img(u)$:\\
			Sei $m\in \Ker(v)$, $P:Am\subseteq M$ und sei $w:P\rightarrow M$ eine Inklusion.\\
			Dann ist $v\circ u=0$, d.h. $w\in\Ker(w\mapsto v\circ w)=\Img(w'\mapsto u\circ w')$.\\
			Also exitsiert $w':P\rightarrow M'$ mit $u\circ w'=w$.\\
			Da $u(w'(m))=w(m)=m$ gilt $m\in\Img(u)$.
		\end{enumerate}
		\item  Übung
	\end{enumerate}
\end{proof}


%VL 07.11.2016
\begin{bem}
	\label{bem313}
	Seiene $M,N,P$ A-Moduln. Dann ist
	\begin{align*}
	\Hom_A(M\otimes_A N,P)&=L_A(M,N;P)&\quad (*)\\
	&=\Hom_A(M,\Hom_A(N,P))\\
	(\al:M\times N\rightarrow P)&\mapsto (n\mapsto \al(m,n))
	\end{align*}
	\begin{align*}
	\text{Sei }T_N:(\text{A-Modul})&\rightarrow (\text{A-Modul})\\
	M&\mapsto M\otimes_AN\\
	(u:M\rightarrow M')&\mapsto u\otimes id_N\\
	N_N:(\text{A-Modul})&\rightarrow \text{(A-Modul)}\\
	P&\mapsto \Hom_A(N,P)
	\end{align*}
	Dann besagt $(*)$:
	\[\Hom(T_M(M),P)=\Hom(M,H_N(P))\]
	d.h. $T_N$ ist linksadjungiert zu $H_N$.\\
	Dann ist $T_N$ rechtsexakt und $H_N$ ist linksexakt.
\end{bem}
\begin{prop}
	Sei $M'\xrightarrow{u}M\xrightarrow{v}M''\rightarrow 0$ eine exakte Sequenz von A-Moduln. Dann ist für jeden A-Modul $N$ die Sequenz
	\[M'\otimes N\xrightarrow{u\otimes id_N}M\otimes_AN\xrightarrow{u\otimes id_N}M''\otimes_A N\rightarrow 0\]
	exakt.
\end{prop}
\begin{proof}
	Formal mit \ref{bem313}.\\
	Sei $M'\rightarrow M \rightarrow M''\rightarrow 0$ exakt.\\
	Dann gilt mit $\ref{bem312}$, dass für alle A-Mdouln $P$:
	\[0\rightarrow \Hom_A(M'',H_N(P))\rightarrow \Hom_A(M,H_N(P))\rightarrow \Hom_A(M',H_N(P))\]
	Ist jeweils gleich (\ref{bem313})
	\[0\rightarrow \Hom_A(T_N(M''),P)\rightarrow \Hom_A(T_N(M),P)\rightarrow \Hom_A(T_N(M'),P)\]
	exakt, sodass mit \ref{312}
	\[T_N(M')\rightarrow \underbrace{T_N(M)}_{=M\otimes_AN}\rightarrow T_N(M'')\rightarrow 0\]
	exakt ist.
\end{proof}
\begin{exm}
	Sei $A=\Z$, $u:\Z\xrightarrow{x\mapsto2x}\Z$. \\
	Dann ist $0\rightarrow \Z\xrightarrow{u}\Z$ exakte und $A\otimes_A M=M$.\\
	Aber 
	\begin{align*}
	0\rightarrow \Z\otimes\Z/2\Z&\xrightarrow{u\otimes id_{\Z/2\Z}}\Z\otimes_\Z\Z/2\Z\\
	\Z/2\Z&\xrightarrow{\cdot 2}\Z/2\Z
	\end{align*}
	ist nicht injektiv.
\end{exm}
\section{Lokalisierung}
\subsection{Lokalisierung von Ringen und Moduln}

\begin{definition}
	Eine Teilmenge $S\subseteq A$ heißt \underline{multiplikativ}, falls $1\in S$ uns $s,t\in S\Rightarrow st\in A$.
\end{definition}
\begin{exm}
	\begin{enumerate}
		\item $S=\Z\setminus \{0\}\subseteq A=\Z$
		\item Sei $f\in A$, dann ist $S_f=\{1,f,f^2,...,\}$ eine multiplikative Teilmenge.
		\item Sei $y\subset A$ Primideal. Dann ist $A\setminus y\subset A$ eine multiplikative Teilmenge.
	\end{enumerate}
\end{exm}
\begin{definition}
	Sei $A$ ein Ring, $S\subseteq A$ eine multiplikative Teilmenge.\\
	Definiere auf $A\times S$ eine Äquivalenzrelation durch
	\[(a,s)\sim (b,t):\Leftrightarrow at=bs\]
	Definiere $S^{-1}A:=(A\times S)/\sim$. Die Äquivalenzklasse von $(a,s)$ wird mit $\frac{a}{s}$ bezeichnet.
\end{definition}
\begin{proof}
	Dies ist eine Äquivalenzrelation:
	\begin{itemize}
		\item Refelxivität
		\item Symmetrie
		\item Transitiv: Sei $(a,s)\sim (b,t)$ und $(b,t)\sim (c,u)$\\
		Dann
		\begin{tikzcd}[row sep=small,column sep=small]
			(tvw)au\ar[r,equal,"(!)"]\ar[d,equal] & (tvw)cs \ar[d,equal]\\
			vbswu\ar[r,equal] & wbuvs
		\end{tikzcd}\\
		Also $\frac{a}{s}=\frac{b}{t}$ genau dann esnn es $v\in S$ gibt sodass $vat=vbs$.
	\end{itemize}
\end{proof}
\begin{bem*}
	$S^{-1}A$ ist Ring mit
	\begin{align*}
	\frac{a}{s}+\frac{b}{t}&:=\frac{at+bs}{st}&&\frac{a}{s}\cdot\frac{b}{t}&:=\frac{ab}{st}
	\end{align*}
	Dies ist Wohldefiniert und macht $A^{-1}A$ zu einem kommutativen Ring mit Eins$=\frac{1}{1}$ und Null$=\frac{0}{1}$.\\
	Die Abbildung $\iota:A\rightarrow S^{-1}A,a\mapsto\frac{a}{1}$ ist ein Ringhomomorphismus und heißt \textbf{kanonisch}.
\end{bem*}

\begin{exm*}
	Sei $S=\Z\setminus \{\}\subseteq A=\Z$. Dann ist $S^{-1}A=\Q$.
\end{exm*}

\begin{satz}[Universelle Eigenschaft]
	Sei $S\subseteq A$ eine multiplikative Teilmenge und sei $1:A\rightarrow S^{-1}$ kanonisch. Sei $B$ ein Ring, $\varphi:A\rightarrow B$ ein Ring-Homomorphimsmus mit $\varphi(s)\in B^\times=\{b\in B\mid  \exists c\in B: bc=1\}$ für alle $s\in S$. Dann existiert ein eindeutiger RIngHomomorphismus $\tilde{\varphi}S^{-1 A\rightarrow B}$ mit $\tilde{\varphi}\circ 1=\varphi:$.
	
	\[
		\begin{tikzcd}
		A\ar{r}{\forall\varphi:\varphi(s)\subseteq B^\times}\ar{d}{1} & B\\
		S^{-1}A\ar{ru}{\exists!\tilde\varphi} &
		\end{tikzcd}
	\]
\end{satz}
\begin{proof}
	\underline{Eindeutigkeit} Für $\frac{a}{s}-in S^{-1}A$ muss für $\tilde{\varphi}$ gilt:
	\begin{align*}
	\tilde\varphi\left(\frac{a}{a}\right)&=\tilde{\varphi}\left(\frac{a}{1}\left(\frac{s}{1}\right)^{-1}\right)=\tilde{\varphi}\left(\frac{a}{1}\right)\tilde{\varphi}\left(\frac{s}{1}\right)^{-1}&(*)\\
	&=\varphi(a)\varphi(s)^{-1}
	\end{align*}
	\underline{Eindeutigkeit}
	Definiere $\tilde\varphi$ durch $(*)$\\
	Z.z: $\tilde\varphi$ ist wohldefiniert.
\end{proof}
\begin{bem}
	Sei $S\subseteq A$ eine multilineare Teilmenge.\\
	Dann gilt: $1:A\rightarrow S^{-1}A$ ist injektive $\Leftrightarrow$ S enthält keien Nullteiler.
\end{bem}
\begin{proof}
	\begin{align*}
	&1\text{ ist injektiv }\\
	\Leftrightarrow&\Ker(1)=0\\
	\Leftrightarrow&(\forall a\in A:\frac{a}{1}=1\Rightarrow a=0)
	\Leftrightarrow&(\forall a\in A:\exists s\in S:as=0\Rightarrow a=0)
	\Leftrightarrow&\text{$S$ enthält eine Nullteiler}
	\end{align*}
\end{proof}
\begin{satz}[Lokalisierung von Moduln]
	Sei $S\subseteq A$ ein multiplikative Teilmenge, $M$ ein A-Modul. Definiere auf $M\times S$ eine Äquivalenz Relation:
	\[(m,s)\sim (n,t)\Leftrightarrow\exists v\in S:vtm=vsm\]
	Man erhält den $S^{-1}A$-Modul $S^{-1}M=(M\times S)/\sim$:
	\begin{itemize}
		\item Mit Addition: $\frac{m}{s}+\frac{n}{t}:=\frac{tm+sn}{st}$
		\item Mit Skalarmultiplikation: $\frac{a}{s}\cdot \frac{m}{t}:=\frac{am}{st}$
	\end{itemize}
\end{satz}
\begin{satz}[Lokalisierung als Funktor]
	Sei $u:M\rightarrow  N$ eine A-lineare Abbildung, $S\subseteq A$ ein multiplikative Teilgruppe. Dann ist
	\begin{align*}
	S^{-1}u:S^{-1}M&\rightarrow S^{-1}N\\
	\frac{m}{s}&\mapsto \frac{u(m)}{s}
	\end{align*}
	eine $S^{-1}A$ lineare Abbildung.
\end{satz}
\begin{satz}[Lokalisierung ist exakt]
	InhaltSei $M'\xrightarrow{u}M\xrightarrow{v}M''$ eine exakte Sequenz von A-Moduln, $S\subseteq$ eine multilineare Teilmenge. Dann ist
	\[S^{-1}M'\xrightarrow{S^{-1}u}S^{-1}M\xrightarrow{S^{-1}v}S^{-1}M''\]
	eine exakte Sequnez von $S^{-1}A$ Moduln.
\end{satz}
\begin{proof}
	$v\circ u=0$. Also ist $S^{-1}v\circ S^{-1}u=0$.\\
	Noch zu zeigen: $\Ker(S^{-1}v)\subseteq \Img(S^{-1}u)$.\\
	Sei $\frac{m}{s}\in S^{-1}M$ mit $S^{-1}v\frac{v}{s}=\frac{v(m)}{s}=0$.\\
	Also gibt es $t\in S:tv(m)=v(tm)=0$.\\
	Damit liegt $tm\in\Ker(v)=\Im(u)$.\\
	Also existiert $m\in M:u(m'=tm)$.
	Dann ist $S^{-1}u\left(\frac{m'}{st}\right)=\frac{u(m')}{st}=\frac{m}{s}$ und damit $\frac{m}{s}\in\Img(S^{-1}u)$
\end{proof}
\begin{prop}
	Sei $M$ ein A-Modul, $S\subseteq A$ eine multiplikative Teilmenge, dann ist
	\begin{align*}
	u:S^{-1}A\otimes_A M&\isomfunc S^{-1 M}\\
	\frac{a}{s}\otimes m\mapsto \frac{am}{s}
	\end{align*}
	ist Homomorphismus von $S^{-1}A$-Moduln.
\end{prop}
\begin{proof}
	\begin{enumerate}
		\item $1$ ist wohldefiniert:
		\begin{enumerate}
			\item Z.z. $\frac{a}{s}=\frac{b}{t}\Rightarrow \frac{am}{s}=\frac{bm}{t}$:\\
			Sei $\frac{a}{s}=\frac{b}{t}$. Ist äquivalent dazu, dass es $v\in S$ gibt mit $vat=vbs$.\\
			Dann erfüllt $v$ auch $vatm=vbsm$ für alle $m\in M$, also auch $\frac{am}{s}=\frac{bm}{t}$.
			\item Z.z. $\frac{am}{s}$ ist linear in $\frac{a}{s}$ und in $m$:
		\end{enumerate}
		\item Existenz einer Umkehrabbildung:\\
		Sei $v:S^{-1}M\rightarrow S^{-1}A\otimes_A M, \frac{m}{s}\mapsto\frac{1}{s}\otimes m$.\\
		Aus $\frac{m}{s}=\frac{n}{t}$ folgt, dass auch $\frac{1}{s}\otimes m=\frac{1}{t}\otimes n$. Also ist die Abbildung wohldefiniert.\\
		Zusätzlich gilt $v\circ u=\id_{S^{-1A\otimes_A M}}$ und $u\circ v=\id_{S^{-1}M}$.
	\end{enumerate}
\end{proof}


%VL 09.11.2016
\begin{satz}[Ideal in $S^{-1}A$]
	Sei $S\subseteq A$ eine multilineare Teilmenge.
	\begin{align*}
	\left\{\text{Ideale in A}\right\}
	{\xrightarrow{\mathfrak a\mapsto S^{-1\mathfrak a}}\atop \xleftarrow[b\mapsto \iota^{-1}(b)]{}
	}\left\{\text{Ideale in $S^{-1}A$}\right\}
	\end{align*}
	\[1:A\rightarrow S^{-1}A,a\mapsto \frac{a}{1}\]
	Nicht zu einander invers.
	\begin{enumerate}
		\item Sei $\mathfrak a\subseteq A$ ein Ideal. Dann ist $S^{-1\mathfrak a}=S^{-1}A$ genau dann wenn $\mathfrak a\cap S\neq 0$.\\
		Dann folgt auch, dass $\mathfrak\mapsto S^{-1\mathfrak a}$ ist nur invertierbar , falls $S\subseteq A^\times$.
		\item Für $b\subseteq S^{-1}A$ Ideal gilt:\\
		\[S^{-1}(\iota^{-1}(b))=b\]
		Dann folgt $b\mapsto\iota^{-1}(b)$ ist injektiv und jedes Ideal von $S^{-1}A$ ist von der Form $S^{-1}\mathfrak a$ für einIdeal $\mathfrak a\subseteq A$.
		\item Sei $\mathfrak a\subseteq A$ ein Ideal. Dann gilt:
		Es gibt ein Ideal $b\subseteq S^{-1}A$ mit $\mathfrak a=\iota^{-1(b)}$.\\
		Dies ist Äquivalent dazu, dass kein $s\in S$ ins $A/\mathfrak a$ Nullteiler ist.
		\item Man hat zueinander inverse Bijektionen: \label{4104}
		\[\left\{q\subset S^{-1 A}\mid \text{Primideal}\right\}
		\xrightarrow{q\mapsto\iota^{-1(q)}}\atop\xleftarrow[\mathfrak p\mapsto S^{-1\mathfrak p}]{}\left\{\text{Primiedeale $\mathfrak p\subset A$ mit $\mathfrak p\cap S=\emptyset$}\right\}
		\]
		%TODO 4
	\end{enumerate}
\end{satz}
\begin{proof}
	\begin{enumerate}
		\item $\frac{1}{1}-in S^{/1 A}$ ist genau dann wenn es ein $a\in\mathfrak a,s\in S$ gibt, sodass $\frac{a}{s}=\frac{1}{1}$.
		\begin{align*}
		&\Leftrightarrow \exists a\in\mathfrak a,s,t\in S: ta=ts\\
		&\Leftrightarrow\mathfrak a\cap S\neq 0
		\end{align*}
		\item Sei $\frac{a}{s}\in S^{-1}(\iota^{-1(b)})$.\\
		Ist äquivalent zu $\exists t\in S$ und $b\in A$ mit $\frac{b}{1}\in b$, so dass
		\[\frac{a}{s}=\frac{b}{t}=\frac{b}{1}\frac{1}{t}\]
		$\Leftrightarrow \frac{a}{s}\in b$
		\item Sei $\mathfrak a=\iota^{-1}(b)$ für ein Ideal $b\subseteq S^{-1}A$.
		\begin{align*}
		&\Leftrightarrow\mathfrak a=\iota^{-1}(S^{-1}\mathfrak a)\\
		&\Leftrightarrow A/\mathfrak 
		a\xrightarrow{\ol{a} \mapsto \ol{\left(\frac{a}{1}\right)}}
		S^{-1}A/S^{-1}\mathfrak a=^{\ref{480}}S^{-1}A/\mathfrak a\quad\text{injektiv}\\
		\end{align*}
		(Wende $\ref{480}$ an auf die exakte Sequenz
		\begin{align*}
		0\rightarrow \mathfrak a&\rightarrow A\rightarrow A/\mathfrak a\rightarrow 0
		\intertext{Dann ist auch}
		0\rightarrow S^{-1}\mathfrak a&\rightarrow S^{-1}A\rightarrow S^{-1}(A/\mathfrak a)\rightarrow 0\\
		\end{align*} exakt.)
		Mit $\ref{450}$ gilt äquivalenz dazu, dass kein $s\in S$ ist Nullteiler in $A/\mathfrak a$.
		\item %TODO Proof 4
	\end{enumerate}
\end{proof}
%TODO 4.10, 4.11
\begin{satz}[Universelle Eigenschaft des Quotientenkörpers]
	Sei $\iota:A\rightarrow \Quot(A)$ kanonisch und sei $\varphi:A\rightarrow K$ ein injektiver Ring-Homomorphismus wobei $K$ ein Körper.\\
	Dann existiert genau ein Homomorphismus von Körpern $\tilde{\varphi}:\Quot(A)\rightarrow K$.
\end{satz}
%TODO. Beweis?
\subsection{Lokale Ringe und Restklassenkörper}
\begin{definition}
	Ein Ring $A$ heißt \underline{lokal} wenn er genau ein Maximales Ideal besitzt.\\
	Dann bezeichnet $\mathfrak m_A$ dieses Maximales Ideal.\\
	Der Körper $\kappa(A):=A/\mathfrak m_A$ heißt \underline{Restklassenkörper von $A$}.
\end{definition}
\begin{exm}
	\begin{itemize}
		\item Jeder Körper ist ein lokaler Ring.
		\item Ein Hauptidealring $A$ ist genau dann lokal, wenn bis auf Multiplikation mit Einheiten genau ein irreduzibles Element existiert.\\
		Oder wenn $A$ Körper ist
	\end{itemize}
\end{exm}
\begin{definition}
	Ein lokaler Hauptideal Ring der kein Körper ist, heißt \underline{diskreter Bewertungsring}.
\end{definition}
\begin{exm}
	Sei $\mathfrak p\subset A$ Primideal, $S:=A\backslash \mathfrak p$ multiplikative Teilmenge, $A_{\mathfrak p}:=S^{-1}A$.
	\[\{\text{Primideals in $A-\mathfrak p$}\}\leftrightarrow\{\text{Primideals $q\subset A$ mit $q\subseteq \mathfrak p$}\}\]
	(mit $\ref{4104}$).\\ 
	Also ist $A_{\mathfrak p}$ ein lokaler Ring mit maximalem Ideal $S^{-1}\mathfrak p$.\\
	Der Körper $\kappa(\mathfrak p):=A/S^{-1}\mathfrak p$ heißt \underline{Restklassenkörper in $\mathfrak p$}.
\end{exm}
\begin{bem}
	Seien $q\subseteq\mathfrak p\subset A$ Primideale.\begin{enumerate}
		\item \begin{align*}
		\{\text{Primideale in $A_{\mathfrak p}$}\}&=\{\text{Primideale in $A$, die in $\mathfrak p$ enthalten sind}\}\\
		\{\text{Primideal in $A/q$}\}&=\{\text{Primideal in $A$, die $q$ enthalten.}\}
		\end{align*}
		\item  Sei $S:=S\backsim\mathfrak p$. Dann ist $S^{-1}(A/q)=S^{-1} A/S^{-1}q$ und
		\[\{\text{Primideal in $S^{-1}(A/q)$}\}=\{\text{Primideals in $A$ die zwischen $q$ und $\mathfrak p$ liegen}\}\]
		\item Speziell für $q=\mathfrak p$:
		\begin{align*}
		S^{-1}(A/\mathfrak p)&=\kappa(\mathfrak p)\\
		&=\Quot(A/\mathfrak p)
		\end{align*}
	\end{enumerate}
\end{bem}
\subsection{Spektren}
\begin{rem}
	Ein \underline{Topologischer Raum} ist ein Paar $(X;\mathfrak T)$ wobei $X$ eine Menge, $\mathfrak T\subseteq \mathscr P(X)$, sodass gilt:
	\begin{enumerate}
		\item $\emptyset \in\mathfrak T,X\in\mathfrak T$
		\item Sei $(U_i)_{i\in I}$ eine Familie von Mengen $U_i\in\mathfrak T$ dann gilt $\forall i\in I:\bigcup_{i\in I}U_i\in\mathfrak T$
		\item $U,V\in\mathfrak T$, dann $U\cap V\in\mathfrak T$
	\end{enumerate}
Die Mengen in $\mathfrak T$ heißen \underline{offen}.
\end{rem}
\begin{rem}
	Seine $X,Y$ topologische Räume. Eine Abbildung $f:X\rightarrow Y$ heißt \ul{stetig}, falls $f^{-1}(V)\subseteq X$ ist offen für alle offenen $V\subseteq Y$.
\end{rem}
\begin{rem}
	Sei $(X,\mathfrak T)$ ein topologischer Raum $B\subseteq \mathfrak T$ heißt \ul{Basis der Topologie}, falls jeder offenen Teilmenge Vereinigung von Menge aus $B$ ist.
\end{rem}
\begin{exm}
	Sei $(X,d)$ eien metrischer Raum, dann heißt $U\subseteq X$ offen, falls \[\forall x\in U\exists\epsilon>0:B_\epsilon(x)\{y\in X\mid M(x,y)<\epsilon\}\subseteq U\]Basis der Topologie: $\{B_\epsilon(x)\mid\epsilon\in \R^{>0},x\in X\}$
\end{exm}
\begin{definition}
	Sein topologischer Raum $X$ heißt \underline{Hausdorffsch}, falls $\forall x,y\in X$ mit $x\neq y$ existieren $x\in U\subseteq X$, $y\in V\subseteq X$ offen, sodass $U\cap V=\emptyset$.
	Metrische Räume sind Hausdorffsch.
\end{definition}
\begin{definition}
	Ein topologischer Raum $X$ heißt \ul{quasikompakt}, falls jede offene Überdeckung $(U_i)_{i\in I}$ von $X$ (d.h. $U_i\subseteq X$ offen für alle $i\in I$ mit $\bigcup_{i\in I}U_i=X$) eine endliche Teilüberdeckung besitzt.
	(d.h. $\exists J\subseteq I$ endliche Teilmenge, sodass $\bigcup_{i\in I}U_i=X$.)
\end{definition}
%VL 14.11.2016
\subsubsection{Spielzeugmodell (der Funktionalanalysis)}
Sei $X$ ein kompakter topologischer Raum,
\[A:=A_X:=\xi (X,\C):=\{f:X\to \C\text{ stetig}\}\]
Sei $x\in X$, dann betrachte
\[\mathfrak M_x:=\{f\in A\mid f(x)=0\}\subseteq A\]
Dies ist ein Minimales Ideal, denn
\[A/\mathfrak M_x\isomfunc \C, \ol{f}\mapsto f(x)\]
\begin{satz}
	Die Abbildung 
\begin{align*}
	X&\rightarrow \operatorname{Max}(A):=\{\mathfrak M\subset A\mid \text{maximales Ideal}\}\\
	x&\mapsto\mathfrak M_x
\end{align*}
	ist bijektiv.
\end{satz}
\begin{kor}
	Sei $f\in A$ und für $\mathfrak M_x\in\operatorname{Max}(A)$ sie $f(x)=$ Bild von $f$ in $A/\mathfrak M_x=\C$.
	\begin{align*}
	D(f)&=\{\mathfrak M\in \operatorname{Max}(A)\mid \text{$\ol{f}$ in $A/\mathfrak M$ ist $\neq 0$}\}\\
	&=\{\mathfrak M\in\operatorname{Max}(A)\mid f\notin\mathfrak M\}\\
	&=\sigma(\{x\in X\mid f(x)\neq 0\})
	\end{align*}
\end{kor}
\begin{definition}
	$U\subseteq\operatorname{Max}(A)$ heißt \textbf{offen}, falls $\exists F\subseteq\operatorname{Max}(A)$ mit
	\[U=\bigcup_{f\in F}D(f)\]
	Dies ist die Topologie uf $\operatorname{Max}(A)$.\\
	(Bemerke: $D(f)\cap D(g)=D(fg)$)
\end{definition}
\begin{satz}
	$\sigma$ ist Homomorphismus
\end{satz}
Seien $X,Y$ kompakte topologische Räume, $F:X\rightarrow Y$ stetig.\\
Mann erhält den $\C-$Algebra-Homomorphismus:
\begin{align*}
\varphi:A_Y&\rightarrow A_x\\
f&\mapsto f\circ F
\end{align*}
Habe Kommutierendes Diagramm
\[\begin{tikzcd}
	X \ar[d,"F"] \ar[d,"\sigma"',"\sim"] & & Y\ar[d,"\sigma"',"\sim"]\\
	\operatorname{Max}(A_x) \ar[rr,"\mathfrak M\mapsto\varphi^{-1}(\mathfrak M)"] && \operatorname{Max}(A_Y)
\end{tikzcd}\]

Es folgt $\forall \mathfrak M\subset A_x$ maximal, sodass $\varphi^{-1}(\mathfrak M)\subset A$ maximal ist.\\
%TODO Kommentar
\begin{def}
	\label{satz414}
	Sei $A$ ein Ring. Setze $X=\Spec(A):=\{y\subset A\mid\text{Primideal con $A$}\}$ als das \textbf{Spektrum von $A$}.\\
	Für $x\in X$ bezeichne $y_x\subset A$ das korrespondierene Primideal.
	Sei $f\in A$, $x\in X$. Dann definiere
	\[f(x):=\text{Bild von $f$ unter}A\rightarrow A/y_x\hookrightarrow\Quot(A/y_x)=\kappa(x)\]
\end{def}
\begin{bem}
	$f$ ist keine Funktion $X\rightarrow ?$.\\
	Seetze
	\begin{align*}
	D(f):&=\{x\in X\mid f(x)\neq 0\}\\
	&=\{x\in X\mid f\notin y_x\}
	\end{align*}
\end{bem}
\begin{definition}
	Eine Teilmenge $U\subseteq X=\Spec(A)$ heißt \textbf{offen}, falls $F\subseteq A$ Teilmenge existiert, sodass $U=\bigcup_{f\in F}D(f)$.\\
	Wir erhalten die sogenannte \textbf{Zanski-Topologie}.
	Dabie
	\begin{align*}
	D(f)\cap D(g)&=D(fg)\\
	\emptyset&=D(0)\\
	x&=D(x)
	\end{align*} 
\end{definition}
\begin{kor}[$D(f)$ als Spektrum]
	Sei $f\in A$ und sei $S_f:=\{1,f,f^2,...,\}$. Dann ist\begin{align*}
	\Spec(S^{-1}_fA)&=\{y\in\Spec(A)\mid y\cap S_f=\emptyset\}\\
	\{y\in\Spec(A)\mid f\notin y\}\\
	&=D(f)
	\end{align*}
\end{kor}
\begin{satz}[Abgeschlossenen Teilmengen]
	Sei $X=\Spec(A)$, $Y\subseteq X$ Teilmenge. Dann
	\[\text{$Y\subseteq X$ abgeschlossen}
	\Leftrightarrow \text{$X\setminus Y\subseteq X$ offen}
	\Leftrightarrow \exists F\subseteq A:X\setminus Y=\bigcup_{f\in f}D(f)\]
	Genau dann wenn
	\begin{align*}
	&\exists F\subseteq A&Y&=\bigcap_{f\in F}(X\setminus D(f))\\
	&&&=\bigcap_{f\in F}\{y\in A\mid f\in y\}\\
	&&&=\{y\in A\text{ Primideal}\mid(F)\subseteq y\}\\
	\Leftrightarrow&\exists\mathfrak a\subseteq A\text{ Ideal}&Y&=\{y\in A\text{ Primiedeal}\mid\mathfrak a\subseteq y\}\\
	&&&=\Spec(A/\mathfrak a)
	\end{align*}
\end{satz}
\begin{satz}[Funktorialität]
	\label{satz420}
	Sei $\varphi
	A\rightarrow  B$ ein Homomorphismus on Ringen.\\
	Dann ist $\varphi\Spec B\to\Spec(A),q\mapsto \varphi^{-1}(q)$ stetig.
\end{satz}
\begin{proof}
	Für $f\in A$ gilt
	\begin{align*}
	\varphi^{-1}(D(f))&=\{y\in\Spec(B)\mid\varphi(y)\in D(f)\}\\
	&=\{q\subset B\text{ Primideal}\mid\varphi^{-1}(q)\in D(f)\}\\
	&=\{q\subset B\text{ Primideal}\mid f\in \varphi^{-1}(q)\}\\
	&=\{q\subset B\text{ Primideal}\mid\varphi(f)\notin q\}\\
	&=D(\varphi(f))\subseteq\Spec(B)\text{ offen.}
	\end{align*}
\end{proof}
\subsection{Lemma von Nokagama???}
\begin{definition}
	Sei $u:M\rightarrow N$ ein Homomorphismus von $A$-Moduln und sei $(m_1,...,m_r)$ ein Erzeugendensystem von $M$ und $(n_1,...,n_s)$ von N.\\
	Dann exitsiert
	\[T=(t_{ij})_{\substack{1\le i\le s\\ 1\le j\le r}}\in M_{s\times r}(A)\]
	sodass
	\[n(m_j)=\sum_{i=1}^{s}t_{ij}n_i\]
	Dann heißt $T$ eine \textbf{Matrix von $U$ bezüglich $(m_1,...,m_r)$ und $(n_1,...,n_s)$}.
\end{definition}
\begin{bem}
	\begin{enumerate}
		\item T ist nicht eindeutig duch $u$ bestimmt\\
		(es sei denn $(n_1,...,n_s)$ ist Basis)
		\item Nicht jede Matrix in $M_{s\times r}(A)$ ist eine Matrix von $u$ bezüglich $(m_1,...,m_r)$ und $(n_1,...,n_s)$.\\
		(Es sei denn $m_1,...,m_r$ ist Basis von $M$)
	\end{enumerate}
\end{bem}
\begin{rem}
	Sei $T\in M_n(A)=A^{n\times m}$, $n\in \N$.\\
	Dann existiert $S\in M_n(A)$, sodass $TS=ST=\det TI_m$. Dann ist $S=(s_{ij})$
	\[s_{ij}=(-1)^{i+j}\det(T_{ji})\]
	($T$ mit $j$-ter Spalte und $i-$ter Spalte gestrichen.)\\
	$S$ heißt die Adjunkte von $T$.
\end{rem}
\begin{satz}[Cayley-Hamilton]
	\label{423CayHam}
	Sei $M$ ein $A$-Modul, $(m_1,...,m_n)$ ein Erzeugendensystem und sei $u:m\rightarrow M$ eine $A$-Lineare Abbildung. Sei $T\in M_r(A)$ eine Matrix von $u$ bezüglich $(m_1,...,m_r)$.\\
	Setze
	\[\chi_T:=\det\underbrace{(X I_r-A)}_{\in M_r(A[x])}=X^r+a_1X^{r-1}+...+a_{r-1}X+a_r\]
	Dann gilt
	\[\chi_T(u)=u^r+a_1u^{r-1}+...+a_{r-1}+a_r\operatorname{Id}_M=0\in\operatorname{End}_A(M)\]
	\begin{enumerate}
		\item Seo $\mathfrak a\subseteq A$ Idela, sod ass $u(M)\subseteq\mathfrak aM$. Dann $a_i\in\mathfrak a^i\forall i=1,...,r$.
	\end{enumerate}
\end{satz}
\begin{proof}
	$u(M)\subseteq\mathfrak aM$. Es folgt, dass die Koeffizienten von $T$ in $\mathfrak a$ liegen.\\
	$a_i$ ist Summe von $i$-fachen Produkten von Koeffizienten von $T$.\\
	Also $a\in\mathfrak a^i\forall i=1,...,r$.\\
	Sei nun $T^T=(t_{ji})_{i\le i,j\le r}$ aber $u(m_j)=\sum_{i}t_{ji}m_i$.\\
	Dann gilt
	\[\sum_i(u\delta_{ji})-t_{ji}m_i=0\]
	Sei nun
	\[C:=(X\delta_{ji}-t_{ji})_{ji}\in M_r(A[X])\]
	wobei $\chi_T=\det(C)$.\\
	Sei
	\[D:=(d_{jk})_{jk}\]
	Die Adjungte von $C$, also
	\[CD=\chi_TI_r\in M_r(A[X])\tag{$\star\star$}\]
	%VL 16.11.2016
	Betrachte den Homomorphismus $u\in\End_A(A)$
	\[A[X]\xrightarrow[f\mapsto f(u)]{}A[u]=\{f(u)\mid f\in A[x] \}\]
	$A[u]$ ist nun eine kommutative $A$-Algebra. Erhalte
	\begin{align*}
	C(u)&=(u\delta_{ij}-t_{ji})_{i,j}\in M_r(A[u])\\
	C(u)&=(\delta_{kj}(u))_{k,j}
	\end{align*}
	Multipliziere $(\star)$ mit $\delta_{kj}(u)$.
	\[0=\sum_{i=1}^{r}\underbrace{\sum_{j=1}^{r}\delta_{kj}(u)(u\delta_{ji}-t_{ji})}_{\substack{\text{k-te Koeffizienten von}\\DC(u)=\chi_T(u)\delta_{ki}}}m_i\]
	Also ...
	%TODO
\end{proof}
\begin{lem}[Lemma von Nakogama (1. Version)]
	\label{424Nakayana1}
	Sei $M$ eine endlich erzeugter $A-$Modul, $\mathfrak a\subseteq A$ ein Ideal, sodass $M=\mathfrak aM$.\\
	Dann existerit $f\in1+\mathfrak a=\{1+x\mid x\in \mathfrak a\}$, sodass $fM=0$
\end{lem}
\begin{proof}
	Wende \ref{423CayHam} auf $u=\id_M:$
	Mit \ref{423CayHam}.1 Gilt
	\[u^r+a_1u^{r-1}+...+a_{r-1}u+a_r\id=0\]
	mit $a_i\in\mathfrak a^i=\mathfrak a$.\\
	Also ist $f\id_M=0$, wobei
	\[f=1+a_1+a_2+...+a_r\in 1+\mathfrak a\]
	sodass $fM=0$
\end{proof}
\begin{bem}
	%Bild: \Spec(A/\mathfrak a)\subset D(f)\subset Spec(A)\subset M
	(Einschränkung von $A$ auf $\Spec(A/\mathfrak a)$)
	\[...=A/\mathfrak a\otimes_AM=M/\mathfrak aM=0\]
	Da $f\in 1+\mathfrak a$ folgt
	\[\Spec(A/\mathfrak a)\subseteq^{(\star)}D(f)=\Spec(S_{g}^{-1}A)\]
	wobei $S_f=\{1,f,f^2,...\}$, sodass
	\[(\text{Einschränkung von $M$ aus $D_f$})=S_f^{-1}A\otimes_AM=S_f^{-1}M\overset{(\star\star)}{=}0\]
	Zu $(\star)$: Sei $x\in\Spec(A)$.
	\[x\in\Spec(A/\mathfrak a)\Leftrightarrow g(\la)=0\forall g\in\mathfrak a \]
	Also gilt für $f=1+g,g\in\mathfrak a$ und $x\in\Spec(A/\mathfrak a)$:
	\[f(x)=1+g(x)=1\neq 0\]
	\[\Rightarrow \Spec(A/\mathfrak a)\subseteq \{x|f(x)\neq 0\}=D(f)\]
	Zu $(\star\star)$: Sei $M$ endlich erzeugt.\\
	Dann $S^{-1}_fM=0$ genau dann wenn $\exists g\in S_f:gM=0$.\\
	$\Leftrightarrow \exists n\in \N: f^nM=0\Leftrightarrow fM=0$
\end{bem}
\begin{lem}[Lemma von Nakagana (2.Version)]
	\label{425Nakagana2}
	Sei $M$ ein endlich erzeugter $A$-Modul, $\mathfrak a\subseteq\Jac(A)$ ein Ideal mit $M=\mathfrak aM$. Dann $M=0$.
\end{lem}
\begin{proof}
	Sei $\mathfrak a\subseteq \Jac(A)\overset{\ref{(118)}}\Rightarrow1\mathfrak a\subseteq A^\times\overset{\ref{424Nakayana1}}\Rightarrow$
		...%TODO... 
\end{proof}
... %TODO
\begin{exm}
	Sei $A=\Z$, $M=\Z$. Dann ist die $\Z$-lineare Abbildung $M\xrightarrow{\cdot 2}\Z$ injektiv aber nicht bijektiv.
\end{exm}
\begin{satz}
	\label{427}Sei $M$ ein endlich erzeugter A-Modul und sein $U:M\rightarrow M$ eine surjektive A-lineare Abbildung.\\
	Dann ist $u$ ein Isomorphismus.
\end{satz}
\begin{proof}
	Fass $(M,u)$ als $A[X]$ Modul auf durch $X\cdot m:=u(m)$ für $m\in M$.\\
	Dann ist $u$ genau dann surjektiv, wenn $X\cdot M=M$ ist.\\
	Es folgt durch $\ref{424Nakayana1}$ mit $\mathfrak a=(X)$, dass es ein $g\in A[X]$ gibt, sodass $(a+gX)(M)=0$.\\
	Sei $m\in\Ker(u)$, dann
	\[u=(1+gX)(m)=m+\underbrace{g(u)(m)u(m)}_{=0}=m\]
	Also ist $u$ injektiv.
\end{proof}
\section{Noethersche und Artinsche Ringe}
\subsection{Noethersche und Artinsche Moduln}
\begin{lem}\label{501lem}
	...%TODO
\end{lem}
\begin{proof}
	... %TODO
\end{proof}
\begin{definition}
	Ein $A$-Modul heißt \textbf{noethersch}, falls die folgenden äquivalenten Bedingungen erfüllt sind:
	\begin{enumerate}
		\item Jede Aufsteigende Kette von Untermoduln von $M$
		\[N_2\subseteq N_2\subseteq...\subseteq M\]
		wird stationär
		\item Jede Nichtleere Menge von Untermoduln von $M$ beseitzt ein Maximales Element
	\end{enumerate}
	Ein $A$-Modul heißt \textbf{artinsch}, falls die folgenden äquivalenten Bedingungen erfüllt sind:
	\begin{enumerate}
		\item Jede absteigende Kette von Untermoduln von $M$
		\[N_2\supseteq N_2\supseteq...\]
		wird stationär.
		\item Jede Nichtleere Menge von Untermoduln von $M$ beseitzt ein minimales Element.
	\end{enumerate}
\end{definition}

%Zerlegung in zwei Definitionen,
%Synchronisation mit VL Nummerierung
\addtocounter{thm}{-1}


\begin{definition}
	Der Ring $A$ heißt \textbf{noethersch}, wenn er als $A$-Modul noethersch ist.
	Äquivalent dazu sind:
	\begin{enumerate}
		\item Jede aufsteigende Kette von Idealen in $A$ wird stationär.
		\item Jede nichtleere Menge von Idealen in $A$ besitzt eine maximales Element.
	\end{enumerate}
	Der Ring $A$ heißt \textbf{artinsch}, wenn er als $A$-Modul artinsch ist.
	Äquivalent dazu sind:
	\begin{enumerate}
		\item Jede absteigende Kette von Idealen in $A$ wird stationär
		\item Jede nichtleere Menge von Idealen in $A$ besitzt eine minimales Element.
	\end{enumerate}
\end{definition}
\begin{exm}
	\begin{enumerate}
		\setcounter{enumi}{-2}
		\item $0$  ist noethersch und artinsch.
		\item Jeder Körper ist noethersch und artinsch.
		\item $\Z$ ist noethersch:\\
		Sei $\mathfrak a_1\subseteq \mathfrak a_2\subseteq...\quad (\star)$ eine aufsteigende Kette. \\
		Dann $\mathfrak a_1=(x_1)$, $x_1=p_1^{l_1}\cdot...\cdot p_r^{l_r}$.
		\[\{\text{Idealis die $\mathfrak a_1$ enthalten}\}\underset{1:1}{\leftrightarrow}\{\text{Teiler von $x_1$}\}/\{\text{Einheiten}\}\]
		Diese Mengen sind endlich also wird $(\star)$ stationär.\\
		$\Z$ ist nicht artinsch:\\
		Sei $x\in\Z$ $x\neq 0,1,-1$. Dann\\
		\[(x)\supsetneqq(x^2)\supsetneqq(x^^3)\supsetneqq...\]
		ist absteigenden Kette die nicht stationär wird.
		\item Sei $p\in\Z$ eine Primzahl. Dann ist der $\Z$-Modul
		\[\{x\in\Q/\Z\mid \exists n\in\N:p^nx=0\}\]
		artinsch aber nicht noethersch.
		(Wir werden zeigen: $A$ artinscher Ring $\Rightarrow$ noethersch)
		\item Sei $\kappa$ Körper, dann ist $\kappa[T_1,T_2,...]$ nciht noethersch:
		\[(T_1)\subsetneqq(T_1,T_2)\subsetneqq(T_1,T_2,T_3)\subsetneqq...\]
	\end{enumerate}
\end{exm}
%VL 21.11.2016
\begin{satz}
	Sei $M$ ein $A$-Modul.\\
	Dann ist $M$ genau dann noethersch, wenn jeder $A$-Untermodul von $M$ endlich erzeugt ist. (Dann ist auch $M$ endlich erzeugt).\\
	Insbesondere ist $M$ genau dann noethersch, wenn jedes Ideal von $A$ endlich erzeugt ist.
\end{satz}
\begin{kor}
	Jeder Hauptidealring ist noethersch.
\end{kor}
\begin{prop}
	\label{506Prop}
	Sei $0\rightarrow M'\xrightarrow{u}M\xrightarrow{v}M''\rightarrow 0$ eine Exakte Sequenz von $A$-Moduln.\\
	Dann gilt
	\begin{enumerate}
		\item $M$ ist genau dann noethersch, wenn $M',M''$ noethersch.
		\item $M$ ist genau dann artinsch, wenn $M',M''$ artinsch.
	\end{enumerate}
\end{prop}
\begin{proof}
	\begin{enumerate}
		\item ''$\Rightarrow$``:
			Es gilt $M'\isomorph u(M')\subseteq M$. Es folgt $M'$ ist noethersch.\\
			\\
			Sei $N_1\subseteq N_2\subseteq ...\subseteq M''$ eine aufsteigende Kette von Untermoduln von $M''$. Da $M$ noethersch ist, gibt es ein $r\in\N$, sodass $v^{-1}(N_r)=v^{-1}(N_{r+1})=...$.\\
			Da $v$ surjektiv ist gilt dann
			\[n_r=v(v^{-1}(N_r))=v(v^{-1}(N_{r+1}))=N_{r+1}\]
			Also wird $N_1\subseteq N_2\subseteq...$ stationär.\\
			\\
		''$\Leftarrow$``: 
			Sei $M_1\subseteq M_2\subseteq...\subseteq M$ eine aufsteigende Kette von Untermoduln in $M$.\\
			Dann sind auch $u^{-1}(M_1)\subseteq u^{-1}(M_2)\subseteq...\subseteq M'$ und $v(M_1)\subseteq v(M_2)\subseteq...\subseteq M''$ aufsteigende Ketten.\\
			Da $M_,M''$ gibt es $r\in\N$, sodass $u^{-1}(M_r)=u^{-1}(M_{r+1})=...$ und $v(M_r)=v(M_{r+1})=...$.\\
			Dies ist äquivalent ($\star$) dazu, dass $M_r=M_{r+1}=...$. Also ist $M$ noethersch.\\
			\\
		Beweis von $(\star)$:\\
		Seien $P\subseteq Q\subseteq M$ Untermoduln mit $u^{-1}(P)=u^{-1}(Q)$ und $v(P)=v(Q)$, sei $q\in Q$.\\
		Dann existiert ein $p\in P$ mir $v(p)=v(q)$. Dann gilt $v(p-q)=0$, also $p-q\in\Img(u)$.\\
		Dann existier auch $m'\in u^{-1}(Q)=u^{-1}(P)$ mit $u(m')=p-q$ und es gilt $u(m')\in P$, also $q\in P$, also $q=P-u(m')$.\\
		Es folgt, dass $P=Q$.
		\item analoge
	\end{enumerate}
\end{proof}
\begin{kor}
	\label{507Kor}
	Seien $_1,...,M_r$ $A$-Moduln und sei $r\in\N$. Dann gilt
	\begin{enumerate}
		\item $\bigoplus_{i=1}^rM_r$ ist genau dann noethersch, wenn $M_i$ noethersch für alle $i=1,...,r$.
		\item $\bigoplus_{i=1}^rM_r$ ist genau dann artinsch, wenn $M_i$ artinsch für alle $i=1,...,r$.
	\end{enumerate}
\end{kor}
\begin{proof}
	Induktion nach $r$:\\
	Der Fall $r=1$ ist klar. Für $r>1$ betrachte die Sequenz
	
	\[\begin{array}{rcl}
		0\rightarrow M_r &\rightarrow& \bigoplus_{i=1}^{r}M_i\rightarrow 0\\
		             m_r &\mapsto    & (0,...,0,m_r)                      \\
		                 &           & (m_1,...,m_r)         \mapsto      (m_1,...,m_{r-1})
	\end{array}\]
	Mit Proposition \ref{506Prop} folgt die Behauptung.
\end{proof}
\begin{kor}
	Ein Ring $A$ ist genau dann noethersch bzw artinsch, wenn jeder erzeugte $A$-Modul noethersch bzw. artinsch ist.
\end{kor}
\begin{proof}
	Sei $A$ noethersch bzw. artinsch und sei $M$ ein endlich erzeugter $A$-Modul. Dann gilt $M\isomorph A^n/N$ für $n\in\N$ und $N\subseteq A^n$ Untermodul. Dann ist die Sequnez $0\rightarrow N\rightarrow A^n\rightarrow M\rightarrow 0$ exakt.\\
	Mit \ref{507Kor} folgt daraus dass $A$ noethersch ist auch dass $A^n$ noethersch ist.\\
	Mit \ref{506Prop} folgt dann dass auch $M$ noethersch ist.
\end{proof}
\begin{kor}
	\label{509Kor}
	Sei $A$ noethersch bzw artinsch und $\mathfrak a\subseteq A$ ein Ideal, dann ist $A/\mathfrak a$ noethersch bzw artinsch.
\end{kor}
\begin{bem}
	Sei $A$ noethersch bzw artinsch und $S$  eine $A$ multiplikative Teilmenge.\\
	Dann ist $S^{-1}A$ noethersch bzw artinsch.
\end{bem}
\begin{proof}
	%TODO Beweis aus Übung
	Beweis in Übung.
\end{proof}


\subsection{Länge von Moduln}

\begin{definition}
	Sei $G$ eine Gruppe und sei $R$ ein (nicht notwendig kommutativer) Ring, sei $M$ ein $R-$(links-)Modul.\begin{enumerate}
		\item Eine \textbf{Kompositionsreihe von $G$} (bzw \textbf{von M}) ist eine Folge $G=G_0\supsetneqq G_1\supsetneqq...\supsetneqq G_r=1$ von Untergruppen, sodass für alle $i=1,...,r$ die Gruppe $G$ ein Normalteiler von $G_{i-1}$ ist.\\
		(Analog für die Folge $m=M_0\supsetneqq M_1\supsetneqq...\supsetneqq M_r=0$ von $R$-Untermoduln)\\
		Dann heißt $r\in \N_0$ die \textbf{Länge der Kompositionsreihe}.
		\item $G$ heißt \textbf{einfach} falls $G\neq\{0\}$ und falls $\{0\}$ und $G$ die einzigen Normalteiler sind.\\
		$M$ heißt \textbf{einfach}, falls $M\neq 0$ und falls $0$ und $M$ die einzigen Untermoduln sind.
		\item Eine Kompositionsreihe heißt \textbf{maximal} oder \textbf{Jordan-Hölder Reihe} falls keine echten Normalteiler (bzw. Untermoduln) eingefügt werden können.\\
		(Äquivalent: $G_{i+1}/G_1$ bzw. $m_{i+1}/M_i$ sind einfach für alle $i=1,...,r$)
	\end{enumerate}
\end{definition}
\begin{bem}
	\begin{enumerate}
		\item Normalerweise existiert keine Jordan-Hölder-Reihe
		\item Sei $R=K$ Körper und sei $V$ ein $K$-Vektorraum. Dann ist $V$ genau dann einfach, wenn $\dim_K(v)=0$.\\
		Sei $(v_1,...,v_r)$ eine Basis von $V$, dann ist $V=\langle v_1,...,v_r\rangle\supsetneqq\langle v_1,...,v_{r-1}\rangle\supsetneqq...\supsetneqq \langle v_1\rangle \supsetneqq 0$ eine JH-Reihe.
		\item Jede Endliche Gruppe besitzt eine JH-Reihe.
	\end{enumerate}
\end{bem}

%Synchronisation mit VL Nummerierung
\addtocounter{thm}{-1}
%Ausgliedern des Bsp
\begin{exm}
	Sei $R=\Z=M$ dann kann man in jede Folge $\Z=n_o\Z\supsetneqq n_1\Z\supsetneqq...\supsetneqq n_r\Z=0$ mit $n_0=1, n_1>1,n_r=0$ zwischen $n_{r-1}\Z$ und $n_r\Z$ die Untergruppe $2n_{r-1}\Z$ einfügen.
\end{exm}

\begin{prop}
	Sei $A$ kein kommutativer Ring, $M$ ein A-Modul, dann gilt $M$ ist genau dann ein einfacher $A$-Modul wenn $M\isomorph A/m$ für maximales Ideal $m\subset A$.
\end{prop}

\begin{proof}
	''$\Leftarrow$``: gilt, da $A/m$ Körper.\\
	''$\Rightarrow$``: Sei $M$ einfach ,$x\in M$, $x\neq 0$. Dann ist $Ax=M$ also ist $u:A\rightarrow M,x\mapsto ax$ surjektiv. Damit ist für $\mathfrak a=\Ker(u)$, dass $M\isomorph A/\mathfrak a$. Da
	\[\{\text{Untermoduln von $A/\mathfrak a$}\}\underset{1:1}{\leftrightarrow} \{\text{Ideale $b\subseteq A$ mit $b\supseteq \mathfrak a$}\}\]
	muss $\mathfrak a$ maximal sein.
\end{proof}

\begin{satz}[Satz von Jordan-Hölder (simple Variante)]
	Sei $G$ eine Gruppe (bzw. $R$ ein nicht notwendig kommutativer Ring und $M$ ein $R$-Modul). Dann besitzen je zwei JH-Reihen von $G$ bzw. $M$ dieselbe Länge.\\
	In diesem Fall kann jede Kompositionsreihe zu einer JH-reihe ergänzt werden.
\end{satz}

\begin{bem*}[Satz von Hölder (genaue Variante)]
	Seien $G=G_0\supsetneqq\supsetneqq G_1\supsetneqq ...\supsetneqq G_r=1$ und $G=G_0'=\supsetneqq\supsetneqq G_1'\supsetneqq ...\supsetneqq G_s'=1$ JH-Reihen.\\
	Dann ist $r=s$ und es existieren Permutationen $\sigma\in S_r$, sodass $G_{i-1}/G_i\isomorph G'_{\sigma(i)-1}/G'_{\sigma(i)}$.
\end{bem*}

\begin{definition}
	Sei $G$ eine Gruppe. Dann heißt
	\[l(G):=\begin{cases}
	\infty&\text{$G$ beseitzt keine JH-Reihe}\\
	r&\text{$G$ beseitzt eine JH-Reihe der Länge $r$}
	\end{cases}\]
	die \textbf{Länge von $G$}.\\
	\\
	Sei $M$ eine $R$-Modul. Dann heißt
	\[l(M):=\begin{cases}
	\infty&\text{$M$ beseitzt keine JH-Reihe}\\
	r&\text{$M$ beseitzt eine JH-Reihe der Länge $r$}
	\end{cases}\]
	die \textbf{Länge von $M$}.
\end{definition}

\begin{bem*}
	Dabei ist $l(M)=1$ genau dann wenn $M$ einfach und $l(M)=0$ genau dann wenn $M=0$.
\end{bem*}

%VL 23.11.2016
\begin{proof}
	(für Moduln, für Gruppen analog)\\
	Sei $M$ ein $R$-Modul.\\
	Setze $l(M):=\inf\{\text{Längen von JH-Reihene von $M$}\}\in \N_0\cup\{\infty\}$
	\begin{enumerate}
		\item $N\subseteq M$ Untermodul $\Rightarrow$ $l(N)\leq l(M)$.\\
		Falls $l(M)=\infty$.\\
		Man kann also annehmen, dass $M$ eine JH-Reihe $M=M_0\supsetneqq M_1\supsetneqq...\supsetneqq M_r=0$ beitzt mit $r=l(M)$.\\
		Sei $N_i:=N\cap M_i$, $\forall i=0,...,r$.\\
		Die Einbettung $N_{i-1}/N_i\hookrightarrow M_{i-1}/M_i$ ist injektiv, da $M_i\cap N_{i-1}=N_i$.\\
		Daraus folgt (da $M_{i-1}/M_i$ einfach ist), dass $N_{i-1}/N_i$ entweder einfach oder $=0$ ist.\\
		Dann kann die Reihe $N=N_0\supseteq N_1\supseteq..\supseteq N_r=0$ durch weglassen einger Terme zu einer JH-Reihe werden.\\
		Dann gilt $l(N)\leq l(M)$.
		
		\item Aus $N\subseteq M$ Untermodul mit $l(N)=l(M)<\infty$ folgt $N=M$:\\
		Wie in $1)$ gilt $M_{i-1}/M_i\isomorph N_{i-1}/N_i$, da $l(N)=l(M)$.\\
		Aus $M_r=N_r=0$ folgt $M_{r-1}=N_{r-1}$ und da $N_{r-2}/N_{r-1}=M_{r-2}/M_{r-1}$ folgt auch $N_{r-2}=M_{r-2}$.\\
		Induktiv gilt damit $N_0=N=M_0=M$
		\item Jede Kompositions Reihe von $M$ besitzte Länge $\leq l(M)$:\\
		($\Rightarrow$ Alle JH-Riehen haben die selbe Länge)\\
		Sei $M=M_0\supsetneqq M_1\supsetneqq...\supsetneqq M_r=0$ eine Kompositions-Reihe.\\
		Aus $1),2)$ folgt $l(M_i)\leq l(M_{i-1})$ für alle $i=1,...,r$. Daraus folgt $s\leq l(M)$.
		
		\item Sei $M=M_0\supsetneqq M_1\supsetneqq...\supsetneqq M_s=0$ eine Kompositions-Reihe, $l(M)<\infty$:\\
		Wenn $s=l(M)$, dann ist $(M_i)$ JH-Reihe. Wenn $s<l(M)$, dann ist $(M_i)$ keine JH-Reihe und die Kompositions-Reihe kann ergänzt werden.
	\end{enumerate}
\end{proof}
\begin{satz} %5.16
	Sei $0\rightarrow M'\xrightarrow{u} M\xrightarrow{v} M''\rightarrow 0$ eine exakte Sequenz von $R$-Moduln. (Dabei ist $R$ nicht notwendiger weise kommutativ) Dann ist $l(M)=l(M')+l(M'')$.\\
	(Insbesondere ist $l(M)<\infty$ genau dann wenn $l(M'),l(M'')<\infty$)\\
	Für Gruppen ergibt sich ein anderes Resultat.
\end{satz}
\begin{proof}
	 Sei $M=M_0\supsetneqq M_1\supsetneqq...\supsetneqq M_r=0$ eine Kompositions-Reihe von $M'$.\\
	 Dann ist $M\supsetneqq u(M')=u(M_0')\supsetneqq...\supsetneqq u(M_r')=0$ eine Kompositions-Reihe und $(M_i'')$ ist eine Kompositionsreihe von $M''$. Dann folgt duch $v^{-1}$, dass es auch eine Kompositionsreihe von $M$.\\
	 Insbesondere folgt aus $l(M')=\infty $ oder $l(M'')=0$, dass $l(M)=\infty$.\\
	 Sei $l(M'),l(M'')<\infty$ und sei
	 $M'=M_0'\supsetneqq M_1'\supsetneqq...\supsetneqq M_r'=0$die JH-Reieh von $M'$ und $M''=M_0''\supsetneqq M_1''\supsetneqq...\supsetneqq M_r''=0$ von $M''$.\\
	 Dann ist
	 \[M=v^{-1}(M_0'')\supsetneqq ...\supsetneqq v^{-1}(M_s'')=\Ker(v)=u(M')\supsetneqq u(M_1')\supsetneqq...\supsetneqq u(M_r')=0\]
	 eine Kompositions-Reihe mit einfachen Subquotienten, also eine JH-Reihe.\\
	 Diese hat Länge $r+s=l(M')+l(M'')$.
\end{proof}
\begin{satz}\label{517Satz}
	Sei $M$ ein $A$-Modul ($A$ ist kommutativer Ring). Dann ist äquivalent:\begin{enumerate}
		\item $l(M)<\infty$ \label{517i}
		\item $M$ ist artinsch und noethersch.\label{517ii}
	\end{enumerate}
\end{satz}
\begin{proof}
	$\ref{517i}\Rightarrow\ref{517ii}$: \\
	Aus$l(My\infty)$ folgt , dass jede nicht stationäre Kette endlich ist und damit $\ref{517ii}$.\\
	$\ref{517ii}\Rightarrow\ref{517i}$: \\
	Sei o.E. $M\neq 0$, $M$ noethersch.\\
	Dann folgt, dass $\{N\subsetneqq M\text{Untermodul}\}$ besitzt maximale Elemente, etwas $M_1$.\\
	Induktiv gilt $M=M_0\subsetneqq M_1\subsetneqq M_2\subsetneqq ...$, woebi $M_{i-1}/M$ ist einfach.\\
	Da $M$ artinsch ist folgt, dass es ein $r\in\N_0$ gibt, sodass $M_r=0$.
\end{proof}
\begin{exm}
	Sei $K$ Körper, $V$ ein $K$-Vektorraum. Dann sind äquivalent:\begin{enumerate}
		\item $\dim_K(V)y\infty$
		\item $l_k(V)y\infty$
		\item $V$ ist noethersch
		\item $V$ ist artinsch
	\end{enumerate}
	Es folgt auch, dass $\dim V=l(V)$.
\end{exm}




\subsection{Noethersche Ringe}
Wenn $A$noethersch, so ist auch $A/\mathfrak a$ noethersch für alle $\mathfrak a\subseteq A$ Ideal und es auch $S^{-1}A$ noethersch für alle $S\subseteq A$ multiplikativ.

\begin{definition}
	Sei $\varphi: A\rightarrow B$ eine $A-$Algebra.\\
	\begin{enumerate}
		\item Die $A$-Algebra $B$ heißt \textbf{endlich erzeugt} oder \textbf{von endlichem Typ}(v.e.T.), wenn $b_1,...,b_n\in B$ existierne, die $B$ erzeugen.\\
		(Äquivalent: $B\isomorph A[X-1,...,X_n]/\mathfrak a$ für $\mathfrak a\subseteq A[X-1,...,_n]$ Ideal.)
		
		\item Die $A-$Algebra $B$ heißt \textbf{endlich}, falls $B$ als $A$-Modul endlich erzeugt ist.
	\end{enumerate}
\end{definition}
\begin{bem}
	Sei $\varphi:A\rightarrow B$ eine $A-$Algebra\begin{enumerate}
		\item $B$ endliche $A$-Algebra,s o folgt, dass $B$ eine $A$-Algebra v.e.T.
		\item Sei $A=K$ Körper, dann ist $K[X]$ eine $K$-Algebra v.e.T., aber $K[X]$ ist nicht endliche $K$-Algebra, da $\dim_K(K[X])=\infty$.
	\end{enumerate}
\end{bem}
\begin{satz}[Hilbertscher Basissatz]
	Sei $\varphi:A\rightarrow B$ eine $A$-Algebra v.e.T. und sei $A$ noethersch.\\
	Dann ist $B$ noethersch.
\end{satz}
\begin{proof}
	\begin{enumerate}
		\item Es gilt $B$ ist genau dann v.e.T. wenn $B\isomorph A[X-1,,...,X_n]/\mathfrak a$.\\
		Also ist o.E. $B=A[X-1,...,X_n]=(A[X-1,...,X_{n-1}])[X_n]$.\\
		Induktiv folgt o.E. $B=A[X]$.
		
		\item Sei $\mathfrak a\subseteq A[X]$ Ideal und sei \\
		$I=\{a\in A\mid\exists f\in\mathfrak a\text{ mit }f=aX^d+(\text{Terme niederen Grades})\}$.\\
		Da $\mathfrak a$ Ideal folgt, dass $I$ Ideal und da $A$ noethersch auch, dass $I$ endlich erzeugt (etwa von $a_1,...,a_n$).\\
		Wähle nun $f_1,...,f_n\in\mathfrak a$, sodass $f_i=a_iX^{r_i}+(\text{Terme niederer Ordnung})$.\\
		Sei nun $\mathfrak a':=(f_1,...,f_n)\subseteq \mathfrak a$ und $r:=\max\{r_i\mid i=1,...,n\}$
		
		\item \label{521iii}Für alle $f\in \mathfrak a$ existiert $g\in\mathfrak a'$, so dass $\deg(f-g)<r$:\\
		Sei $f=aX^m+(\text{Terme niedere Ordnung})$, $s\in I$.\\
		Im Fall $m<r$ folgt die Behauptung.\\
		Falls $m\geq r$ Setze $a=b_1a_+...+b_na_n$ mit $b_i\in A$. Dann hat
		\[f-\underbrace{\sum_{i=1}^{n}b_if_iX^{m-r_r}}_{\in\mathfrak a}\]
		Grad $<m$.\\
		Induktiv folgt die Behauptung.
		
		\item Sei $M=A+AX+...+AX^{n-1}$ eine endlich erzeugter $A$-Modul.\\
		\ref{521iii} bedeutet, dass $\mathfrak a=\mathfrak a'+(\mathfrak a\cap M)$, sodass (da $A$ noethersch) $\mathfrak a\cap M$ als $A$-Modul endlich erzeugt von $g_1,...,g_r$.\\
		Dann ist $\mathfrak a=(f_1,...,f_n,g_1,...,g_r)$.
	\end{enumerate}
\end{proof}

%VL 28.11.2016
\begin{kor}\label{522kor}
	Sei $K$ Körper. Dann ist $K[X_1,...,X_n]$ noethersch.
\end{kor}



\subsection{Artin-Ringe}
\begin{lem}\label{523lem}
	In einem Artinring $A$ ist jedes Primideal ein maximales Ideal.
\end{lem}
\begin{proof}
	Sei $\mathfrak p\subset A$ Primiedeal, dann ist $B:=A/\mathfrak p$ eine nullteilerfreier Artinring.\\
	Behauptung: $B$ ist Körper ($\mathfrak p$ ist maximal).\\
	Sei $x\in B,0\neq x$. Betraahte die Kette $(x)\supseteq(x^2)\supseteq....$.\\
	Da $B$ Artinring ist gibt es ein $n\in\N$,sodass $(x^n)=x^{n+1}$, also $x^n=yx^{n+1}$ für ein $y\in B$.\\
	Daraus folgt (da $x$ kein Nullteiler) dass $1=xy$, also $y\in B^\times$.
\end{proof}
\begin{satz}\label{524Satz}
	Jeder Artinring beseitzt nur endlich viele Primideale.
\end{satz}
\begin{proof}
	Sei $\Sigma:=\{m_1\cap...\cap m_r\mid r\ge 0mm_i\subset A\text{ maximale Ideale}\}$. Dann folgt aus $A\in\Sigma$, dass $\sigma\neq \emptyset$.\\
	Da $A$ artinsch folgt, dass $\Sigma$ ein minimales Element beseitzt (etwa $m_1\cap...\cap m_n$).\\
	Sei $m\subset A$ ein maximales Ideal. Dann ist $m\cap m_1\cap...\cap m_n=m_1\cap...\cap m_n$.\\
	Dann ist $m\supset m_1\cap...\cap m_n=m_1\cdot ...\cdot m_n$. Dann gibt es mit \ref{111} ein $i$, sodass $m\supseteq m_i$. Da $m_i$ minimal folgt, dass es sogar ein $i$ gibt mit $m=m_i$.\\
	Also gilt $\{m\subset A\text{maximales Ideal}\}=\{m_1,...,m_n\}$.\\
	Dann folgt, mit \ref{523lem} die Behauptung.
\end{proof}

\begin{lem}\label{525lem}
	Sei $A$ Artinring, dann exitsiert $k\in N$, sodass $(\Nil(A))^k=0$.
\end{lem}
\begin{proof}
	Da $A$ artinsch, wird $\Nil(A)\supseteq\Nil(A)^2\supseteq...$ stationär.\\
	Also exitsiert ein $k\in\N$, sodass $\Nil(A)^k=\Nil(A)^{k+1}=...=:\mathfrak a$.\\
	Annahme: $\mathfrak a\neq 0$.\\
	Sei $\Sigma=\{\text{$b\supseteq A$ Ideal}\mid b\mathfrak a\neq 0\}$. Dann gtil $A\in\Sigma$. Da $A$ artinsch gibt es ein maximales elemetn $b_0\in\Sigma$.\\
	Sei nun $x\in b_0$ mit $x\mathfrak a\neq 0$. Dann ist $(x)\mathfrak a\neq 0$ und es folgt (da $(x)\subseteq b_0$), dass $(x)=b_0$.\\
	Da auch $(x\mathfrak a)\mathfrak a=x\mathfrak a^2=x\mathfrak a\neq 0$ gilt (da $x\mathfrak a\subseteq(x)$), dass $x\mathfrak a=(x)$.\\
	Also ist $x=xy$ für ein $y\in\mathfrak a=\Nil(A)^k\subseteq\Nil(A)$.\\
	Aber mit $x=xy=xy^2=...$ da $y$ nilpotent folgt $x=0$.
\end{proof}
\begin{theorem}
	Sei $A$ ein Ring dann sind äquivalent\begin{enumerate}
		\item $A$ ist artinsch
		\item $A$ ist noethersch und jedes Primiedeal ist maximal
		\item $l_A(A)<\infty$.
	\end{enumerate}
\end{theorem}
\begin{proof}
	$3)\Rightarrow 1)$: gilt mit \ref{517Satz}\\
	$3)\Rightarrow 2)$: ??? \\%TODO
	$1)\Rightarrow 3)$: Aus \ref{524Satz} folgt, dass es endlich viele maximale Ideale gibt, etwa $m_1\cap...\cap m_n=m_1\cdot...m\cdot m_n$.\\
	Mit \ref{525lem} folgt, dass es ein $k\in\N$ gibt, sodass $m_1^km_2^k\cdot...\cdot m_n^k=\Nil(A)^k=(0)$. Schriebe $(0)=M_1M_2...M_s$ mit $M_i\subset A$ maximal.\\
	Behauptung: Für $j=0,...,s$ gilt $l_A(M_1M_1,...,M_j)<\infty$:\\
	Für $j=s$ gilt die Behauptung.\\
	Für $j\leq s$ ist 
	\[0\rightarrow \underbrace{M_1...M_jM_{j+1}}_{\text{Länge $<\infty$}}
	\rightarrow M_1...M_j
	\rightarrow \underbrace{(M_1...M_j/M_1...M_{j+1})}_{
		\substack{A/M_{j+1}-VR\\
		\text{ist artinsch}\\
		\text{(\ref{518} hat endliche Länge}}}
	\rightarrow 0\]
	Es folgt, dass $l_A(M_1...M_j)<\infty$.\\
	$2)\Rightarrow 3)$: Sei $l_A(A)=\infty$ und Sei $\Sigma:=\{\mathfrak a\subseteq A\mid l_A(A/\mathfrak a)=\infty\}$ mit $(0)\in\Sigma$.\\
	Dann folgt, da $A$ noethersch, dass $\Sigma$ maximales Element $\mathfrak a_0$ besitzt.\\
	Behauptung: $\mathfrak a_0$ ist Primideal.\\
	Sei $a,b\in A:ab\in\mathfrak a_0, a\notin\mathfrak a_0$.\\
	Betrachte nun die exakte Sequenz
	\[0\rightarrow A/\underbrace{\{x\in A\mid xa\in \mathfrak a_0\}}_{=:\mathfrak a'}\xrightarrow{\cdot a}A/\mathfrak a_0\rightarrow \underbrace{A/(\mathfrak a_0+(a))}_{l_A(\cdot)<\infty}\]
Dann folgt $l_A(A/\mathfrak a')=\infty$.\\
Wähle $b\neq \mathfrak a_0$. $\mathfrak '\geq \mathfrak a_0+(b)\supsetneqq \mathfrak a_0$.\\
Dann folgt $l(A/\mathfrak a')<l(A/\mathfrak a_0+(b))<\infty$, da $\mathfrak a_0$ maximal mit $l(A/\mathfrak a_0)=\infty$.\\
Aus dem Wiederspuch folgt,dass $\mathfrak a_0$ ein maximales ideal ist, \\
sodass $l(A/\mathfrak a_0)=1\neq \infty$. Widerspruch!
\end{proof}
\begin{kor}\label{527Kor}
	Sei $A$ ein lokaler Artinring.\\
	Dann $\Spec(A)=\{m\}$m $m=\Nil(A)$ und es gibt ein $k$, sodass $m^k=0$, $A\backslash m=A^\times$.
\end{kor}
\begin{exm*}
	Sei $A$ ein lokaler noetherscher Ring und $m\subset A$ maximal.\\
	Dann gilt für alle $n\ge 1$, dass $A/m^n$ ein lokaler Artinring ist.\\
	Man kann zeigen, dass $\bigcap_{n\ge 1}m^n=\{0\}$.\\
	Definiere eine Metrik auf $A$: $0<\rho 1$,$\rho\in \R$  mit $d(x,y):=\rho ^n$, falls $x-y\in m^n\backslash m^{n+1}$.\\
	Approximation von 
	\[\hat A:=\text{Vervollstädnigung  von $A$ bezüglich $d$ durch $A/m^n$}\]
\end{exm*}
\begin{exm*}
	Sei $\Z(p):=\{\frac{a}{b}\in\Q\mid \text{$p$ teilt nicht $b$}\}$ für $p$ Primzahl.
	%TODO "Märchengeschichte"
\end{exm*}
\begin{satz}[Struktursatz für Artinringe]\label{528Struktursatz}
	Jeder Artinring $A$ ist Produkt von endlichen lokalen Artinringen.
\end{satz}
\begin{proof}
	Seien $m_1,...,m_n\subset A$ die maximalen Ideal.\\
	Dann existier ein $k\in \N$, sodass 0=$m_1^k...m_n^k=m_1^k\cap...\cap m_2^k$.
	Mit derm Chinisischen Restsatz folge, dass
	\[A\isomfunc \prod_{i=1}^{n}\underbrace{A/m_i^k}_{\substack{\text{lokale}\\\text{Artin-Ringe}}}\]
	ist ein Isomorphismus.
\end{proof}




\section{Ganzheit}

\subsection{Ganze Ring-Homomorphismen}


\begin{definition}
	Sei $\varphi:A\rightarrow B$ ein Ring Homomorphismus:
	\begin{enumerate}
		\item Ein Element $b\in B$ heißt \textbf{ganz über $A$}(bezüglich $\varphi$) falls ein normiertes Polynom $f\in A[X]$ exitsiert, sodass $f(b)=b^n+\varphi(a_{n-1})b^{n-1}+...+\varphi(a_0)=0$.
		\item $\varphi$ heißt \textbf{ganz}, falls jedes Elemtn $b\in B$ ganz über $A$ ist.
	\end{enumerate}
\end{definition}
\begin{bem}
	\begin{enumerate}
		\item Sei $\varphi:A\rightarrow B$ ein surjektiver Ring Homomorphismus.\\
		Dann ist $\varphi$ ganz:\\
		Sei $b\in B$. Wähle $a\in A$ mit $\varphi(a)=b$.\\
		Dann $f(b)=0$, wobei $f=X-a$.
		\item Sei $\varphi:A\rightarrow B$ ein Ring-Homomorphismus, $b\in B$.\\
		Dann ist $b$ ganz über $A$ genau dann wenn $b$ ganz über $\varphi(A)$.
	\end{enumerate}
\end{bem}
\begin{exm}\label{603Exm}
	Sei $A$ ein faktorieller Ring, $K=\Quot(A)$. Dann ist $x\in K$ ganz über $A$ genau dann wenn $x\in A$.
\end{exm}
\begin{proof}
	$\Rightarrow$ Sei $x=\frac{1}{b}$ mit $a,b\in A, b\neq 0$, sodass kein Primielement $a$ und $b$ teilt.\\
	Da $x$ ganz ist folgt
	\[\left(\frac{a}{b}\right)^n+a_{n-1}\left(\frac{a}{b}\right)^{n-1}+...+a_1\frac{a}{b}+a_0=0\]
	für $a_0,...,a_{n-1}\in A$:
	Multiplikaiton mit $b^n$ ergibt:
	\[a^n+ba_{n-1}a^{n-1}+...+b^{n-1} a_1a+b^na_0=0\]
	Sei $p$ ein Primteiler von $b$, also $p$ teilt $a^n$. Dann teilt $p$ auch $a$. Widerspruch!\\
	Also $b\in A^\times$, also $x\in A$.
\end{proof}


\begin{exm*}
	Sei $A=\Z$, $x=\frac{1}{2}$, $f(x)=0$ mit $f=2X-1$
\end{exm*}
\begin{bem*}[Anwendung]
	Sei $f=X^n+a_{n-1}X^{n-1}+...+a_0\in \Z[X]$.\\
	Falls $f(x)=0$ für $x\in\Q$, dann $x\in\Z$ und $x$ Teiler von $a_0$.
\end{bem*}

%VL 30.11.2016
\begin{satz}\label{604Satz}
	Sei $\varphi:A\rightarrow B$ ei Ring-Homomorphismus und $b\in B$. Dann ist äquivalent:
	\begin{enumerate}
		\item $b$ ist ganz über $A$.
		\item $A[b]=\{f(b)\mid f\in A[T]\}=\{\sum_{i=1}^{n}\varphi(a_i)b^i\mid a_i\in A,n\in\N \}$ ist eine endliche $A$-Algebra (d.h. $A[b]$ ist als $A$-Modul endlich erzeugt)
		\item $A[b]$ ist in einem Unterring $C\subseteq B$ enthalten, sodass $C$ eine endliche $A$-Algebra ist.
	\end{enumerate}
\end{satz}
\begin{proof}
	\begin{itemize}
		\item 1)$\Rightarrow$2): $b$ ist ganz über $A$, also gibt es $a_i\in A$, sodass $b^n=-(\varphi(a_{n-1})b^{n-1}+...+\varphi(a_0))$. Dann auch
		\[b^{n+r}=-(\varphi(a_{n-1})b^{n-1+r}+...+\varphi(a_0)b^r)\]
		für alle $r\ge 0$. Dann ist $A[b]$ der $A$-Modul, der von $1,b,...,b^{r-1}$ erzeugt wird.
		\item $2)\Rightarrow3)$: $C=A[b]$.
		\item $3)\Rightarrow1)$: Sei $U:C\rightarrow C,c\mapsto bc$. Mit \ref{423CayHam} folgt, dass es $a_i\in A$ gibt, sodass $u^n+a_{n-1}u^{n-1}+...+a_0=0\in\End_A(C)$. Dann ist aber (mit $b=u(1)$)
		\[b^n+\varphi(a_{n-1})b^{n-1}+...+\varphi(a_0)=0\]
	\end{itemize}
\end{proof}

\begin{satz}\label{605Satz}
	Sei $\varphi:A\rightarrow B$ ein Ring Homomorphismus. Dann sind äquivalent:
	\begin{enumerate}
		\item $\varphi$ endlich
		\item $\varphi$ ist von endlichem Typ und ganz
		\item Es gibt $b_1,...,b_n\in B$, sodass $b_i$ ganz über $A$ ist und $B=A[b_1,...,b_n]$
	\end{enumerate}
\end{satz}
\begin{proof} durch Ringschluss:
	\begin{itemize}
		\item $1)\Rightarrow 2)$: nach \ref{604Satz}
		\item $2)\Rightarrow 3)$: Betrachte die Abbildung $A[T_1,...,T_n]\isomfunc B$, wobei $b_i:=\psi(T)$.
		\item $3)\Rightarrow 1)$: Sei $B=A[b_1,...,b_n]$ mit $b_i$ ganz über A.\\
		Wir wissen, dass $A[b_1]$ eine endliche $A$-Algebra ist.\\
		Sei nun $A_k:=A[b_1,...,b_k]$ für $k\leq n$.\\
		Dann ist $A_k=A_{k-1}[b_k]$ 
		%TODO ende des Bew
	\end{itemize}
\end{proof}


\begin{satz}\label{606Satz}
	Seien die Ring-Homomorphismen $\varphi:A\rightarrow B$, $\psi:B\rightarrow C$ ganz.\\
	Dann ist auch $\psi\circ\varphi$ ganz.
\end{satz}
\begin{proof}
	OE (referenz auf bem) $A\subseteq B\subseteq C$. Sei $x\in C$, also existieren $b_0,...,b_{n-1}\in B$ sodass $x^n+b_{n-1}x^{n-1}+...+b_0=0$.\\
	Betrachte nun $B'=A[b_0,...,b_{n-1}]$. Dann ist $B'$ ein endlich erzeugter $A$-Modul und $B'[x]$ ist ein endlich erzeugter $B'$-Modul.\\
	(d.h. es gibt surjektive Abbildungen $A^r\rightarrow B',(B')^k\rightarrow B'[x]$, also auch surjektives $B^{rk}\rightarrow B'[x]$)\\
	Also ist $B'[x]$ ein endlich erzeugter $A$-Modul und damit ist nach \ref{604Satz} $x$ ganz über $A$.
\end{proof}

\subsection{Ganzer Abschluss}
\begin{kor}\label{607Kor}
	Sei $\varphi:A\rightarrow B$ ein Ring-Homomorphismus. Dann ist
	\[C:=\{b\in B\mid\text{$b$ ist ganu über $A$}\}\label{6071Eq}\tag{\ref{607Kor}.1}\]
	ein Unterring von $B$.
\end{kor}
\begin{proof}
	Sei $x,y\in C$. Betrachte $A[x,y]$ (ist nach \ref{605Satz} endliche $A$-Algebra).\\
	Dann ist mit \ref{605Satz} die Abbildung $A\rightarrow A[x,y]$ ganz.\\
	Insbesondere sind $x\cdot y,x\pm y\in A[x,y]$ ganz über $A$.
\end{proof}

\begin{definition}
	\begin{enumerate}
		\item Sei $\varphi A\rightarrow B$ ein Ring-Homomorphismus. Der Unterring $C$ (aus \ref{6071Eq}) wird der \textbf{ganze Abschluss von $A$ in $B$} genannt.
		\item $A$ heißt \textbf{ganz abgeschlossen}, falls $C=\varphi(A)$.
	\end{enumerate}
\end{definition}

\begin{kor}
	Sei $\varphi:A\rightarrow B$ ein Ring,Homomorphismus und sei $C$ der ganze Abschluss von $A$ in $B$, dann ist $C$ ganz abgeschlossen.
\end{kor}

\begin{proof}
	Sei $b\in B$ und $b$ ganz über $C$ (bezüglich der Inklusion $C\subseteq B$). Da $C$ ganz über $A$ist, ist auch $b$ ganz über $A$ (vgl \ref{606Satz}). Also ist $b\in C$.
\end{proof}

\begin{bem}
	Sei $\varphi:A\rightarrow B$ ein ganzer Ring-Homomorphismus, $\mathfrak b\subseteq B$ ein ideal. Dann ist\[A/\varphi^{-1}(\mathfrak b)\rightarrow B/\mathfrak b\]
	auch ganz.
\end{bem}

\begin{satz}\label{611Satz}
	Sei $\varphi:A\rightarrow B$ ein Ring-Homomorphismus, $C\subseteq B$ der Ganze Abschluss von $A$ in $B$ und sei $S\subseteq A$ ein multiplikative Teilmenge.\\
	Dann ist $\varphi(S)^{-1}C$ der ganze Abschluss von $S^{-1}A$ in $\varphi(S)^{-1}B$.\\
	Insbesondere ist $ \varphi(S)^{-1}B $ ganz über $S^{-1}A$, falls $\varphi$ ganz ist.
\end{satz}
\begin{proof}
	OE $A\subseteq B\subseteq C$. Wir zeigen zuerst, dass $S^{-1}C$ ganz über $S^{-1}A$.\\
	Sei dazu $\frac{c}{s}\in S^{-1 C}$. Es existieren $a_i$, sodass $c^na_{n-1}c^{n-1}+...+a_0=0$. Dann ist
	\[\left(\frac{c}{s}\right)^n+\left(\frac{c}{s}\right)^{n-1}\underbrace{\left(\frac{a_{n-1}}{s}\right)}_{\in S^{-1}A}+...+\frac{a_0}{s^n}\]
	ist Ganzheitsgleichung für $\frac{c}{s}$ über $S^{-1}A$, also ist $\frac{c}{s}$ ganz  über $S^{-1}A$.\\
	Sei nun $\frac{b}{s}\sin S^{-1}B$ ganz über $S^{-1}A$, d.h. es gibt $a_i\in A,s_i\in S$, sodass
	\[\tag{$\star$}\left(\frac{b}{s}\right)^n+\frac{a_{n-1}}{s_{n-1}}\left(\frac{b}{s}\right)^{n-1}+...+\frac{a_0}{s_0}=0\]
	Sei $t=s_0\cdot...\cdot s_{n-1}$. Multipliziere ($\star$) mit $(ts)^n$, dann ist
	\[(tb)^n+a_{n-1}x_1(tb)^{n-1}+...+x_n=0\]
	(wobei $x_1,...,x_n\in A$)Ganzheitsgleichung von $t\cdot B$ über $A$.
\end{proof}

\begin{definition}
	Ein Nullteiler freie Ring heißt \textbf{ganz Abgeschlossen}(ohne Spezifizierung worin) oder \textbf{normal}, falls $A$ ganz abegschlossen in $\Quot(A)$.
\end{definition}

\begin{satz}
	Jeder faktorielle Ring ist normal
\end{satz}
\begin{proof}
	in Beispiel \ref{603Exm}.
\end{proof}



\subsection{Going-Up}

\begin{satz}\label{614Satz}
	Sei $B$ ein nulltieiler freier Ring und $A\subseteq B$ ein Unterring und sei $B$ ganz über $A$.\\
	Dann ist $A$ genau dann ein Körper wenn $B$ ein Körper ist.
\end{satz}
\begin{proof}
	\begin{itemize}
		\item Sei $A$ Körper und $y\in B$ mit $y\neq 0$. Nehem Ganzheitsgleichung von $y$ über $A$ mit minimalem Grad:\\
		\[y^n+a_{n-1}y^{n-1}+...+a_0=0\]
		Da $B$ nullteilerfrei ist, gilt $a_0\neq 0$.\\
		(Nehme an , dass $a_0=0$, dann $y(y^{n-1+a_{n-1}y^{n-2}+...+a_1})=0$ also Grad nciht minimal)\\
		Sei $\delta:=-a_0^{-1}(y^{n-1}+a_{n-1}y^{n-2}+...+a_1)\in B$ mit $\delta y=1.$ Also ist $B$ Körper.
		\item Sei nun $B$ Körper, $x\in A\backslash\{0\}$. Es gilt $x^{-1}\in B$, also ganz über $A$.\\
		Also finden wir zur Gleichung $x^{-m}+a_{m-1}x^{-m+1}+...+a_0=0$ durch Multiplikation mit $x^{m-1}$
		\[x^{-1}+\underbrace{a_{m-1}+a_{m-2}+...+a_0x^{m-1}}_{\in A}=0\]
		Also liegt $x^{-1}\in A$.
	\end{itemize}
\end{proof}

\begin{kor}\label{615Kor}
	Sei $\varphi: A\rightarrow B$ eine ganzer RIng-Homomorphismus. Sei $q\subseteq B$ Primideal, $p:=\varphi^{-1}(q)$. Damit ist $q$ maximal gdw $p$ maximal.
\end{kor}
\begin{proof}
	Es gilt $A/p\rightarrow B/q$ ist ganz. Satz \ref{614Satz} gibt uns, dass $A/p$ genau dann Körper ist, wenn $B/q$ Körper ist. Es folgt die Behauptung
\end{proof}


%VL 05.12.2016
\begin{kor}
	Sei $\varphi:A\rightarrow B$ ein ganzer Ring-Homomorphismus, seien $q\subseteq q'\subset B$ Primideale, so dass $p:=\varphi^{-1}(q)=\varphi^{-1}(q')$. Dann gilt $q=q'$
\end{kor}
\begin{proof}
	In $A_p=S^{-1}A,S=A\backslash p$ ist $pA_p$ maximal. Betrachte
	\[
		\begin{tikzcd}[row sep=large,column sep=large]
		A\ar[r,"\varphi"] \ar[d,"a\mapsto\frac{a}{1}"'] 
			& B \ar[d,"b\mapsto\frac{b}{1}"]\\
		A_p \ar[r,"\psi=S^{-1}\varphi"'] 
			& B_p
		\end{tikzcd}
	\]
	Wobei $pA_p\subset A_p$ und $qB_p\subseteq B_p=\varphi^{-1}SB$ und auch $qB_p\subseteq qB_p$ Primideal.\\
	Mit \ref{611Satz} folgt $\psi$ ist ganz.\\
	Also gilt OE $p\subset A$ ist maximal, sodass mit \ref{615Kor} folgt, dass $q,q'$ maximal sind und da $q\subseteq q'$ gilt $q=q'$.
\end{proof}

\begin{satz}\label{617Satz}
	Sei $\varphi:A\rightarrow B$ ein injektiver ganzer Ring Homomorphismus. Dann existiert für jedes Primideal $p\subset A$ ein Primideal $q\in B$ mit $\varphi^{-1}(q)=p$.\\
	(D.h. $\Spec(B)\rightarrow \Spec(A),q\mapsto \varphi^{-1}(q)$ ist surjektiv.)
\end{satz}
\begin{proof}
	Ersetze $A$ durch $A_p$, dann gilt OE, dass $p\subset A$ maximal und $A$ lokal ist.\\
	Da $\varphi$ injketiv ist folgt $B\neq 0$.\\
	Also existiert ein maximales Ideal $q\subseteq B$ und mit \ref{615Kor} ist $\varphi^{-1}(q)$ maximal, also $\varphi^{-1}(q)=p$.
\end{proof}

\begin{theorem}[Going Up]\label{618ThmGoingUp}
	Sei $\varphi:A\rightarrow B$ ein ganzer injektiver Ring-Homomorphismus und seien $n\geq m\geq 0$ ganze Zahlen. Sei $p_i\subsetneqq...\subsetneqq p_m\subsetneqq...\subsetneqq p_n\subset A$ eine Kette von Primidealen und sei $q_1\subseteq...\subseteq q_n\subset B$ eine Kette von Primidealen mit $\varphi(q_i)=p_i$ für $i=1,...,m$.\\
	Dann gilt $q_1\subsetneqq...\subsetneqq q_m$ und es existiert eine Kette von Primidealen $q_1\subsetneqq...\subsetneqq q_m\subsetneqq q_{m+1}\subsetneqq...\subsetneqq q_n\subset B$ mit $\varphi^{-1}(q)=p_i$ für alle $i=1,...,n$.
\end{theorem}
\begin{proof}
	Sei OE $n>m$, $n_1=1,m=0$. Dann folgt mit \ref{617Satz}, dass $q_1\subsetneqq...\subsetneqq q_m$:\\
	Vollständige Induktion: Sei OE $m=1,n=2$. Betrachte
	\[
	\begin{tikzcd}
	A\ar[r,"\varphi"] \ar[d] & B\ar[d]\\
	A/p_1 \ar[r,"\ol{\varphi}"] & B/q
	\end{tikzcd}
	\]
	Wobei $\ol\varphi$ ganz und injektiv ist, da $\varphi^{-1}(q_i)=p_1$ und $p_2/p_1\subseteq A/p_1$.\\
	Dann folgt mit \ref{617Satz}, dass es das Primideal $\ol{q_2}\subset B/q_i$ gibt mit $\ol{\varphi}^{-1}(\ol{q_2})=p_2/p_1$.\\
	Dass ist $\ol{q_2}\isomorph q_2\subset B$, wobei $q_2$ Primideal mir $q_2\supseteq q_1$ und $\varphi^{-1}(q_2)p_2$. 
\end{proof}



\section{Irreduziblität}
\subsection{Satz von Gauß}

\begin{rem}\label{701rem}
\begin{enumerate}
	\item Sei $A$ ein nullteilerfreier Ring. Ein Element $p\in A$ heißt
		\begin{enumerate}
			\item \textbf{irrefuzibel}, falls $0\neq p\notin A^\times$ und falls $p=ab$ mit $a,b\in A$, so gilt $a\in A^\times$ oder $b\in A^\times$.
			\item \textbf{Primelelement}, falls $p\neq 0$ und $(p)$ ist Primideal.\\
		\end{enumerate}
		Es gilt, wenn $p$ Primelement ist, so ist $p$ irreduzibel.
	\item $A$ heißt \textbf{faktoriell}, falls er die folgenden äquivalenten Bedingung erfüllt:
	\begin{enumerate}
		\item Jedes $0\neq a\notin A^\times$ ist Produkt von irreduziblen Elemente und diese Zerlegung ist eindeutig bis auf Reihenfolge und Multiplikation mit Einheiten.
		\item Jedes Elemente $o\neq a\notin A^\times$ ist Produkt von Primelemten.
		\item Jedes Irreduzible Element ist ein Primelement und jede aufsteigende Kette von Hauptidealen wird stationär.
	\end{enumerate} 
\end{enumerate}
\end{rem}
\begin{proof}
	\begin{itemize}
		\item b)$\Rightarrow$a): Einführung in die Algebra (Beweis HIR sind faktoriell)
		\item a)$\Rightarrow$c): 
		\begin{enumerate}
			\item Sei $p\in A$ irreduzibel. Seien $a,b\in A$ mit $ab\in(p)$.\\
			Setze $ab=dp$ mit $d\in A$. Seien $a=p_1...p_r$, $b=q_1...q_s$ und $d=l_1...l_t$ irreduzible Zerlegungen. Dann
			\[p_1..p_rq_1...q_s=pl_1...l_t\]
			Aus der eindeutigkeit folg, dass es ein $i$ gibt sodass $(p)=(p_i)$ oder ein $j$, sodass $(p)=(q_j)$.\\
			Daraus folgt, $p$ teilt $a$ oder $b$.
			\item gibt, dass jdese Elemente $\neq0$ hat nur endlich viele Teiler. (Bis auf Multiplikation mit Einheiten).\\
		\end{enumerate}
		Mit Anderen Worten: Für jedes Hauptideal $\mathfrak a\neq 0$ existieren nur endlich viele Hauptideal, die $\mathfrak a$ enthalten.\\
		$\Rightarrow$ Jede aufsteigende Kette von Hauptidealen wird stationär.
		
		\item c)$\Rightarrow$ b): Sei $\Sigma:=\{(a)\mid\text{$0\neq a\in A^\times$} und $a$ ist nicht Produkt von irreduziblen Elementen \}$.\\
		Angenommen $\Sigma\neq 0$:\\
		Dann folgt mit $\ref{501lem}$ %TODO
	\end{itemize}
\end{proof}

\begin{exm}
	Jeder Hauptidealring ist faktoriell. Insbesondere auch $\Z,K[X]$
\end{exm}
\begin{definition}
	Sei $A$ ein Ring, $f=a_mX^m+...+a_1X+a_0\in A[X]$ heißt \textbf{primitiv}, falls $(a_1,...,a_n)=A$.
\end{definition}
\begin{exm*}
	\begin{enumerate}
		\item Sei $A$ faktoriell. Dann ist $f$ genau dann Primitiv, wenn kein Primelement alle $a_i$ teilt.
		\item Seien $f,g\in A[X]$. Dann sind $f,g$ genau dann primitv, falls $fg$ primitiv.
	\end{enumerate}
\end{exm*}

%Synchronisation. ()
\addtocounter{thm}{-1}
\begin{definition}
	Sei $A$ faktoriell. Ein $c(f)\in A$ heißt \textbf{Inhalt von $f$}, falls $c(f)$ ein größter gemeinsammer Teiler von $a_1,...,a_0$ ist.\\
\end{definition}
\begin{bem*}
	Also ist $g$ genau dann primitav, falls $c(f)\in A^\times$.\\
	Für $f\in A[X]$ gilt, dass $f=c(f)\tilde{f}$ mit $\tilde{f}$ primitv.
\end{bem*}
\begin{bem*}
	Sei $f=3X^{1000}+30X^7+21X+27$, dann $c(f)=3$ oder $-3$.\\
	Dann $f=3\tilde{f}$, also $\tilde{f}=X^{1000}+10X^7+7X+9$.
\end{bem*}

\begin{theorem}[Satz von Gauß]\label{704Gauss}
	Sei $A$ ein faktorieller Ring. Dann ist auch $A[X]$ faktoriell.\\
	Die irreduziblen Elemente von $A[X]$ sind:
	\begin{enumerate}
		\item $p\in A$ irreduzibel und
		\item $f\in A[X]$ primitv, sodass $f\in \Quot(A)[X]$ irreduzibel ist.
	\end{enumerate}
\end{theorem}
\begin{exm*}
	Sei $A=\Z$, 
	\begin{itemize}
		\item $2X+4\in\Z[X]$ ist reduzibel, da $2X+4=2(X+2)$
		\item $X^3-5\in\Z[X]$ ist primitv und irreduzibel in $\Q[X]$
	\end{itemize}
\end{exm*}

%VL 07.12.2016
\begin{proof}
	\begin{enumerate}
		\item Seien $f,g\in K[X]\setminus\{0\}$. Schreibe $f=c(f)\tilde{f}$, $g=c(g)\tilde{g}$ mit $\tilde{f},\tilde{g}$ primitiv. Dann $fg=c(f)c(g)\tilde{g}\tilde f$, sodass $c(fg)=c(f)=c(g)$ gilt.
		\item Behauptung:$p\in A$ ist irreduzibel, dann ist $p\in A[X]$ Primelement:\\
		\[A[X]/pA[X]=(A/p)[X]\]
		ist nullteilerfrei (da $A/p$ nullteilerfrei ist).
		Dann ist $p\in A$ prim.
		\item Sei $q\in A[X]$ primitiv, $q\in K[X]$ irreduzibel.\\
		Behauptung: $qK[X]\cap A[X]=qA[X]$:
		\begin{itemize}
			\item ``$\supseteq$'' ist klar
			\item ``$\subseteq$``: Sei $f\in K[X]$ mit $qf\in A[X]$, sei $f=c(f)\tilde f$ mit $\tilde f$ primitv. Dann gilt $c(qf)\in A$ und $c(qf)=c(q)c(f)$ wobei $c(q)\in A\times$.\\
			Dann folgt, dass $c(q)c(f)=c(f)$ und damit $f\in A[X]$.\\
		\end{itemize}
		Die Behauptung gilt also genau dann wenn $A[X]/qA[X]\rightarrow K[X]/qK[X]$ injektiv ist.\\
		Also ist $q\in A[X]$ Primelemnt.
		\item Jedes $f\in A[X]$ mit $0\neq f\notin A^\times$ ist Produkt der Primelemente von $(a)$ und $(b)$.\\
		Schrieeb $f=c(f)\tilde f$, $c(f)$ ist Produkt von Primelementen in $(a)$ und $\tilde{f}$ ist primitv.\\
		Sei $\tilde{f}=g_1,...,g_r$ mit $g_i\in K[X]$ irreduzibel, $g_i=c_i\tilde{g}_i$, $c_i\in K^\times$, $\tilde g_i$ primitiv.\\
		Es folgt, dass $\tilde f=c_1...c_r\tilde g_1...\tilde g_r$.\\
		Da $c(\tilde f)\in A^\times$ und $c(\tilde g_1...\tilde g_r)\in A^\times$ ist auch $c_1...c_r\in A^\times$.\\
		Mit \ref{701rem} folgt die Aussage.
	\end{enumerate}
\end{proof}

\begin{kor}
	Sei $A$ ein faktorieller Ring. Dann ist $A[X_1,...,X_n]$ faktoriell. \\
	Insbesondere folgt dies wenn $A$ Körper.
\end{kor}

\subsection{Irreduziblitätskriterien}
Sei $K$ Körper, $f\in K[X]$, $f\neq 0$.
\begin{enumerate}
	\setcounter{enumi}{-1}
	\item Sei $\deg(f)=0$, dann $f$ nicht irreduzibel in $K[X]$, da $f\in K[X]^\times=K^\times$.
	\item Sei $\deg(f)=1$, dann ist $f$ immer irreduzibel in $K[X]$.
	\item Sei $\deg(f)=2$ oder $\deg(f)=3$, dann ist $f$ genau dann reduzibel, wenn $f$ eine Nullstelle hat.
	\item Sei $\deg(f)>1$ und $f$ habe eine Nullstelle, dann ist $f$ reduzibel
\end{enumerate}

\begin{satz}[Reduziblitätskriterium]
	Sei $A$ ein faktorieller Ring, $K=\Quot(A)$, $f=a_nX^n+...+a_1X+a_0\in A[X]$, zu $p\in A$ Primelement mit $p$ teilt nicht $a_n$.\\
	Sei $\ol f\in A/p[X]$ das Bild von $f$.\\
	Dann folgt aus $\ol f$ irreduzibel in  $A/p[X]$, dass $f$ in $K[X]$ irreduzibel ist.
\end{satz}
\begin{proof}
	Betrachte zuerst $f$ primitiv:\\
	Sei $f\in K[X]$ reduzibel, dann folgt mit \ref{704Gauss}, dass $f$ in $A[X]$ reduzibel ist.\\
	Also gibt es $g,h\in A[X]$, mit $\deg(g),\deg(g)\geq 1$, sodass $f=gh$.\\
	Da der Führende Koeffizient von $f$ nach Voraussetzung nicht durch $p$ teilbar ist, sind auch die Führenden Koeffizienten von  $g,h$ nicht durch $p$ teilbar.\\
	Da $\deg(\ol g)=\deg(g)\geq 1$ und $\deg(\ol h)=\deg(h)\geq 1$ folgt, dass $\ol f=\ol g\ol h$ reduzibel ist.\\
	\\
	Allgemeiner Fall: Schriebe $f=c(f)\tilde f$ mit $c(f)\in A\setminus\{0\}$ und $\tilde f$ primitiv.\\
	$f$ ist genau dann in $K[X]$ reduzibel, wenn $\tilde f$ in $K[X]$ reduzibel ist.\\
	Im gezeigten Spezialfall folgt aus $\tilde{ f}$ ist reduzibel in $A/p[X]$, dass $\ol f=\ol{c(f)}\ol{\tilde{f}}$ reduzibel ist.
\end{proof}

\begin{exm*}
	\begin{enumerate}
		\item Sei $f=3X^4+2X^2+7X^2+X-5\in \Z[X]$. Dann gilt $\mod 2$:
		\[f=X^4+X^3+X^2+X+1\in\mathbb{F}_2[X]\]
		Betrachte nun die Reduziblen Polynome mit $\deg=2$:
		$\{X^2+X+1,X^2+1,X^2\}$, wobei deren Quadrate keien Teiler von $f$ sind.\\
		Also ist $f$ irreduzibel.
		\item Sei $f=X+Y^2+YX-2Y+3\in\Q[X,Y]$ ist gleich $XY^2+(X-2)Y+3\in(\Q[X])[X]$ modulo $X-2$ gilt:\\
		$2Y^2+3\in Q[Y]=\Q[X,Y]/(X-2)$ ist irreduzibel, also ist $f$ irreduzibel.
	\end{enumerate}
\end{exm*}


\begin{satz}[Eisensteinkriterium]\label{707Eisenstein}
	Sei $A$ faktoriell, $f=a_nX^n+...+a_1X+a_0\in A[X]$ primitiv und es existiert ein Primelement $p\in A$, sodass
	\begin{enumerate}
		\item $p$ teilt nicht $a_n$
		\item $p$ teil $a_i$ für alle $i=0,...,n-1$
		\item $p^2$ teilt nicht $a_0$
	\end{enumerate}
	Dann ist $f$ irreduzibel in $\Quot(A)[X]$.
\end{satz}
\begin{proof}
	Sei $f$ reduzibel in $A[X]$,$f=gh$ für $g,h\in A[X]$ mit $\deg(g),\deg(f)\geq 1$ (und $<n$).\\
	Modulo $p$ gilt: $\ol a_nX^n=\ol f=\ol g\ol h\in A/p[X]$ und $a_n\neq 0$.\\
	Da die irreduzible Zerlegung Eindeutig in $\Quot(A/p)[X]$ ist:\\
	$\ol g=uX^m$, $\ol h=vX^r$, mit $u,v\neq 0$ und $m,r>0$.\\
	Dann sind die Absoluten Koeffizienten von $g,h$ duch $p$ Teilbar, was einen Widerspruch zu 3) darstellt.
\end{proof}

\begin{exm}
	Sei $A$ faktorielle $p\in A$ prim, $n\geq 1$.\\
	Dann ist $X^n-p$ irreduzibel.
\end{exm}

%VL 12.12.2016
\section{Algebraische Körpererweiterungen}
\subsection{Körpererweiterungen}

\begin{definition}
	Eine $K$-Algebra $\iota:K\hookleftarrow L$ heißst \textbf{Körpererweiterung}, falls $L$ Körper ist. (Also $K\to L$ injektiv).\\
	Eine \textbf{Teilerweiterung} ist ein Unterkörper $M$ von $L$, sodass $\iota(K)\subset M$.
\end{definition}

%TODO
\todomark{die VL vom 12.12.2016}


%VL 14.12.2016
\setcounter{subsection}{4}
\setcounter{thm}{16}


\begin{definition}
	Sei $A$ eine $K$-Algebra, $a\in A$ algebraisch. Betrachte den $K$-Algebra Homomorphismus $\varphi:K[X]\rightarrow A$, $f\mapsto f(a)$.
	Dann ist $\mu_{a,K}\in K[X]$ das \textbf{Minimalpolynom von $a$ über $K$}, wenn $\Ker(\varphi)=(\mu_{a,K})$.
\end{definition}
\begin{bem*}
	Sei $A$ eine $K$-Algebra, $a\in A$. Betrachte den $K$-Algebra Homomorphismus $\varphi:K[X]\rightarrow A$, $f\mapsto f(a)$. Dann ist
	\[\Img\varphi=\{f(a)\in A\mid f\in K[X]\}=K[a]\]
	und es sind äquivalent:
	\begin{enumerate}
		\item $a$ ist algebraisch
		\item $\varphi$ ist nicht injektiv
		\item $\Ker(\varphi)=(\mu_{a,K})$ für ein eindeutiges, normiertes Polynom $\mu_{a,K}\in K[X]$.
		\item  $[K[a]:K]<\infty$.\\
		In diesem Fall gilt $[K[a]:K]=\deg(\mu_{a,K})$
	\end{enumerate}
\end{bem*}
\begin{proof}
	\begin{itemize}
		\item 1)$\Leftrightarrow$2)$\Leftrightarrow$3) ist klar.
		\item 3)$\Rightarrow$4): Es gilt, 3) ist äquivalent dazu, dass $K[a]=K[X]/(\mu_{a,K})$ für normierte Polynome $\mu_{a,K}$.\\
		Es folgt, dass $K[a]$ eine endliche $K$-Algebra ist mit $[K[a]:K]=\deg(\mu_{a,K})$.
		\item 4)$\Rightarrow$2): gilt, da sonst $K[a]\isomorph K[X]$.
	\end{itemize}
	
\end{proof}

\subsubsection{Bestimmung von $\mu_{a,K}$ I} Sei $A$ eine $K$-Algebra, $a\in A$ algebraisch. Sei $f\in K[X]$ mit $f(a)=0$, dann ist $\mu_{a,K}$ ein Teiler von $f$. Also gilt für $f\in K[X]$:\\
$\mu_{a,K}$ ist genau dann gleich $f$, wenn $f$ normiert $f(a)=0$ und $\deg(f)\leq [K[a]:K]$.

\begin{exm*}
	Sei $A=K\times K$, (mit $x\mapsto (x,x)$), sei $a=(1,0)$. Dann ist $\mu_{a,K}=X^2-X=X(X-1)$.
\end{exm*}

\begin{prop}\label{819Prop}
	Sei $K\hookrightarrow K$ eine Körpererweiterung, $a\in L$.\\
	Dann ist $a$ genau dann algebraisch über $K$, wenn $K[a]=K(a)$ ($\Leftrightarrow K[a]$ Körper).
\end{prop}

\subsection*{Bestimmung von $\mu_{a,K}$ II} Für $f\in K[X]$:\\
$f=\mu_{a,K}$ genau dann wenn $f$ normiert, $f(a)=0$ und $f$ irreduzibel ist.
\begin{proof}
	``$\Rightarrow$``: Sei $a$ algebraisch, dann ist $K[a]\subseteq L$ nullteilerfrei und ganz über $K$.\\
	Dann folgt mit \ref{813}, dass $K[a]$ ein Körper ist, sodass $K(a)=K[a]$.\\
	Ferner gilt $K[a]=K[X]/(\mu_{a,K})$ ist genau dann Körper wenn $\mu_{a,K}$ eine maximales Ideal, 
	was äquivalent dazu ist, dass $\mu_{a,K}$ irreduzibel ist.
	``$\Leftarrow$``: Sei $a$ transzendent, dann folgt mit \ref{817ii}, dass $K[X]\isomfunc K[a]$, dann ist $K[a]$ kein Körper.
\end{proof}


\begin{exm}
	Sei $K=\Q$.\begin{enumerate}
		\item Sei $a=\sqrt 2\in \R$, dann ist $\mu_{a,\Q}=X^2-2$ (da $X^2-2$ irreduzibel, normiert und $(\sqrt{ 2})^2-2=0$ ist.)\\
		Allgemein: Sei $p$ Primzahl, $a=\sqrt[n]{p}\in\C$. Dann ist $\mu_{a,\Q}=X^n-p$ (da $X^n-p$ mit \ref{707Eisenstein} irreduzibel ist.)
		\item Sei $a=\sqrt[4]{2}$, dann ist $\mu_{a,\Q[\sqrt{2}]}=X^2-\sqrt{2}\in \Q[\sqrt{2}][X]$.
		\item Sei $p$ Primzahl, $\zeta\in\C$, $\zeta\neq 1$ mit $\zeta^p=1$.\\
		(Dann $\zeta=e^{\frac{2\pi i k}{p}}$ für $k=1,...,p-1$)
		Sei $f=X^p-1$, dann $f(\zeta)=0$ und
		\[f=(X-1)(X^{p-1}+...+X+1)\]
		ist irreduzible Zerlegung.\\
		Da $\zeta\neq 1$, gilt $\mu_{a,K}=X^{p-1}+...+X+1$.\\
		Also $[\Q[\zeta]:\Q]=p-1$.
	\end{enumerate}
\end{exm}


\subsection{Algebraische Erweiterungen}
\begin{definition}
	Eine $K$-Algebrau $A$ heißt \textbf{algebraisch über $K$}, falls $A$ eine ganze $K$-Algebra ist. (d.h. jedes $a\in A$ ist algebraisch über $K$).
\end{definition}
\begin{prop}
	Sei $A$ eine $K$-Algebra. Dann sind äquivalent:
	\begin{enumerate}
		\item $[A:K]<\inf$ (d.h. $A$ ist endliche $K$-Algebra)
		\item $A$ ist algebraisch und endlich erzeugt $K$-Algebra.
		\item Es gibt algebraische Elemente $a_1,...,a_n\in A$, sodass $A=K[a_1,...,a_n]$
	\end{enumerate}
\end{prop}
\begin{proof}
	Siehe \ref{604Satz}
\end{proof}

 \begin{prop}
 	Sei $K\hookrightarrow L$ eine Köerpererweiterung und $L\hookrightarrow A$ ist $L$-Algebra, dann gilt:\\
 	$A$ ist algebraisch über $K$ genau dann, wenn $L$ algebraische Erweiterung von $K$ und $A$ algebraisch über $L$.
 \end{prop}
\begin{proof}
	Siehe \ref{606Satz}
\end{proof}

\subsection{Algebraischer Abschluss}

\begin{definition}
	Ein Körper $K$ heißt \textbf{algebraisch abgeschlossen}, falls die folgenden äquivalenten Bedingungen erfüllt sind:
	\begin{enumerate}
		\item Jedes Polynom $f\in K[X]$ mit $\deg(f)\geq 1$ besitzt eine Nullstelle in $K$.
		\item Jedes Polynom $f\in K[X]$ mit $\deg(g)\geq 1$ ist Produkt von Polynomen vom Grad 1.
		\item Jedes irreduzible Polynom in $K[X]$ hat Grad 1.
		\item Jede algebraische Körpererweiterung von $K$ hat Grad 1.
	\end{enumerate}
\end{definition}
\begin{proof}
	\begin{itemize}
		\item 1)$\Leftrightarrow$2)$\Leftrightarrow$3).
		\item 3)$\Rightarrow$4): Sei $K\hookrightarrow L$ algebraische Körpererweiterung, $a\in L$.\\
		Dann folgt aus $3)$, dass $\mu_{a,K}$ Grad 1 hat, also $\mu_{a,K}=X-a\in K[X]$. Also $a\in K$.
		\item 4)$\Rightarrow$3): Sei $f\in K[x]$ irreduzibel.\\
		Dann ist $K[X]/(f)$ eine endliche Körpererweiterung mit $[K[X]/(f):f]=\deg(f)$.\\
		Es folgt mit 4), dass $\deg(f)=1$.
	\end{itemize}
\end{proof}

\begin{exm}
	$\C$ ist Algebraisch abgeschlossen.
\end{exm}
\begin{definition}
	Sei $K$ Körper. Eine Algebraische Erweiterung $K\hookrightarrow \ol K$ heißt \textbf{algebraischer Abschluss von $K$}, wenn $\ol K$ abgeschlossen ist.
\end{definition}
\begin{exm*}
	\begin{enumerate}
		\item $\R\hookrightarrow \C$ ist algebraischer Abschluss.
		\item $\Q\hookrightarrow \C$ ist kein algebraischer Abschluss.
	\end{enumerate}
\end{exm*}

\begin{theorem}
	Sei $K$ Körper.\\
	Dann existiert ein algebraischer Abschluss von $K$.
\end{theorem}

%VL 19.12.2016

\subsection{Fortsetzung von Körperhomomorphismen}
\begin{bem}
	Seien $K\hookrightarrow A_1$, $K\hookrightarrow A_2$ $K$-Algebren und sei
	\[\Hom_{K-\text{Alg}}(A_1A_2)=\{\varphi:A_1\rightarrow A_2|\varphi\text{ist $K-$Algebra-Homomorphismus}\}\]
	Jedes $\varphi\in\Hom_{K-\text{Algebra}}$ ist $K$-linear.\\
	Falls $A-1=L$ ein Körper, $A_2\neq 0$, dann ist $\varphi$ injektiv und es gilt
	\begin{enumerate}
		\item $[L:K]\leq[A_2:K]$
		\item Falls $[L:K]=[A_2:K]\leq \infty$, dann ist $\varphi$ ein Homomorphismus von $K$-Algebren.
	\end{enumerate}
\end{bem}

\begin{satz}
	Sei $K\hookrightarrow L$ und $K\hookrightarrow L'$ Körpererweiterungen. Sei $a\in L$ algebraisch über $K$.
	\begin{enumerate}
		\item Sei $\varphi:K[a]\to L'$ ein $K-$Algebra-Homomorphismus.\\
		Dann ist $\varphi(a)\in L'$ algebraisch und $\mu_{\varphi(a),K}=\mu_{a,K}$.
		\item Es gibt die Bijektion
		\begin{align*}
		Inhalt\Hom_{K-\text{Algebra}}(K[a],L')&\rightarrow \{a'\in L'|\mu_{a,K}=0\}\\
		\varphi&\mapsto\varphi(a)
		\end{align*}
		Insbesondere gilt 
		\[\deg(\mu_{a,K})=[K[a]:K]\geq \#\Hom_{K-\text{Algebra}}(K[a],L')\]
		mit Gleichheit genau dann wenn $\mu_{1,K}$in $L'$ vollständig in Linearfaktoren zerfällt und alle Nullstellen von $\mu_{a,K}$ in $L'$ paarweise verschieden sind.
	\end{enumerate}
\end{satz}
\begin{proof}
	Sei $\varphi:K[a]\rightarrow L'$ ein K-Algebra-Homomorphismus.\\
	Dann ist $\mu_{a,K}=0$, denn:\\
	Sei $\mu_{a,K}=X^+\la_{n-1}X^{n-1}+...+\la_0\in K[X]$.
	\begin{align*}
	\mu_{a,K}(\varphi(a))&=\varphi(a)^n+\la_{n-1}\varphi(a)^{n-1}+...+\la_0\\
	&=\varphi(a^n)+\varphi\left(\la_{n-1}a^{n-1}\right)+...+\la_0\\
	&=\varphi\left(a^n+\la_{n-1}+...+\la_0\right)\\
	&=\varphi(0)=0
	\end{align*}
	Also ist $\varphi(a)$ algebraisch und $\mu_{\varphi(a),K}$ teilt $\mu_{a,K}$.\\
	Da $\mu_{a,K}$ irreduzibel ist folgt, dass $\mu_{\varphi(a),K}=\mu_{a,K}$.\\
	Dies zeugt (1) und dass die Abbildung $\varphi\mapsto \varphi(a)$ in (2) wohldefiniert ist.\\
	\\
	Sei $a'\in L'$ mit $\mu_{a,K}(a)=0$, dann teilt $\mu_{a',K}$ das Polynom $\mu_{a,K}$, also $\mu_{a',K}\mu_{a,K}$.
	\[K[a]=\Ker[X]/(\mu_{a,K})=K[X]/(\mu_{a',K})=K[a']\subseteq L\]
	stellen $K$-Algebra Homomorphismen $\varphi:K[a]\rightarrow L'$ mit $\varphi(a)=a'$ dar.\\
	$\varphi$ ist eindeutig, da die K-Algebra $K[a]$ durch $a$ erzeugt wird.
\end{proof}

\begin{satz}\label{830Satz}
	Sei $K\hookrightarrow L$ eine algebraische Erweiterung und sie $L'$ eine algebraische abgeschlossene Erweiterung von $K$.\begin{enumerate}
		\item Dass existiert ein $K-$Algebra-Homomorphismus $\varphi:L\hookrightarrow L'$.
		\item Falls $L$ und $L'$ algebraisch Abschlüssen von $K$ sind, ist $\varphi$ ein Homomorphismus.
	\end{enumerate}
\end{satz}

\begin{kor}
	Sei $\ol K$ und $\ol K'$ algebraische Abschlüsse von $K$. Dann existiert ein $K$-Algebra-Homomorphismus $\ol K_1\isomfunc \ol K_2$.
\end{kor}

\begin{proof}
	Sei $\mathfrak F:=\{(Z,\tau)\mid K\hookrightarrow Z \subseteq L\text{ Teilkörper und }\tau:Z\hookrightarrow L' \text{ K-Algebra-Homomorphismen}\}$.\\
	Für $(Z,\tau).(Z',\tau')$ schreibe 
	\[(Z,\tau)\leq(Z',\tau'):\Leftrightarrow Z\subset Z',\tau=\tau'|_Z\]
	Also ist $\leq$ eine partielle Ordnungn auf $\mathfrak F$.\\
	Und da $(K,K\hookrightarrow L')\in \mathfrak F$ gilt $\mathfrak F\neq \emptyset$.\\
	Sei nun $\xi\subseteq\mathfrak F$ eine total geordnete Teilmenge, dass ist 
	\[\left(\bigcup_{(Z,\tau_{Z})\in\xi}Z,\tau\right)\]
	mit $\tau|Z=\tau$ für alle $(Z,\tau_Z)\in\xi$ eine obere Schranke in $\mathfrak F$.\\
	Mit \ref{104LemZorn} folgt, dass es ein maximales Element $(Z_0,\tau_0)\in \mathfrak F$ gibt.\\
	\\
	Behauptung: $Z_0=L$ (setze dann $\varphi:=\tau_0$)\\
	Angenommenes existert ein $a\in L\setminus Z_0$. Dann ist $a$ algebraisch über $Z_0$ und 
	\[\Hom_{Z_0}(Z_0[a],L')\overset{\leftrightarrow}{\ref{829}}\{a'\in L'\mid \mu_{a,Z_0}(a')=0\}\neq \emptyset\]
	Also existiert ... 
	%TODO
\end{proof}


\section{Normale und separable Körpererweiterungen}
\subsection{Zerfällungskörper}


\begin{definition}
	Sei $\mathfrak F\subseteq K[x]$ eine Menge nicht konstanter Polynome. Eine Körpererweiterung $K\hookrightarrow L$ heißt \textbf{Zerfällungskörper} von $\mathfrak F$, falls gilt
	\begin{enumerate}
		\item Jedes $f\in\mathfrak F$ zerfällt über $L$ vollstädnig ein Linearfaktoren
		\item Für $f\in\mathfrak F$ sei $R_f:=\{a\in L|f(a)=0\}$. Dann ist
		\[L=K\left(\bigcup_{f\in \mathfrak F}R_f\right)\]
	\end{enumerate}
\end{definition}
\begin{bem*}
	Dann ist $L=K\left[\bigcup_{f\in \mathfrak F}R_f\right]$ eine algebraische Erweiterung von $K$.
\end{bem*}

\begin{exm}
	Sei $f\in K[X],\deg(f)\geq 1$ und Sei $\ol K$ ein algebraischer Abschluss von $K$.\\
	Seien $a_1,...,a_\in \ol K$ die Nullstellen von $F$.\\
	Dann ist $K[a_1,...,a_n]\subseteq \ol K$ ein Zerfällungskörper von $f$.
\end{exm}

\begin{prop}
	Sei $\mathfrak F\subseteq K[X]$ eine Menge nicht konstanter Polynome.
	\begin{enumerate}
		\item Dann existiert ein Zerfällungskörper von $\mathfrak F$.
		\item Seien $L_1$ und $L_2$ Zerfällungskörper von $\mathfrak F$, seien $\ol L_1$ und $\ol L_2$ algebraische Abschlüsse von $L_1$ bzw $L_2$ und sei $\varphi:ol L_1\rightarrow \ol L_2$ ein $K$-Algebra-Homomorphismus.\\
		Dann ist $\varphi(L_1)=L_2$
	\end{enumerate}
\end{prop}
\begin{proof}
	\begin{enumerate}
		\item Sei $\ol K$ ein algebraischer Abschluss und sei $S:=\{a\in \ol K\mid \exists f\in \mathfrak F:f(a)=0\}$.\\
		Dann ist $K(S)$ Zerfällungskörper von $\mathfrak F$.
		\item Seien $\ol L_1$ und $\ol L_2$ bereits algebraische Abgeschlüsse von $K$.\\
		Dann folgt \ref{830Satz}, dass $\varphi$ Homomorphismus ist.\\
		Sei $S_1:=\{a\in L_1\mid\exists f\in\mathfrak F: f(a)=0 \}$.\\
		Es folgt, dass $L_1=K(S_1)$.\\
		Zeige: $\varphi(S_1)\subseteq L_2$. Sei: $f\in\mathfrak F$, $a\in L_1$ Nullstelle von $f$.\\
		Dann ist $f(\varphi(a))=\varphi(f(a))=0$. Also $\varphi(a)\in\ol L_2$, also Nullstelle von $f$ ist. Es folgt $\varphi(a)\in L_2$.\\
		Also folgt $\varphi(S_1)\subseteq L_2$, dann ist $\varphi(L_1)\subseteq L_2$.\\
		Analog für $\varphi^{-1}:$ $\varphi^{-1}(L_2)\subseteq L_1$.\\
		Zusammen folgt, dass $\varphi(L_1)=L_2$.
	\end{enumerate}
\end{proof}

\begin{kor}
	Sei $\mathfrak F\subseteq K[X]$ eine Menge nicht konstatnert Polyome, sei $\Omega$ Körpererweiterung von $K$ und seien $L_1,L_2\subseteq\Omega$ Zerfällngskörper von $\mathfrak F$. Dann ist $L_1=L_2$.
\end{kor}
\begin{proof}
	Übergang zu einem algebraischen Abschluss  von $\Omega$:\\
	Sei OE $\Omega$ ein algebraischer Abgeschlossen.\\
	Dann folgt aus $L_1,L_2$ ist algebraisch über $K$, dass $L_1,L_2\subseteq \{q\in R\mid \text{$a$ algebraisch über $K$}\}$.\\
	Also ist OE $\Omega$ algebraischer Abschluss von $K$.\\
	Dann ist $\Omega$ ein algerischer Abschluss von $L_1$ und von $L_2$.\\
	Wende nun \ref{9032} an auf $\ol L_1=\ol L_2\Omega$ und $\varphi=\id_{\Omega}$
\end{proof}

\begin{exm}
	Sei $p\in\N$ Primzahl, sei $f=X^3-p$. (Es folgt $f$ ist irreduzibel über $K=\Q$) und sei $\al=\sqrt[3]{p}\in\R_{>0}$.\\
	Sei $\zeta:=e^{\frac{2\pi i}{3}}$. Dann sind $\al,\zeta\al,\zeta^2\al\in\C$ die Nullstellen von $f$.\\
	Der Zerfällungskörper von $f$ ist
	\[\Q[\al,\zeta\al\zeta^2\al]=\Q[\al,\zeta]\]
	\[\begin{tikzcd}[column sep=small]
		&\Q[\al,\xi]&\\
		\R\supseteq \Q[\al] \ar[ur,hook,"2"] && \Q[\xi] \ar[ul,hook,"3"]\\
		& \Q \ar[ul,hook,"3"] \ar[uu,hook,"6"] \ar[ur,hook,"2"] &
	\end{tikzcd}\] 
\end{exm}

%VL 21.12.2016
\subsection{Normale Erweiterungen}
\begin{definition}\label{906}
	Eine algebraische Körpererweiterung $K\hookrightarrow L$ heißt \textbf{normal}, falls eine der folgenden äquivalenten Bedingungen erfüllt ist\\
	\begin{enumerate}
		\item Es existiert eine Menge $\mathfrak F\subseteq K[X]$ mit konstanten Polynomen, sodass $L$ der Zerfällungskörper von $\mathfrak F$ in $A$ ist.
		\item Sei $f\in K[X]$ irreduzibel mit Nullstelle in $L$, dann zerfällt $f$ in $L[X]$ vollständig in Linearfaktoren.
		\item Für jede Körpererweiterung $L'$ von $L$ und für jeden $K$-Algebra-Homomorphismus $\varphi:L\hookrightarrow L'$ gilt $\varphi(L)=L$.
		\item Für jeden algebraischen Abschluss $\Omega$ von $L$ und für jeden $K$-Algebra-Automorphismus $\varphi:\Omega\rightarrow \Omega$ gilt $\varphi(L)=L$.
	\end{enumerate}
\end{definition}
\begin{proof}
	\begin{itemize}
		\item 1)$\Rightarrow$2): Sei $L$ Zerfällungskörper von $\mathfrak F$, dann folgt $\varphi(L)$ ist zerfällungskörper von $\mathfrak F$. Dann folgt mit \ref{904}, dass $\varphi(L)=L$.
		\item 3)$\Rightarrow$4): Sei $\varphi:\Omega\isomfunc\Omega$ ein $K$-Algebra-Automorphismus.\\
		Wende 3) auf $\varphi|_L:L\rightarrow \Omega$ an.
		\item Sei OE $L'$ algebraisch abgeschlossen. Ersetze $L'$ durch
		\[L'_{\text{alg}}:=\{a\in L'|\text{$a$ ist algebraisch über $K$}\}\]
		Da $K\subseteq L$ algebraisch ist, folgt, dass $\varphi(L)\subseteq L'_\text{alg}$. Also ist OE $L'$ algebraischer Abschluss von $L$.\\
		Aus \ref{830Satz} folgt die Exitsnez einer Fortsetzung $\varphi':L'\rightarrow L'$ zu $\varphi$ und $\varphi'$ ist Automorphismus.\\
		Also $\varphi(L)=\varphi'(L)=L$.\\
		\item 3)$\Rightarrow$2): Sei $f\in K[X]$ irreduzible, $a\in L$ mit $f(a)=0$.\\
		Sei $L'$ ein algebraischer Abschluss von $L$, $b\in L'$ mit $f(b)=0$.\\
		Zu Zeigen: auch $b\in L$.\\
		Sei OE $f$ normiert. Dann $f=\mu_{a,K}$. Also exitsiert ein eindeutiger $K$-Algebra-Homomorphismus $\ol \varphi:K[a]\rightarrow L'$ mit $\ol{\varphi}(a)=b$.\\
		Setze nun $\ol \varphi$ fort mit $\varphi:L\rightarrow L'$ (existenz durch \ref{830Satz}).
		Dann folgt durch 3), dass $\varphi(L)=L$, also $\varphi(a)=b\in L$.
		\item Sei $S\subseteq L$ Teilmenge und $L=K(S)$. Sei $\mathfrak F:=\{\mu_{a,K}\mid a\in S\}$.\\
		Aus $2)$ folgt, dass $\mu_{a,K}$ über $L$ für alle $a\in S$ in Linearfaktoren zerfällt.\\
		Sei $S':=\{b\in L\mid\exists f\in\mathfrak F:f(b)=0 \}\supseteq S$.\\
		Dann ist $K(s)=L$, $K(s)=\subseteq K(s')\subseteq L$.\\
		Also $L=K(s')$, d.h. $L$ ist Zerfällungskörper von $\ol f$.
	\end{itemize}
\end{proof}

\begin{exm*}
	Sei $L=K[a]$ normal, dann ist $L$ Zerfällungskörper von $\mu_{a,K}$.
\end{exm*}

\begin{prop}
	Sei $K\hookrightarrow L$ eine normale Körpererweiterung. Sei $M\subseteq L$ Teilkörpererweiterung.
	\begin{enumerate}
		\item Jeder $K$-Algebra-Homomorphismus $\varphi:M\hookrightarrow L$ kann ein einem $K$-Algebra-Automorphismus $\ol\varphi:L\isomfunc L$ fortgesetzt werden.
		\item $K\hookrightarrow M$ ist genau dann normal, wen für jeden $K$-Automorphims $\sigma:L\isomfunc L$ gilt $\sigma(M)=M$.
	\end{enumerate}
\end{prop}
\begin{proof}
	\begin{enumerate}
		\item Betrachte $\varphi':M\hookrightarrow L\hookrightarrow L'$ und $L'$ ist algebraischer Abschluss von $L$.\\
		Dann gibt \ref{830Satz} die Existenz einer Fortsetzung $\ol\varphi':L'\isomfunc L'$, die $K$-Algebra-Automorphismus ist.\\
		Dann folgt mit \ref{906}.3, dass $\ol \varphi'=L$, sodass $\ol \varphi=\ol{\varphi'|_L}$ ein $K$-Algebra-Automorphismus von $L$ ist.
		
		\item ``$\Rightarrow$´´ ist durch \ref{906}.3 gegeben.
		``$\Rightarrow$´´ Sei $L'$ algebraischer Abgeschluss von $L$, $\ol \sigma:L'\isomfunc L'$ Fortsetzung von $\sigma$ und jeder Automorphismus von $L$ ist Einschränkung eines Automorphismus von $L'$.\\
		Also gilt $\ol \sigma(M)=M$ für alle $\ol \sigma\operatorname{Aut}_{\text{$K$-Algebra}}(L')$.\\
		Dann folgt mit \ref{906}.3, dass $K\hookrightarrow M$ normal ist.
		\end{enumerate}
\end{proof}

\begin{exm}
	\begin{enumerate}
		\item Sei $\varphi:K\hookrightarrow L$ Körpererweiterung mit $[L:K]=2$. Dann ist $\varphi$ normal.
		\begin{proof}
			Sei $f\in K[X]$ irreduzible, $a\in L$ mit $f(a)=0$. Dann ist $f=\mu_{a,K}$, also $\deg(\mu_{a,K})=[K[a]:K]\leq 2$.
			Wenn $\deg(\mu_{a,K})=1$, dann $\mu_{a,K}=X-a$ mit $a\in K$.\\
			Wenn $\deg(\mu_{a,K})=2$ genau dann gilt $a\in L\setminus K$. Dann ist $\mu_{a,K}=(X-a)g$ mit $g\in L[X]$ vom Grad $1$, also $g=X-b\in L[X]$.\\
			Also sind ie Nullstellen von $\mu_{a,K}$ beide in $L$.\\
			Dann folgt mit \ref{906}.3, dass $K\hookrightarrow L$ normal ist.
		\end{proof}
		\item Sei $K\hookrightarrow \ol K$ ein algebraischer Abschluss. Dann ist $K\hookrightarrow \ol K$ eine normale Erweiterung.\\
		(z.B. ist $\ol K$ Zerfällungskörper von $\{f\in K[x]\mid\text{$f$ nicht konstant}\}$).
		\item $\Q\subset \Q[\sqrt[3]{7}]$ ist nicht normal.\\
		Denn $X^3-7$ hat Nullstelle in $\Q[\sqrt[3]{7}]$, aber nicht jede Nullstelle von $X^3-7$ liegt in $\Q[\sqrt[3]{7}]$:
		\[\Q\subset \Q[\sqrt[3]{7}]\subset \Q[\sqrt[3]{7},\zeta]\]
		für $\zeta=e^{\frac{2\pi i}{3}}$.
	\end{enumerate}
\end{exm}

\begin{bem}
	Seien $K\hookrightarrow L\hookrightarrow M$ Körpererweiterungen.
	\begin{enumerate}
		\item  Wenn $K\hookrightarrow M$ normal ist, dann ist $L\hookrightarrow M$ normal.\\
		($M$ ist Zerfälllungskörper von $\mathfrak F\subseteq K[X]\subseteq L[X]$).
		\item Aus $K\hookrightarrow M$ normal folgt i.A. \textbf{nicht}, dass $K\hookrightarrow L$ normal ist mit \ref{908}.3.
		\item Aus $K\hookrightarrow L$, $L\hookrightarrow M$ normal folgt i.A. \textbf{nicht}, dass $K\hookrightarrow M$ normal.
	\end{enumerate}
\end{bem}



\subsection{Separabilitätsgrad}

\begin{prop}
	Sei $A$ ein Ring, sei $E\neq 0$ ein freier $A$-Modul.\\
	Dann ist die Sequenz $0\to M'\to M''\to 0$ von $A$-Moduln genau dann exakt, wenn
	\[0\to E\otimes_A M'\to E\otimes_A M\to E\otimes_A M''\to 0\]
	exakt ist.\\
	(Insbesondere $E\otimes_A M=0\Leftrightarrow M=0$)
\end{prop}
\begin{proof}
	$E$ ist genau dann frei, wenn $E\isomorph A^{(I)}$ mit $I\neq \emptyset$.\\
	Man erhält insbesondere die Isomorphismen
	\[
	\begin{tikzcd}
	0 \ar[r] 
		& E\otimes_A M' \ar[r,"\operatorname{id}_E\otimes u"] \ar[d, leftrightarrow, "\sim"]
			& E\otimes_A M \ar[r,"\operatorname{id}_E\otimes v"] \ar[d, leftrightarrow, "\sim"]
				& E\otimes_A M''\ar[r] \ar[d, leftrightarrow, "\sim"]
					&0\\
	0 \ar[r] 
		& (M')^{(I)} \ar[r,"(m_i')_{i\in I}\mapsto (u(m_i'))_{i\in I}"] 
			& M^{(I)} \ar[r] 
				& (M'')^{(I)} \ar[r] 
					&0 
	\end{tikzcd}
	\]
	Es folgt die Behauptung.
\end{proof}

\begin{bem}
	Sei $A$ eine endliche $K$-Algebra.
	Dann folgt mit \ref{813}, dass $A=\prod_{i=1}^{r}A/m_i e_i$, mit $m_1,...,m_r\subset A$ maxmimale Ideale.\\
	Sei $B$ eine nullteilerfreie $K$-Algebra, sie $\varphi:A\to B$ein $K$-Algebra-Homomorphimsmus.\\
	Dann ist $\varphi(A)\subseteq B$ nullteilerfrei, oder $\Ker(\varphi)=m_i$ für ein $I\in\{1,...,r\}$.\\
	Also faktorisiert $\varphi$ in $A\to A/m_i\hookrightarrow B$- Insebsondere:
	\[\Hom_{K-\text{Algebra}}(A,B)=\dot{\bigcup}_{i=1}^r\Hom_{k-\text{Algebra}}(A/m_i,B)\label{911.1}\]
\end{bem}

\begin{bem}
	Sei $K\hookrightarrow A$ eine $K$-Algebra, $K\hookrightarrow K$ eine Körpererweiterung, $L\hookrightarrow B$ ein $L$-Algebra. Dann hat man zueinander inverse Bijektionen:
	\begin{align*}
	\Hom_{K\text{-Algebra}}(A,B)&&\overset{1:1}{\leftrightarrow}&&&\Hom_{L\text{-Algebra}}(L\otimes_K A,B)\\
	\varphi&&\mapsto&&&(l\otimes a\mapsto l\varphi(a))\\
	(a\mapsto \varphi(1\otimes a))&&\mapsfrom&&&\varphi
	\end{align*}
\end{bem}

%TODO


%VL 09.01.2017
%COUNTER
\setcounter{thm}{14}
\begin{bem}\label{915bem}
	Sei $A$ algebraische $K$-Algebra, $K\hookrightarrow L$ Körpererweiterung. Dann\[[A\otimes_K L:L]_S=[A:K]_S\]
\end{bem}
\begin{proof}
	Sei $\Omega$ eine algebraisch abgeschlossene Erweiterung von $L$. Dann gibt es die Bijektion
	\begin{align*}
	\Hom_{K-\text{Algebra}}&\overleftrightarrow{1:1}\Hom_L(A\otimes_K L,\Omega)\\
	\sigma&\mapsto (a\otimes l\mapsto l\sigma(a))\\
	a\mapsto\tau(a\otimes 1)&\mapsfrom\tau
	\end{align*}
\end{proof}

\begin{lem}\label{916lem}
	Sei $A$ eine endliche $K$-Algebra. Dann ist $(A:K)_S$ die Anzahl der maximalen Ideale von $A\otimes_K\Omega$. ($\Omega$ als algebraisch abgeschlossene Erweiterung von $K$)
\end{lem}
\begin{proof}
	Mit \ref{915bem} folgt, dass OE $\Omega=K$.\\
	Seien $m_1,...,m_r\subset A$ die maximalen Ideal. Dann ist $A/m_i$ eine endliche Körpererweiterung von $\Omega$, also $A/m_I=\Omega$.\\
	Dann folgt mit \ref{911.1}, dass \[\#\Hom_{\Omega-\text{Algebra}}(A,\Omega)=\#\bigcup_{i=1}^r\Hom_{\Omega-\text{Algebra}}(\underbrace{A/m_i}_{=\Omega},\Omega)=r\]
\end{proof}

\begin{prop}\label{917Prop}
	Sei $a$ endliche $K$-Algebra. Dann gilt
	\[[A:K]_S\leq[A:K](=\dim_K(A))\]
\end{prop}
\begin{proof}
	Sei OE $K=\Omega$ algebraisch abgeschlossen. Sei $A=\prod_{i=1}^rA/m_ie_i$. Also
	\begin{align*}
	[A:K]_S&\overset{\ref{916lem}}{=}r=\sum_{i=1}^{r}\dim_K(\underbrace{A/m_i}_{=K})\\
	&\leq\sum_{i=1}^{r}\dim_K(A/m_ie_i)=[A:K]
	\end{align*}
\end{proof}

\begin{bem}
	Der Beweis von \ref{917Prop} zeigt $[A:K]_S=[A:K]$ $\Leftrightarrow$ $A\otimes_K\Omega$ ist reduziert $\Leftrightarrow$ $A\otimes_K \Omega\isomorph \Omega\times...\times\Omega$ ($r=[A:K]$ mal).
\end{bem}

\begin{prop}\label{918prop}
	Sei $K\hookrightarrow L$ algebraische Körpererweiterung, $A$ ganze $L$-Algebra.\\
	Dann ist
	\[[A:K]_S=[A:L]_S\cdot[L:K]_S\]
\end{prop}
\begin{proof}
	Sie $\Omega$ ein algebraischer Abschluss in $L$. Betrachte
	\[\rho:\Hom_{K-\text{Algebra}}(A;\Omega)\rightarrow \Hom_{K-\text{Alg}}(L,\omega),\quad\sigma\mapsto\sigma|_L\]
	$\rho$ ist surjektiv (\ref{830Satz}). Sei $\varphi:L\hookrightarrow \Omega$ ein K-Algebra-Homomorphismus. Dann ist
	\begin{align*}
	\rho^{-1}(\{\varphi\})&=\{\sigma\in\Hom_{K-\text{Algebra}}(A,\Omega)\mid \sigma|_L=\varphi\}\\
	&=\Hom_{L-\text{Algebra}}(A,\Omega)
	\end{align*}
	wobei $\Omega$ von $\varphi$ als $L$-Algebra aufgefasst wird.\\
	Also
	\begin{align*}
	[A:K]_S&=\#\Hom_{K-\text{Alg}}(A,\Omega)\\
	&=\sum_{\varphi\in\Hom_{K-\text{Alg}}(L,\Omega)}\#\varrho^{-1}(\{\varphi\})\\
	&=\sum_{\varphi}[A:L]_S\\
	&=[L_K]_S[A_L]_S
	\end{align*}
\end{proof}


\subsection{Separable Polynome}
\begin{definition}
	Sei $f=a_nX^n+...+a_1X+a_0\in A[X]$.\\
	Definiere \[f':=na_nX^{n-1}+(n-1)a_{n-1}X^{n-2}+...+2a_2X+a_1\]
	$f'$ heißt die (formale) \textbf{Ableitung} von $f$.
\end{definition}
\begin{bem*}
	Seien $f,g\in A[X]$ und $a,b\in A$.
	\begin{itemize}
		\item Die Ableitung ist Linear: $(af+bg)'=af'+bg'$
		\item Es gilt die Leibnitz-Regel $(fg)'=fg'+f'g$
	\end{itemize}
\end{bem*}
\begin{proof}
	\begin{itemize}
		\item Linearität. klar.
		\item Aus der Linearität können wir OE annehmen, dass $f=X^i,g=X^j$.\\
		Dann \[(fg)'=(X^{i+j})'=(i+j)X^{i+j-1}=iX^{i-1}X^j+jX^iX^{j-1}=fg'+f'g\]
	\end{itemize}
\end{proof}

\begin{exm*}
	Sei $\dim(K)=p>0$. Dann folgt aus $f=X^p+1$, dass $f'=pX^{p-1}=0$.
\end{exm*}

\begin{definition}
	Sei $f\in K[X]$, $a\in K$. Dann ist
	\[\ord_a(f):=\sup\{n\geq 0\mid\text{$(X-a)^n$ teilt $f$}\}\]
	die \textbf{Ordnung der Nullstelle} $a$ von $f$.
\end{definition}
%COUNTER
\addtocounter{thm}{-1}
\begin{bem}\label{921bem}
	\begin{itemize}
		\item $\ord_a(f)=\infty\Leftrightarrow f=0$.
		\item $\ord_a(f)=0\Leftrightarrow f(a)\neq 0$.
		\item $\ord_a(f)=1\Leftrightarrow f(a)=0$ und $f'(a)\neq 0$.
		\begin{proof}
			$\ord_a(f)=1$ genau dann wenn $f=(X-a)g$ mit $g(a)\neq 0$.\\
			$\Leftrightarrow f(a)=0$ und $f'(a)=g(a)+g'(a)(a-a)=g(0)\neq 0$.
		\end{proof}
	\end{itemize}
\end{bem}


\begin{definition}
	Ein Polynom $f\in K[X]$, $f\neq 0$ heißt \textbf{separabel}, falls alle Nullstellen  in einem Zerfällungskörper paarweise verschieden sind.
\end{definition}
\begin{prop}\label{923prop}
	Sei $\Omega$ eine algebraisch abgeschlossen Erweiterung von $K$, $f\in K[X],f\neq 0$. Dann sind äquivalent:
	\begin{enumerate}
		\item $f$ ist separabel
		\item Alle Nullstellen von $f$ in $\Omega$ sind verschieden
		\item $f$ und $f'$ haben in $\Omega$ keine gemeinsame Nullstelle.
		\item $f$ und $f'$ sind in $K[X]$ teilerfremd
	\end{enumerate}
\end{prop}
\begin{proof}
	\begin{description}
		\item[(1)$\Leftrightarrow$ (2)] Sei $L$ ein Zerfällungskörper von $f$. Dann existiert (\ref{830Satz}) eine eindeutiges Körpererweiterung $L\hookrightarrow \Omega$.
		\item[(ii)$\Leftrightarrow$(iii)] aus \ref{921bem}
		\item[(iii)$\Leftrightarrow$(iv)] $f$ und $f'$ zerfallen in $\Omega[X]$ in Linearfaktoren.\\
		Also ist (iii) äquivalent daszu, dass $f$ und $f'$ sind in $\Omega[X]$ teilerfremd sind.\\
		Ist äquivalent $\Omega\otimes_KK[X]/(f,f')\isomorph\Omega[X]/(f,f')=0$.\\
		\ref{910} gibt uns dann die Äquivalenz zu $K[X]/(f,f')=0$, genau dann wenn $f,f'$ auch teilerfremd in $K[X]$ sind.
	\end{description}
\end{proof}

\begin{exm*}
	\begin{enumerate}
		\item $(X^3-2)(X-1)\in \Q[X]$ ist separabel
		\item Sei $K=\Quot(\mathbb{F}_p[T])$ und $f=X^p-T\in K[X]$ ist nach dem Eisensteinkriterium mit $p=T$ irreduzibel.\\
		Aber $f$ ist nicht separabel:\\
		Im Zerfällungskörper $K[\sqrt[p]{T}]$ gilt $f=(X-\sqrt[p]{T})^p$.\\
		Äquivalent: $f$ ist nciht teilerfremd zu $f'=pX^{p-1}=0$.
	\end{enumerate}
\end{exm*}

\begin{satz}\label{924satz}
	Sei $f\in K[X]$ irreduzibel. Dann gilt
	\begin{enumerate}
		\item $f$ ist separabel genau dann wenn $f'\neq 0$.
		\item Sei $\cha(K)=0$. Dann ist $f$ separabel.
	\end{enumerate}
\end{satz}
\begin{proof}
	\begin{enumerate}
		\item Sei $f'=0$, dann sind $f'$ und $f$ zueinander teilerfremd und somit (\ref{923prop}) $f$ separabel.
		\item Sei $\cha(K)=0$, dann $\deg(f')=\deg(f)-1$, also $\deg(f'\geq 0)$.\\
		Also ist $f\neq 0$, sodass (1) $f$ separabel ist.
	\end{enumerate}
\end{proof}



\subsection{Separable Algebren}

\begin{definition}
	Eine algebraisch $K$-Algebra $A$.\\
	Ein $a\in A$ heißt \textbf{separabel}, falls $\mu_{a,K}$ separabel ist.\\
	$A$ heißt \textbf{separabel}, falls jedes $a\in A$ separabel ist.
\end{definition}

\begin{theorem}\label{926thm}
	Sei $A$ eine endliche $K$-Algebra und sei $\Omega$ eine algebraisch abgeschlossen Erweiterung von $K$.\\
	Dann sind äquivalent:
	\begin{enumerate}
		\item $A$ ist separable $K$-Algebra
		\item $[A:K]_S=[A:K]$
		\item $A\otimes_K\Omega$ ist reduziert.
		\item $A\otimes\Omega\isomorph \Omega^r$ also $\Omega$-Algebra.
		\item Es existieren $a_1,...,a_n\in A$ separabel, sodass $A=K[a_1,...,a_n]$
		\item Es exitsiert $a\in A$ separabel, sodass $A=K[a]$.
	\end{enumerate}
\end{theorem}
\begin{proof}
	Zeige:\\
	\begin{center}
		\begin{tikzcd}
		(6)\ar[r,Rightarrow] & (4) \ar[r,Leftrightarrow,"\ref{918prop}"] & (3) \ar[r,Leftrightarrow,"\ref{918prop}"] \ar[dr,Rightarrow] & (2)\\
		& (5) \ar[u,Rightarrow] & & (1) \ar[ll,Rightarrow] \ar[lllu,Rightarrow,dashed,"\ref{931}9.31"'] %TODO set Label
	\end{tikzcd}
	\end{center}

	\begin{description}
		\item[(3)$\Rightarrow$(1)] Sei $a\in A$. (Zz. $a$ ist separabel)\\
		Dann ist $K[a]=K[X]/\mu_{a,K}\hookrightarrow A$.\\
		Dass ist $\Omega\otimes_K K[a]\hookrightarrow \Omega\otimes_K A$ injektiv.\\
		Dann ist (mit (3)) $\Omega\otimes_K K[a]=\Omega[X]/(\mu_{a,K})$ ist reduziert.\\
		Mit \ref{814} folgt, dass alle Nullstellen von $\mu_{a,K}$ in $\Omega$ verschieden sind. Also ist $\mu_{a,K}$ separabel, also auch $a$.
		\item[(1)$\Rightarrow$(5)] klar
		\item[(6)$\Rightarrow$(4)] Es gelte (6), dann ist $A=K[X]/(\mu_{a,K})$.\\
		Dann ist 
		\[A\otimes_K\Omega=\Omega[X]/(\mu_{a,K})\isomorph\prod \Omega[X]/(X-\al_i)=\prod\Omega\]
		Da $\mu_{a,K}$ in $\Omega$ in Linearfaktoren zerfällt.
		%VL 11.01.2017
		\item[(5)$\Rightarrow$(4)] Seien $a_1,...,a_n\in A$. Wir verwenden (6)$\Rightarrow$(4). Also gilt $K[a_i]\otimes_\Omega\isomorph \Omega^{^d_i}$ und $\mu_{a,K}=\prod (x-a_i)$.\\
		Dann gilt, dass
		\begin{align*}
		(K[a_1])\otimes_K...\otimes_K K[a_n])\otimes_\Omega \Omega &=(K[a_i]\otimes_K\Omega)\otimes_K...\otimes_K(K[a_n]\otimes_K\Omega)\\
		&\isomorph\omega^{d_1}\otimes_\Omega \Omega^{d_2}\otimes_\Omega...\otimes_\Omega\\
		&=\Omega^{d_1\cdot...\cdot d_n}
		\end{align*}
		Wähle nun
		\begin{align*}
		\varphi:K[a_1])\otimes_K...\otimes_K K[a_n]&\to K[a_1,...,a_n]\\
		x_1\otimes...\otimes y_n&\mapsto x_1\cdot...\cdot x_n
		\end{align*}
		Dann ist $\varphi$ surjektiver $K$-Algebra-Homomorphimsmus.\\
		Es folgt, dass $A\otimes_K\Omega$ Quotient der $\Omega$-Algebra $\Omega^{d_1\cdot...\cdot d_n}$ und damit $A\otimes_K\Omega\isomorph \Omega^m$, $m\leq d_1\cdot...\cdot d_n$
	\end{description}
\end{proof}

\begin{definition}
	Ein Körper $K$ heißt \textbf{perfekt} wenn $\cha(K)=0$ ist oder $\cha(K)=p>0$ und $x\mapsto x^p$ surjektiv ist.
\end{definition}
%COUNTER
\addtocounter{thm}{-1}

\begin{satz}\label{927satz}
	Sei $K$ perfekt. Dann ist jede endliche Körpererweiterung separabel
\end{satz}
\begin{proof}
	Sei $K\hookrightarrow L$ eine endliche Körpererweiterung, $a\in L$. Z.z. $\mu_{a,K}$ ist separabel.\\
	Wir wissen $\mu_{a,K}$ ist irreduzibel und (\ref{924satz}) falls $\cha(K)=0$ auch separabel.\\
	Sei nun $\cha(K)=p>0$. Z.z. $\mu_{a,K}\neq 0$.\\
	Sei $\mu_{a,K}=X^n+a_{n-1}X^{n-1}+...+a_0$. \\
	Angenommen $\mu_{a,K}'=nX^n+(n-1)a_{n-1}X^{n-1}+...+a_1=0$ dann muss $a_i=0$ falls $p$ nicht $i$ teilt.\\
	Dann ist $\mu_{a,K}=X^{pk}+b_kX^{p(k-1)}+...+b_0$ mit $b_j=a_{p\cdot j}$.\\
	Wähle nun $\beta_j^p=b_j$. \\
	Dann ist 
	\[\mu_{a,K}=\sum_j\beta_j^pX^{pj}=\left(\sum_j\beta_jX^j\right)^p\]
	Also ist $\mu_{a,K}$ nicht irreduzibel. Widerspruch!
\end{proof}

\begin{exm}
	Sei $K=\Quot(\mathbb{F_p}[T])$.\\
	Dann ist $K(\sqrt[p]{T})$ eine nicht separable Erweiterung von $K$.
\end{exm}

\begin{prop}\label{929prop}
	Sei $K\hookleftarrow L$ eine endliche Körpererweiterung, $L\hookleftarrow A$ endliche $L$-Algebra, $A\neq 0$. Dann gilt:\\
	$A$ ist genau dann separable $K$-Algebra, wenn $A$ separabel $L$-Algebra und $L$ separabel $K$-Algebra.
\end{prop}
\begin{proof}
	Sei $A$ separabel $K$-Algebra. Dies ist äquivalent (\ref{926thm})dazu , dass 
	\[[A:L][L:K]=[A:K]=[A:K]_S=[A:L]_S[L_K]_S\]
	$\Leftrightarrow$ $A$ ist separable $L$-Algebra und $L$ ist separable $K$-Algebra.
\end{proof}


\subsection{Satz vom primitiven Element}

\begin{satz}\label{930satz}
	Sei $G\subseteq(K^\times,\cdot)$ eine endliche Untergruppe.\\
	Dann ist $G$ zyklisch ($\Leftrightarrow G\isomorph (\Z/n\Z,+)$)
\end{satz}
\begin{proof}
	Sei $G$ endliche abelsche Gruppe.\\
	Dann $\prod_{i=1}^{r}\Z/n_i\Z$ mit $1<n_r$ und $n_r|n_{r-1}\vdots|n_1$.\\
	ALso gilt für jedes $g\in G\subseteq K^\times$, dass $g$ Nullstelle von $X^{n_1}-1\in K[X]$, Also $\#G\subset n_1$, Also $G\isomorph \Z/n_1\Z$.
\end{proof}

\begin{definition}
	Sei $A$ eine endliche separable $K$-Algebra und sei $a\in A$ mit $A=K[a]$ dann heißt $a$ \textbf{primitves} Element.
\end{definition}

%COUNTER
\addtocounter{thm}{-1}

\begin{theorem}[Satz vom primitiven Element]\label{932thmPrimitiv}
	Sei $A$ eine endliche separable $K$-Algebra. Dann existiert ein primitives Element $a\in A$.
\end{theorem}
\begin{proof}
	Sei $\Omega$ eine algebraisch abegschlossen Erweiterung von $K$ zu $\Hom_{K-\text{Algebra}}(A,\Omega)=\{\varphi_1,...,\varphi_m\}$, $m=[A:K]_S=[A:K]$.
	\begin{enumerate}
		\item Sei $a\in A$. Z.z. $a$ ist primitives Element ist äquivalent $\varphi_i(a)\neq\varphi_j(a)$ für alle $i\neq j$:
		\begin{description}
			\item["$\Rightarrow$"] ist klar, da $a$ Erzeuger von $A$ als $K$-Algebra ist.
			\item["$\Leftarrow$"] Seien $\varphi_i(a)\neq\varphi_j(a)$ für alle $i\neq j$, dann sind auch $\varphi_I|_{K[a]}$ paarweise verschieden.\\
			Also gilt
			\[m\le [K[a]:K]_S\leq[A:K]_S=[A:K]=m\]
			Daraus folgt, dass $[K[a]:K]=[A:K]$ und damit $A=K[a]$.
		\end{description}
		\item Sei $A$ endlich und separabel, $\Leftrightarrow$ ( Übung %TODO Übung
		) $A\isomorph K_1\times...\times K_d$ für endliche separable Erweiterungen $K_i$ von $K$.\\
		Falls $_i=K[a_i]$, so gilt $A=K[a_,...,a_d]$.\\
		Als ist $A=L$ endliche separable Körpererweiterung.
		\item Sei $K$ endlich. Dannn ist $L$ endlich, also $L^\times=\{1=a^0,a,a^2,...\}$ für $a\in L^\times$ (\ref{930satz}).\\
		Dann ist $L=K[a]$.
		\item Sei nun $K$ unendlich, $L=[a_1,...,a_n]$, $a_i\in L$ separabel.\\
		Wir beweisen durch Induktion nach $n$.
		\begin{description}
			\item[$n=1$] Klar.
			\item[$n>1$] $L=K[a_1,...,a_{n-1}][a_n]=K[b,a_n]$. Also gilt OE$L=K[b,c]$
		\end{description}
		\item Z.z. Sei $N:=\{\la\in K\mid \la b+c\text{ nicht primitiv}\}$, dann ist $\#N\leq \frac{m(m-1)}{2}$.
		\begin{align*}
		N&\overset{(1)}{=}\left\{\la\in K\mid \exists i<j:\varphi_i(\la b+b)=\varphi_j(\la b+c)\right\}\\
		&=\bigcup_{1\leq i<j\leq m}\underbrace{\left\{\la\in K|\la(\varphi
		_i(b)-\varphi_j(b))+\varphi_i(c)-\varphi_j(c)=0\right\}}_{\text{hat $\leq 1$ Elemente, da $b,c$ $L$ erzeugen}}
		\end{align*}
		Da $K$ unendlich ist folgt die Behauptung\\
	\end{enumerate}
\end{proof}

%COUNTER
\addtocounter{thm}{-1}
\begin{exm}
	Sei $L=\Q[\sqrt[3]{7},\sqrt{5}]$, $\varphi:L\rightarrow \C$.
	(...) %TODO...
\end{exm}






\section{Galois-Theorie}
\subsection{Galois-Erweiterungen}


\begin{definition}
	Eine algebraische Körpererweiterung $K\hookrightarrow L$ heißt \textbf{Galois-Erweiterung} oder \textbf{galoisch}, falls sie normal und separabel ist.
\end{definition}

\begin{definition}
	Sei $K\hookrightarrow L$ eine Körpererweiterung. Dann ist
	\[\Aut_{K-\text{Algebra}}(L):=\{\sigma:L\rightarrow L,\text{ bijektiver $K$-Algebra-Homomorphismen}\}\]
\end{definition}
\begin{bem*}
	Sei $K\hookrightarrow L$ eine Körpererweiterung. Dann ist $\Aut_{K-\text{Algebra}}(L)$ eine Gruppe bezüglich der Komposition.
\end{bem*}


\begin{exm*}
	\begin{enumerate}
		\item $\Aut_{\Q-\text{Algebra}}(\Q[\sqrt{7}])=\{\id_{\Q[\sqrt{7}]},a+b\sqrt{7}\mapsto a-b\sqrt{7}\}$
		\item $\Aut_{\Q-\text{Algebra}}(\Q[\sqrt[3]{2}])=\{\id\}$
	\end{enumerate}
\end{exm*}

%COUNTER
\addtocounter{thm}{-1}
\begin{definition}
	Sei $K\hookrightarrow L$ eine Galois-Erweiterung. Dann heißt
	\[\Gal(L/K):=\Aut_{K-\text{Algebra}}(L)\]
	\textbf{Galoisgruppe} von $K\hookrightarrow L$.
\end{definition}
%COUNTER
\begin{definition}
	Sei $K\hookrightarrow L$ eine Körpererweiterung und sei $H\subseteq \Aut_{K-\text{Algebra}}(L)$ eine Untergruppe. Dann heißt\[L^H:=\{a\in L\mid\sigma(a)=a,\forall\sigma\in H\}\]
	 der \textbf{Fixkörper} von $H$.
\end{definition}

%VL 16.01.2016
%TODO
\todomark{die VL vom 16.01.2016}
\setcounter{thm}{7}
%VL 18.01.2016
\begin{satz}
	Sei $K\hookrightarrow L$ eine endliche Galoiserweiterung und $M\subseteq L$ ein Zwischenkörper.\\
	Dann ist $K\hookrightarrow L$ normal $\Leftrightarrow$ $\Gal(L/M)\subseteq \Gal(L/K)$ ist Normalteiler.\\
	\\
	In diesem Fall ist die Sequenz
	\begin{align*}
	1\to\Gal(L/M)\to\Gal(L/K)&\to\Gal(M/K)\to 1\\
	\sigma&\mapsto\sigma|_M
	\end{align*}
\end{satz}
\begin{proof}
	Sei $\sigma\in\Gal(L/K)$, $H\subseteq \Gal(L/K)$. Dann ist 
	\begin{align*}
	\sigma(L^H)&=\{\sigma(a)\mid a\in L^H\}\\
	&=\{\sigma(a)\mid\forall\gamma\in H:\gamma(a)=a\}\\
	&\overset{a'=\sigma(a)}{=}\{a'\in L\mid\forall\gamma\in H:\underbrace{\gamma(\sigma^{-1}(a'))=\sigma^{-1}(a')}_{\Leftrightarrow\sigma(\gamma(\sigma^{-1}(a')))=a'}\}\\
	&=L^{\sigma H\sigma^{-1}}
	\end{align*}
	Sei $M=L^H$. Dann ist $K\hookrightarrow L$ normal $\overset{\ref{907}}{\Leftrightarrow}$ für alle $\sigma\in\Gal(L/K)$ gilt $L^{\sigma H\sigma^{-1}}\sigma(M)=M=L^H$.\\
	\\
	Da $H\mapsto L^H$ injektiv ist, folgt
	\begin{align*}
	\text{$K\hookrightarrow M=L^H$}&\Leftrightarrow\forall\sigma\in\Gal(L/K):\sigma H\sigma^{-1}=H\\
	&\Leftrightarrow \text{$H\subseteq \Gal(L/K)$ Normalteiler}
	\end{align*}
	Dann folgt mit \ref{907}, dass $\sigma\mapsto \sigma|_M$ ist surjektiv und $\Ker(\sigma\mapsto \sigma|_M)=\Gal(L/M)$.
\end{proof}

\begin{bem}
	Bestimmung von $L^H$: Sei $K\hookrightarrow L$ eine endliche Galois-Erweiterung, $H\subseteq \Gal(L/K)$.
	\begin{enumerate}
		\item Sei $a\in L$. Setze $Z_a^H:=\{\sigma(a)\mid\sigma\in H\}\subseteq L$. Dann ist
		\[\mu_{a,L^H}=\prod_{b\in Z_a^H}(X-b)\]
		\item Sei $a\in L$ mit $L=K[a]$ und sei $S\subseteq L^H$ die Menge der Koeffizienten von $\mu_{a,L^H}$.\\
		Dann ist $L^H=K[S]$.
	\end{enumerate}
\end{bem}
\begin{proof}
	\begin{enumerate}
		\item Sei $K\hookrightarrow L$ normal Dann zerfällt $\mu_{a,L^H}$ über $L'$ vollständig in Linearfaktoren.\\
		Die Nullstellen von $\mu_{a,L^H}$ sind $\{\sigma(a)\mid\sigma\in\Gal(L/L^H)=H\}$. Es folgt die Behauptung
		\item Es ist klar, dass $K[S]\subseteq L^H$.\\
		Zusätzlich ist $\mu_{a,L^H}$ irreduzibel in $K[S][X]$, also ist $\mu_{a,L^H}=\mu_{a,K[S]}$. Dann ist
		\begin{align*}
		[L:L^H]&\overset{L=K[a]}{=}[L^H[a]:L^H]=\deg \mu_{a,L^H}=\deg \mu_{a,K[S]}\\
		&=[K[S][a]:K[S]]=[L:K[S]]
		\end{align*}
		Es folgt die Behauptung.
	\end{enumerate}
\end{proof}

\begin{exm}
	Sei $g=X^3+a_2X^2+a_1+a_0\in K[X]$ und $\cha(K)\neq 3$.\\
	Substituiere $X\mapsto M_\frac{1}{3}a_2$:
	\[f=X^3+aX+b\in K[X]\]
	Beachte: $f$ ist genau dann irreduzibel wenn $g$ irreduzibel ist. (bzw separabel)\\
	Sei $L$ ein Zerfällungskörper von $f$ (dann ist $K\hookrightarrow L$ normal) .\\
	$f'=3X^2+...\neq 0$, also ist $f$ separabel, also ist $K\hookrightarrow L$ Galois-Erweiterung.\\
	Es gilt $3\leq[L:K]$ und $[L:K]$ teil $3!=6$, also
	\begin{enumerate}
		\item Entweder $[L:K]=3$,
		\item oder $[L:K]=6$
	\end{enumerate}
	$\Gal(L/K)$ ist isomorph zu einer Untergruppe von $S_3$. Also im Fall
	\begin{enumerate}
		\item $\Gal(L:K)\isomorph A_3:=\{\sigma\in S_3\mid\operatorname{sgn}(\sigma)=1\}$
	\item $\Gal(L/K)\isomorph S_3$
	\end{enumerate}
	Seien $a_1,a_2,a_3\in L$ die Nullstellen von $f$. Schriebe
	\[\delta_f=\prod_{i\leq i<j\leq 3}(a_i-a_j)=(a_1-a_2)(a_1-a_3)(a_2-a_3)\]
	$\Delta_f:=\delta_f^2$ heißt die \textbf{Diskriminante} von $f$.\\
	Jedes $\sigma\in \Gal(L/K)$ permutiert die Nullstellen und
	\[\sigma(\delta_f)=\operatorname{sgn}(\sigma)\delta_f\]
	Es folgt $\operatorname{sgn}(\Delta_f)=\Delta_f$.\\
	\\
	Also $\Delta_f\in L^{\Gal(L/K)}=K$.\\
	Es ist $\Delta_f=-4a^3-27b^2$:\\
	und $\delta_f\in K$ genau dann wenn $\Gal(L/K)\isomorph A_3$.\\
	\\
	Fazit: $[L:K]=3\Leftrightarrow\Gal(L:K)\isomorph A_3\Leftrightarrow$ $\Delta_f$ ist Quadrat.
\end{exm}



\section{Anwendung der Galois-Theorie}
\subsection{Endliche Körper}



\begin{bem}
	Sei $K$ ein endlicher Körper.
	\begin{enumerate}
		\item $\cha(K)=p>0$ dann ist $\#K=p^m$ mit $m=K:\F_p$
		\item $K$ ist perfekt. Insbesondere ist jede algebraische Erweiterung $K\hookrightarrow L$ separabel.
	\end{enumerate}
\end{bem}

\begin{satz}\label{1102satz}
	Sei $p$ Primzahl und $\ol{\F_p}$ ein algebraischer Abschluss von $\F_p$. Dann ist für alle $m\in\N$:
	\[K=\{a\in\ol{\F_p}\mid a^{p^m}=a\}\]
	ein Körper mit $p^m$ Elementen.\\
	Jeder Körper mit $p^m$ Elementen ist Zerfällungskörper von $X^p-X\in F_p[X]$. (Dann ist $K,K'$ Körper mit $p^m$ Elementen, $K\isomorph K'$.)\\
	Es gilt: $K$ besteht genau aus den Nullstellen von $X^p-X$.
\end{satz}
\begin{proof}
	Sei $f=X^{p^m}-X$, dann ist $f'=-1$, also ist $f$ separabel.\\
	Es folgt $\#\{a\in\ol{\F_p}\mid a^{p^m}=a\}=p^m$.\\
	Sei $K$ ein beliebiger Körper mit $p^m$ Elementen.\\
	Wähle einen $\F_p$-Algebra-Homomorphimsmus $K\hookrightarrow\ol{\F_p}$ (betrachte $K$ als Unterkörper von $\ol{\F_p}$)\\
	Also $K^\times=\{a\in\ol{F_p}\mid a^{p^m-1}=1\}$. Es folgt $K=\{a\in\ol{\F_p}\mid a^{p^m}=a\}$
\end{proof}

\begin{satz}
	Sei $K$ ein endlicher Körper, sei $q:=\#K$, $K\hookrightarrow L$ eine endliche Erweiterung und $d:=[L:K]$.\\
	Dann ist $K\hookrightarrow L$ Galois-Erweiterung mit $\Gal(L/K)\isomorph\Z/d\Z$ erzeugt von $\varphi:X\mapsto x^q$.
\end{satz}
\begin{proof}
	Aus \ref{1102satz} folgt, dass $K\hookrightarrow L$. Dann ist
	\begin{align*}
	L&=\{a\in\ol L\mid a^{q^d}=a\}\\
	K&=\{a\in\ol L\mid a^q=a\}
	\end{align*}
	Also $\varphi\in\Gal(L/K)$ hat Ordnung $d$, und dann $\Gal(L/K)$ ist zyklisch.
\end{proof}




%VL 23.01.2017
\subsection{Zyklische Erweiterungen}
\begin{definition}
	Sei $n\in \N$. Ein $\xi\in K$ heißt $n$-te \textbf{Einheitswurzel}, falls $\xi^n=1$.
	Definiere $\mu_{n}(K):=\{\xi\in K\mid \xi^n=1\}\subseteq K^\times$ als Menge der $n$-ten Einheitswurzeln von $K$.
\end{definition}
\begin{bem*}
	$\mu_{n}(K)$ ist Untergruppe von $(K^\times,\cdot)$.
\end{bem*}

\begin{bem}
	Sei $n\in \N$. Definiere $m:=n$, falls $\cha(K)=0$.\\
	Falls $\cha(K)=p>0$ schreibe $n=p^rm$ ($r\in\N_0$) mit $m$ teilerfrems zu $p$.
	\begin{enumerate}
		\item $\mu_n(K)=\mu_m(K)$
		\item $\mu_n(K)$ ist endlich erzeugte zyklische Gruppe und $\#\mu_n(K)$ teilt $m$.
		\item Ist $K$ algebraisch abgeschlossen, dann ist $\#\mu_n(K)$
	\end{enumerate}
\end{bem}
\begin{proof}
	\begin{enumerate}
		\item $\mu_n(K)=\{\text{Nullstellen von $X^n-1$ in $K$}\}$. Nun gilt
		\[X^n-1=(X^m)^{p^r}-1=(X^m-1)^{p^r}\]
		Also gilt $\mu_n(K)=\{\text{Nullstellen von $X^m-1$ in $K$}\}=\mu_m(K)$.
		\item $\mu_n(K)$ ist endlich, da $X^n-1$ nur endlich viele Nullteiler hat.\\
		Dann folgt mit \ref{925}, dass $\mu_n(K)$ zyklisch ist.\\
		Sei $\ol K$ algebraischer Abschluss. Dann hat $X^m-1$ genau $m$ Nullstellen, da $X^m-1$ separabel ist.
		(Denn $mX^{m-1}\neq 0$ teilerfremd zu $X^m-1$).\\
		Also ist $\mu_n(\ol K)=\mu_n(\ol K)$ und hat Ordnung $m$.\\
		Da $\mu_n(K)\subseteq \mu_n(\ol K)$ Untergruppe ist, folgt $\#\mu_m(K)$ teilt $m$.
	\end{enumerate}
\end{proof}

\begin{definition}
	Sei $n\in\N$. Eine $n$-te Einheitswurzel $\xi\in K$ heißt \textbf{primitv}, falls $\ord(\xi)=n$.
\end{definition}
\begin{exm}
	\begin{enumerate}
		\item Sei $K=\C$.
		\[\mu_n(\C)=\left\{e^\frac{2\pi i k}{n}\mid k\in\Z/n\Z\right\}\isomorph Z/n\Z\]
		$e^{\frac{2\pi i k}{n}}$ ist genau dann primitiv, wenn $k$ teilerfremd zu $m$ ist. Genau dann wenn $k\in(\Z/n\Z)^\times$
		\item Sei $K=\Q$
		\[\mu_m(\Q)=\mu_m(\R)=\begin{cases}
		\{+1,-1\}&\text{$n$ ist gerade}\\
		\{1\}&\text{$n$ ist ungerade}
		\end{cases}\]
		\item Sei $q$ Primzahlpotenz. Dann
		\[\mu_{q-1}(\F_q)=\F_q^\times\]
	\end{enumerate}
\end{exm}

\begin{definition}
	Die Abildung 
	\begin{align*}
	\varphi:\N_0&\to\N\\
	\varphi(n)&\mapsto\#(\Z/n\Z)=\#\{0\leq k<n-1\mid\text{$k$ teilerfrems zu $n$}\}
	\end{align*}
	heißt \textbf{Eulersche $\varphi$-Funktion}
\end{definition}
\begin{prop}\label{1109prop}
	\begin{enumerate}
		\item Seien $m,n\in\N$ teilerfremd, dann
		\[\varphi(mn)=\varphi(m)\varphi(n)\]
		\item Sei $p$ Primzahl, $l\in \N$. Dann ist
		\[\varphi(p^l)=p^l-p^{l-1}=(p-1)p^{l-1}\]
	\end{enumerate}
\end{prop}
\begin{proof}
\begin{enumerate}
	\item Es gilt:
	\[\varphi(mn)=\#(\Z/mn\Z)^\times=\#((\Z/m\Z)\times (\Z/n\Z)^\times)=\varphi(m)\varphi(n)\]
	\item $\varphi(p^l)=\#\{0\leq <p^l\mid\text{$p$ teilt nicht $k$}\}=p^l-^{l-1}$ \\
	und $p^{l-1}=\#\{0\leq k<p^l\mid \text{$p$ teilt nicht $k$} \}$
\end{enumerate}
\end{proof}

\begin{exm*}
	\begin{align*}
	\varphi(1200)&=\varphi(3\cdot2^4\cdot 5^2)\\
	&=\varphi(3)\cdot\varphi(2^4)\cdot\varphi(5^2)\\
	&=(3-1)(2^4-2^3)(5^2-5^1)\\
	&=2\cdot8\cdot 20=2^6\cdot 5
	\end{align*}
\end{exm*}

%TODO Notation?

\begin{satz}\label{1110satz}
	Die Körpererweiterung $K\hookrightarrow K[\zeta]$ ist endlich und galoisch. Die Abbildung
	$$\al:\Gal(K[\zeta]/K)\to(\Z/n\Z)^\times$$,$\sigma\mapsto a_\sigma$, wobei $\sigma(\zeta)=\zeta^{a_0}$ ist wohldefinierter injektiver Gruppen-Homomorphimsmus.\\
	Insbesondere $[K[\zeta]:K]:\#\Gal(K[\zeta]/K)$ teilt $\varphi(n)$.
\end{satz}
\begin{proof}
	\begin{itemize}
		\item $K\hookrightarrow K[\zeta]$ ist separabel, denn $\mu_{\zeta,K}$ teilt $X^n-1$ und $X^n-1$ ist separabel, da $n\in K^\times$.\\
		$K[\zeta]$ ist Zerfällunskörper von $X^n-1$, also ist $K\hookrightarrow K[\zeta]$ normal.
		\item Z.z $\al$ ist wohldefiniert (insbesondere $a_0\in(\Z/n\Z)^\times$):\\
		Da Gruppen-Automorphismen die Ordnung erhalten ist $\sigma(\zeta)$ primitv. Also $\sigma(\zeta=\zeta^{a_0})$ Einheint von $\Z/n\Z$.
		\item Z.z. $\al$ ist Gruppen-Homomorphimsmus:\\
		Seien $\sigma,\tau\in\Gal(K[\zeta]/K)$. Dann ist
		\[\tau(\sigma(\zeta))=\tau(\zeta^{a_0})=\zeta^{a_\tau a_\sigma}\]
		Es folgt, dass
		\[\al(\tau\sigma)=a_\tau a_\sigma=\al(\tau)\al(\sigma)\]
		\item Z.z. $\Ker(\al)=\{\id\}$:\\
		Sei $\sigma\in\Ker(\al)$ ist äquivalent $\sigma(\zeta)=\zeta$.\\
		Dann ist $\sigma=\id$.
	\end{itemize}
\end{proof}


\begin{theorem}\label{1111th}
	Sei $K=\Q$, $\zeta\in\C$ primitve $n$-te Einheitswurzel. Dann ist $\Q\hookrightarrow\Q$ eine endliche Galois-Erweiterung und $\Gal(\Q[\zeta]/\Q)\isomorph(\Z/n\Z)^\times$\\
	Insbesondere ist $[\Q[\zeta]:Q]=\varphi(n)$.
\end{theorem}
\begin{proof}
	Sei $0\leq r<n$ mit $(r,n)=1$
	\begin{itemize}
		\item Z.z. Es existiert $\sigma\in G:=\Gal(\Q[\zeta]/\Q)$ mit $\sigma(\zeta)=\zeta^r$.\\
		Sei $r=p_1p_2...p_s$ Primfaktoerzerlegung, $p_i$ Primzahlen mit $(p_i,n)=1$.\\
		Falls ein $\sigma_i\in G$ mit $\sigma_i(\zeta)=\zeta^{p_i}$. Dann schreiben $\sigma=\sigma_1\sigma_2...\sigma_s$.\\
		Dann genügt es zu zeigen, dass $f:=\mu_{\zeta,\Q}=\mu_{\zeta^p,\Q}=:g$ für $p$ Primzahl mit $(p,n)=1$.\\
		\\
		Sei $f\neq g$, dann $X^n-1=fgu$ mit normiertem $u\in\Q[X]$. \\
		Mit \ref{704Gauss}(?Lemma) $f,g,u\in\Z[X]$.\\
		Reduktion modulo $p$ ergibt
		\[X^n-1=\ol f\ol g\ol u\in\F_p[X]\] %Als \star bezeichnen
		$\ol g(X^p)=\ol g(X^p)=\ol f$. Sei $v$ ein irreduzibbler Teiler von $\ol g$.\\
		Dann ($\star$) gilt $v^2$ teilt $X^n-1$ in $\F_p[X]$. Dann ist also $X^n-1$ nicht separabel in $\F_p[X]$. Widerspruch zu $(p,n)=1$.\\
		Also ist $\al$ separabel und mit \ref{1110satz} folgt die Behauptung.
		
	\end{itemize}
\end{proof}

%TODO... "Verstehen endlicher Erweiteungen"

\begin{satz}[Satz von Konecker Weber]\label{1112kronweb}
	Sei $\Q\hookrightarrow L$ eine endliche Galois-Erweiterung mit abelscher Galoisgruppe.\\
	Dann existiert $n\in\N$ und primitive $n$-te Einheitswurzeln $\zeta\in\C$ und eine Einbettung $L\hookrightarrow\Q[\zeta]$
\end{satz}
\begin{proof}
	Nicht im Zeitrahmen der Vorlesung
\end{proof}


\subsection{Konstruktion mit Zirkel und Lineal}
\begin{definition}
	Sei $M\subseteq \C$ Teilmenge.\\
	Dann heißst $z\in\C$ \textbf{mit Zirkel und Lineal aus $M$ konstruierbar}, falls $z$ durch endlich viele Elementarkonstruktionen aus Elementen von $M$ konstruierbar ist.\\
	\\
	Als \textbf{Elementarfunktionen} aus $S$ bezeichnet man
	\begin{enumerate}
		\item Schnittpunkte von zwei Geraden die jeweils durch Punkte in $S$ gegeben sind.
		\item Schnittpunkte von einer Geraden $y$ durch zwei Punkte in $S$ und einem Kreis mit Mittelpunkt in $S$ und einem Radius der der Entfernung von zwei punkten in $S$ entspricht
		\item Schnittpunkt von Kreisen wie in (2).
	\end{enumerate}
%VL 25.01.2016
	\[K(M):=\{z\in\C\mid \text{$z$ kann aus $M$ mit Zirkel und Lineal konstruiert werden}\}\]
\end{definition}

\begin{theorem}\label{1114thm}
	Seien $0,1\in M$. Setze $\ol M:=\{\ol z\mid z\in M\}$.
	\begin{enumerate}
		\item $\Q(M\cup \ol M)\hookrightarrow K(M)$ ist eine algebraische Körper-Erweiterung.
		\item Für $z\in\C$ sind äquivalent:
		\begin{enumerate}
			\item $z\in K(M)$
			\item $z$ ist enthalten in einer Galois-Erweiterung $L$ von $\Q(M\cup \ol M)$, sodass $[L:\Q(M\cup\ol M)]=2^n$ für ein $n\in\N$.
		\end{enumerate}
	\end{enumerate}
\end{theorem}
\begin{proof}
	Bosch, Algebra, §6.3
\end{proof}

\begin{exm}
	\begin{enumerate}
		\item Quadratur des Kreises: Es ist nicht möglich aus einem Kreis mit Radius 1 ein Quadrat mit gleichem Flächeninhalt mit Zirkel und Lineal zu Konstruieren.\\
		Äquivalent: $\sqrt{\pi}\notin K(\{0,1\})$.\\
		\begin{proof}
			Durch ``Satz von Lindemenn'': $\pi$ ist transzendent über $\Q$.
		\end{proof}
		\item Verdopplung des Würfels: Es ist nicht möglich aus einem Würfel $W$ mit Kantenlänge 1 einen Würfel $W'$ zu konstruieren, sodass $\operatorname{vol}(W')=2\operatorname{vol}(W)$.\\
		Äquivalent: $\sqrt[3]{2}\notin K(\{0,1\})$.
		\begin{proof}
			Angenommen $\sqrt[3]{2}\in K(\{0,1\})$, dann $\sqrt[3]{2}\in L$ mit $[L:\Q]=2^n$ für ein $n\in\N$.\\
			Aus $\Q[\sqrt[3]{2}]\subseteq L$ folgt jedoch $[L:Q]$ ist durch $3$ teilbar. Widerspruch!
		\end{proof}
		\item Konstruktion eines regelmäßigen $n$-Ecks: Das in den Einheitskreis eingeschlossenen regelmäßige $n$-Eck mit Ecke $1$ ist genau dann in $K(\{0,1\})$, wenn $\varphi(n)$ eine $2$-er Potenz ist.\\
		Äquivalent: $e^{\frac{2\pi i}{n}}\in K(\{0,1\})\Leftrightarrow [Q[e^{2\pi i/n}:\Q]]=2^k$ für ein $k\in\N$.\\
		Sei $n=p_q^{l_1}...p_r^{l_r}$, dann $\varphi(n)=(p_1^{l_1}-p_1^{l_1-1})...(p_r^{l_r}-p_r^{l_r-1})$.\\
		$\varphi(n)$ ist $2$-er Potenz $\Leftrightarrow $ $n=2^kp_1...p_s$, wobei $p_i$ verschiedene Primzahlen von der Form $2^{r_i}+1$ für $r_i\in\N$.
	\end{enumerate}
\end{exm}





\subsection{Auflösbare Erweiterungen}
\begin{definition}
	Eine Körpererweiterung $K\hookrightarrow L$ heißt \textbf{durch Radikale auflösbar}, falls es eine Kette von Erweiterungen $K=K_0\subseteq K_1\subseteq...\subseteq K_r=L$ gibt, sodass $K_i=K_{i-1}[a_i]$, wobei:
	\begin{enumerate}
		\item Falls $\cha(K)=0$: $a_i=\sqrt[n_i]{\alpha_i}$, $\alpha_i\in K_{i-1}$, $n_i\in\N$.
		\item Falls $\cha(K)=p>0$: $a_i=\sqrt[n_i]{\alpha_i}$, $\alpha_i\in K_{i-1}$, $n_i\in\N$ mit $(n_i,p)=1$\\
		Oder $a_i$ ist Nullstelle eines Polynoms der Form $X^p-X-\al_i$, $\al_i\in K_{i-1}$.
	\end{enumerate}
\end{definition}

\begin{definition}
	Eine Gruppe $G$ heißt \textbf{auflösbar}, falls es eine Kompositionsreihe (Kette von Untergruppen) $\{e\}=G_0\subseteq g_1\subseteq...\subseteq G_r=G$ gibt, sodass $G_i/G_{i-1}$ abelsch ist für alle $i=1,...,r$.
\end{definition}

\begin{bem}
	Sei $G$ auflösbar und $H\subseteq G$ Untergruppe, dann ist $H$ auflösbar.
\end{bem}

\begin{exm}
	Die Symmetrische Gruppe $S_n$ ist genau dann auflösbar, wenn $n\leq 4$.
\end{exm}

\begin{definition}
	Eine endliche separable Erweiterung von Körpern $K\hookrightarrow L$ heißt \textbf{auflösbar}, wenn Erweiterung $L\hookrightarrow M$ existiert, sodass $K\hookrightarrow L$ galoisch und mit ausflösbarer Galois-Gruppe ist.
\end{definition}
\begin{proof}
	Man kann $M$ als normale Hülle von $K\hookrightarrow L$ wählen.
\end{proof}

\begin{theorem}\label{1121thm}
	Sei $K\hookrightarrow L$ endlich separabel.\\
	Dann ist $K\hookrightarrow K$ genau dann durch Radikale auflösbar, wenn $K\hookrightarrow L$ auflösbar ist.
\end{theorem}
\begin{proof}
	Lang, Algebra, VI, §7
\end{proof}

\begin{definition}
	Sei $f\in K[X]$ normiert und separabel. Dann heißt $f$ \textbf{durch Radikale auflösbar}, falls der Zerfällungskörper $L$ von $f$ eine auflösbare Erweiterung von $K$ ist.
\end{definition}
%COUNTER
\addtocounter{thm}{-1}
\begin{kor}\label{1122kor}
	$f\in K[X]$ ist genau dann durch Radikale auflösbar, wenn $\Gal(L/K)$ auflösbar ist.
\end{kor}

\begin{exm}
	Sei $\cha(K)=0$, $f\in K[X]$ irreduzibel und $n:=\deg(f)$.\\
	Sei $L$ Zerfällungskörper von $f$. Dann ist $\Gal(L/K)$ isomorph zu Untergruppen von $S_n$.\\
	Also folgt aus $n\leq 4$, dass $f$ durch Radikale auflösbar ist.\\
	Falls $n\geq 5$ und $\Gal(L/K)\isomfunc S_n$, dann ist $f$ nicht durch Radikale auflösbar.
\end{exm}

\begin{exm}
	Sei $\cha(K)=0$, $\deg(f)=3$. Sei OE $f=X^2+3pX+2q$, mit $p,q\in K$.\\
	Berechnung der Nullstellen von $f$:
	\[u:=\sqrt[3]{-q+\sqrt{q^2+p^3}},\quad v:=\sqrt[3]{-q-\sqrt{q^2+p^3}}\]
	So dass $uv=-p$.\\
	Dann sind die Nullstellen von $f$:
	\[x_1=u+p,\quad x_2=\zeta u+\zeta^2 v\,\quad x_3:=\zeta^2u+\zeta v\]
	Wobei $\zeta$ primitive $3$-te Einheitswurzel.
	(In einem Zerfällungskörper von $f$)
\end{exm}
\begin{proof}
	Van der Waerden: Algebra(?)
\end{proof}






\section{Endlich erzeugt Algebren über Körper}
(Oder: Lineare Algebra)


\subsection{Hilbertscher Nullstellensatz}

\end{document}